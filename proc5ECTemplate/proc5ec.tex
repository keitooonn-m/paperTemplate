%#!pdfpLaTeX
%
% 北村研究室用卒論予稿のTeXテンプレートファイル
% 本ファイルは非公式であり,下記で公開されているワードのテンプレートが公式である.
% https://www.kagawa-nct.ac.jp/EE/local/index.html (学内限定アクセス)
%
% 2020年1月18日 北村大地作成
%

\documentclass[a4j]{jsarticle}
\usepackage{nitkagawaproc5ec} % 予稿テンプレートクラスファイル
\usepackage{amsmath,amssymb} % 数式環境
\usepackage{bm} % 数式の太字斜体
\usepackage[dvipdfmx]{color}
\usepackage[dvipdfmx]{graphicx}
\usepackage{flushend} % 2段組みの最終ページの高さを揃える
\usepackage{url}

\renewcommand{\baselinestretch}{0.8} % 行間設定(標準は0.8)

%%%%%%%%%%%%%%%%%%%%%%%%%%% 論文情報 %%%%%%%%%%%%%%%%%%%%%%%%%%%

%%%%% タイトル %%%%%
\title{レーダセンサ及びブラインド信号源分離に\\基づく心拍推定}
\etitle{Heart Rate Estimation Based on Radar Sensor and Blind Source Separation}

%%%%% 著者 %%%%%
\author{村田 佳斗}
\eauthor{Keito Murata}


%----- 図表題が英語の場合は次の2行を有効化 -----%
\renewcommand{\figurename}{Fig.~} 
\renewcommand{\tablename}{Table~}
%--------------------------------------------%

\begin{document}
\maketitle% タイトル生成

%----- ハイフン付きページ番号を表示する場合は次3行を有効化 -----%
%\setcounter{page}{1} % 開始ページ番号(3にすると3ページと4ページの2枚になる)
%\pagestyle{hyphenpage}
%\thispagestyle{hyphenpage}
%---------------------------------------------------------%

%----- ハイフン無しページ番号を表示する場合は次3行を有効化 -----%
%\setcounter{page}{1} % 開始ページ番号(3にすると3ページと4ページの2枚になる)
%\thispagestyle{plain}
%\pagestyle{plain}
%---------------------------------------------------------%

%----- ページ番号を削除する場合は次2行を有効化 -----%
\thispagestyle{empty}
\pagestyle{empty}
%-----------------------------------------------%

%%%%%%%%%%%%%%%%%%%%%%%%%%% 本文 %%%%%%%%%%%%%%%%%%%%%%%%%%%
%----------------------------------------------
\section{はじめに}
%----------------------------------------------

自動車の運転中に運転者が睡眠,突発的な発作,体調の悪化による意識喪失等に見舞われることは多くの場合致命的な状況となる.そのため,運転中に運転者の状態を何らかの方法で管理することが重要課題の一つとなっている.

この課題に取り組むために,本稿では,Fig.~\ref{fig:sensorstructure}に示す運転中の運転者の心拍をレーダ非接触型生体センサアレイ(以後,レーダセンサと呼ぶ)を活用した常時モニタリングシステム(以後,振動測定系と呼ぶ)を取り扱う.この振動測定系においては,目的としている心拍信号以外にも振動測定系自体のノイズ,運転者の体動,呼吸による体表面変動等の信号も同時に観測されるため,信号対ノイズ比(SNR)を著しく低下する.本稿の目的は,観測信号が低SNRとなる問題に対してブラインド信号源分離(BSS)\cite{originica}を適用することで,心拍の推定精度の向上を目指すことである.本稿で仮定する状況では,レーダセンサや体表面の位置関係が事前に未知であるため,心拍信号の分離にBSSを適用することは妥当と考えられる.また,心拍推定アルゴリズムを適用して心拍推定を行い心拍推定精度についても同様に検討を行う.

%----------------------------------------------
\section{振動測定系と測定条件}
%----------------------------------------------

%----------------------------------------------
\subsection{振動測定系}
%----------------------------------------------

本稿で仮定するレーダセンサの振動測定系はFig.~\ref{fig:sensorstructure}を示す.運転者を模した被験者が座った状態で振動測定系全体を上下方向に振動させる.Fig.~\ref{fig:sensorimg}に示す通り,レーダセンサを背部にあたるシートの内部に埋め込んでいるため,運転者の背部の体表面の微小変位を測定することが可能である.またFig.~\ref{fig:sensorimg}に示すように,4チャネルの異なる指向性レーダで駆動しているため,同時に近傍4点の体表面変位を測定した4~chの観測信号を得ることができる.
%-%-%-%-%-%-%-%-%
\begin{figure}[tb]
  \centering
  \vspace{0pt} % 図上部の余白調整
  \includegraphics[width=0.8\columnwidth]{sensorsystem.pdf}
  \vspace{0pt} % 図とキャプション間の余白調整
  \caption{(a) Vibration measurement system and driver's seat with radar sensor and (b) Beams from radar sensor for measurering back of driver's body.}
  \vspace{0pt} % キャプション下部の余白調整
  \label{fig:sensorstructure}
\end{figure}
%-%-%-%-%-%-%-%-%

%-%-%-%-%-%-%-%-%
%\begin{figure}[t]
%  \centering
%  \vspace{0pt} % 図上部の余白調整
%  \includegraphics[width=0.8\columnwidth]{sensorimg.pdf}
%  \vspace{0pt} % 図とキャプション間の余白調整
%  \caption{Beams from radar sensor for measurering back of driver's body.}
%  \vspace{-20pt} % キャプション下部の余白調整
%  \label{fig:sensorimg}
%\end{figure}
%-%-%-%-%-%-%-%-%

%----------------------------------------------
\subsection{レーダセンサの観測スペクトログラム}
%----------------------------------------------

%本研究では,計測時間が420~sで,計測開始から60~sは振動を加えない状態で計測し,その後の300~sは振動を加えた状態,残りの60~sは再度,振動を加えていない状態で計測している.振動台の振動は上下方向に加えている.振動台の振幅及び周波数はそれぞれ10~mm及び1.2~Hzとしてい
レーダセンサで観測された信号の内1~chのスペクトログラムをFig.~\ref{fig:1chobs}に示す.観測信号には心拍以外にも,振動台による振動成分及び呼吸に起因する体動成分等が混在していることが確認できる.そのため,本稿ではBSSを用いて心拍成分のみを抽出することを目指す.

また本稿では,分離した信号と比較する参照値を得るために,Zephyr Technology社の接触型心電図センサ(ECG)Bioharnessを用いてECG信号を取得している.心拍の参照値の計算には,このBioharness内部で実装されているアルゴリズムを用いる.Bioharnessの技術的な資料は公開されておらず原理は不明であるが,恐らく一般的な心拍推定アルゴリズムであるR-R間隔(R波と呼ばれる心拍波形の間隔)推定に基づくものと予想される.また,このECGセンサのサンプリング周波数は250~Hzである.接触型ECGセンサであるため,振動台の振動が加えられても高精度な心拍を得ることが可能である.本稿では,このECGセンサから得られる心拍と同程度の精度でレーダセンサの信号から心拍を推定することが目的となる.

%-%-%-%-%-%-%-%-%
\begin{figure}[t]
  \centering
  \vspace{0pt} % 図上部の余白調整
  \includegraphics[width=1.0\columnwidth]{1chobsspect.pdf}
  \vspace{-20pt} % 図とキャプション間の余白調整
  \caption{Each component in first channel spectrogram obtained by radar sensor.}
  \vspace{-20pt} % キャプション下部の余白調整
  \label{fig:1chobs}
\end{figure}
%-%-%-%-%-%-%-%-%

%----------------------------------------------
\section{BSS及び心拍推定アルゴリズム}
%----------------------------------------------

%----------------------------------------------
\subsection{$t$-ILRMA}
%----------------------------------------------
音響信号におけるBSSとして独立低ランク行列分析(ILRMA)~\cite{ILRMA}が提案されている.ILRMAは,各信号源が低ランクな時間周波数構造(同じスペクトルを多くの時間で含む)を持つと仮定しながら周波数ごとの分離行列を推定することでBSSを実現する.本稿で適用する$t$-ILRMA~\cite{tdist1}は,各信号源の低ランク時間周波数モデルに複素$t$分布を用いることで,幅広いタイプの信号源が取り扱えるように一般化した手法である.なお本稿では,呼吸に由来する成分の除去を目的として,$t$-ILRMAを適用する前にカットオフ周波数1.5~Hzのハイパスフィルタを適用している.

%----------------------------------------------
\subsection{心拍推定アルゴリズム}
%----------------------------------------------
一般的な心拍推定アルゴリズムにはR-R間隔推定が用いられるが,レーダセンサの観測信号にBSSを適用して得られる分離心拍信号にはR波が顕著に現れない.そこで,本稿では下記に示す心拍推定アルゴリズムを適用する.

まず,心拍信号の調波構造を強調しつつ低周波成分を除去するために,心拍分離信号に二階微分フィルタ及び半波整流を適用する.次に,心拍信号の基本周波数成分を強調するために0.7~Hz--1.4~Hzを通過帯域とするバンドパスフィルタを適用する.最後に,短時間フーリエ変換を適用してスペクトログラムに変換し,振幅のピークとなる周波数を時間フレーム毎に求めることで,推定心拍値を得る.

%----------------------------------------------
\section{実験結果}
%----------------------------------------------
$t$-ILRMAを適用して得られる分離信号のスペクトログラムをFig.~\ref{fig:stilrmaa5}に示す.3番目の分離信号を見ると,心拍の高調波成分が2.5~Hz,3.5~Hz,5~Hz,及び6~Hz付近に分離されていることがわかる.1番目,2番目,及び4番目の分離信号には心拍成分が見られず,振動台による振動及び呼吸の高調波成分が分離されていることが分かる.しかし,振動が加えられる60~sでは振動成分が残留していることが分かる.これは,振動が加えられた瞬間の体動は大きくなるためであると考えられる.

Fig.~\ref{fig:stilrmaa5}の3番目の分離信号に対して心拍推定アルゴリズムを適用した心拍推定値をFig.~\ref{fig:hrtilrmaa5}に示す.振動が加えられた60~s--100~sでは参照値と合致していないが,その他の時間においては参照値と合致していることがわかる.

%-%-%-%-%-%-%-%-%
\begin{figure}[t]
  \centering
  \vspace{0pt} % 図上部の余白調整
  \includegraphics[width=1.0\columnwidth]{spect_tILRMA_ampdom_dofp5_64_est.pdf}
  \vspace{-20pt} % 図とキャプション間の余白調整
  \caption{Spectrograms of separated signal obtained by $t$-ILRMA (high-pass-filtered).}
  \vspace{0pt} % キャプション下部の余白調整
  \label{fig:stilrmaa5}
\end{figure}
%-%-%-%-%-%-%-%-%

%-%-%-%-%-%-%-%-%
\begin{figure}[t]
  \centering
  \vspace{-10pt} % 図上部の余白調整
  \includegraphics[width=0.8\columnwidth]{hr_tILRMA_ampdom_dofp5_64_ch3.pdf}
  \vspace{-10pt} % 図とキャプション間の余白調整
  \caption{Estimated (red) and reference (blue) heart rates obtained by $t$-ILRMA.}
  \vspace{-20pt} % キャプション下部の余白調整
  \label{fig:hrtilrmaa5}
\end{figure}
%-%-%-%-%-%-%-%-%

%----------------------------------------------
\section{結論}
%----------------------------------------------

本稿では,レーダセンサから得られる観測信号にBSSの手法の一つである$t$-ILRMAを適用し,心拍信号の分離を行った.実験結果より加振時に一時的に参照値と合致していないが,その他の時間では概ね合致していることを確認した.

%%%%% 参考文献 %%%%%
\begin{thebibliography}{99}% 10以上の文献数であれば99とする
%参考文献のフォントサイズを小さくしたい場合は下記2行のコメントアウトを解除
\itemsep 3pt % 項目の間隔微調整用
\fontsize{8pt}{10pt}\selectfont  % 項目のフォントサイズ微調整用

\bibitem{originica}
P. Comon,``Independent component analysis, A new concept?,'' {\em Signal Process.}, vol. 36, no. 3, pp. 287--314, 1994.

\bibitem{isnmf}
C.~Févotte, N.~Bertin, and J.-L.~Durrieu, 
``Nonnegative matrix factorization with the Itakura-Saito divergence. With application to music analysis.'' 
{\em Neural Comput.}, vol.~21, pp. 793--830, 2009.

\bibitem{ILRMA}
D.~Kitamura, N.~Ono, H.~Sawada, H.~Kameoka, and H.~Saruwatari,
``Determined blind source separation unifying independent vector analysis and nonnegative matrix factorization,'' 
{\em IEEE/ACM Trans. ASLP}, vol.~24, no.~9, pp.~1626--1641, 2016.

\bibitem{tdist1}
S.~Mogami, D.~Kitamura, Y.~Mitsui, N.~Takamune, H.~Saruwatari, and N.~Ono, ``Independent low-rank matrix analysis based on complex Student's t-distribution for blind audio source     separation,''
{\em Proc.~MLSP}, 2017. 

\end{thebibliography}

\end{document}