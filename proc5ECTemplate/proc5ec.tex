%#!pdfpLaTeX
%
% 北村研究室用卒論予稿のTeXテンプレートファイル
% 本ファイルは非公式であり,下記で公開されているワードのテンプレートが公式である.
% https://www.kagawa-nct.ac.jp/EE/local/index.html (学内限定アクセス)
%
% 2020年1月18日 北村大地作成
%

\documentclass[a4j]{jsarticle}
\usepackage{nitkagawaproc5ec} % 予稿テンプレートクラスファイル
\usepackage{amsmath,amssymb} % 数式環境
\usepackage{bm} % 数式の太字斜体
\usepackage[dvipdfmx]{color}
\usepackage[dvipdfmx]{graphicx}
\usepackage{flushend} % 2段組みの最終ページの高さを揃える
\usepackage{url}

\renewcommand{\baselinestretch}{0.8} % 行間設定(標準は0.8)

%%%%%%%%%%%%%%%%%%%%%%%%%%% 論文情報 %%%%%%%%%%%%%%%%%%%%%%%%%%%

%%%%% タイトル %%%%%
\title{レーダセンサ及びブラインド信号源分離に\\基づく心拍推定}
\etitle{Heart Rate Estimation Based on Radar Sensor and Blind Source Separation}

%%%%% 著者 %%%%%
\author{村田 佳斗}
\eauthor{Keito Murata}


%----- 図表題が英語の場合は次の2行を有効化 -----%
\renewcommand{\figurename}{Fig.~} 
\renewcommand{\tablename}{Table~}
%--------------------------------------------%

\begin{document}
\maketitle% タイトル生成

%----- ハイフン付きページ番号を表示する場合は次3行を有効化 -----%
%\setcounter{page}{1} % 開始ページ番号(3にすると3ページと4ページの2枚になる)
%\pagestyle{hyphenpage}
%\thispagestyle{hyphenpage}
%---------------------------------------------------------%

%----- ハイフン無しページ番号を表示する場合は次3行を有効化 -----%
%\setcounter{page}{1} % 開始ページ番号(3にすると3ページと4ページの2枚になる)
%\thispagestyle{plain}
%\pagestyle{plain}
%---------------------------------------------------------%

%----- ページ番号を削除する場合は次2行を有効化 -----%
\thispagestyle{empty}
\pagestyle{empty}
%-----------------------------------------------%

%%%%%%%%%%%%%%%%%%%%%%%%%%% 本文 %%%%%%%%%%%%%%%%%%%%%%%%%%%
%----------------------------------------------
\section{はじめに}
%----------------------------------------------

自動車の運転中に運転者が睡眠,突発的な発作,体調の悪化による意識喪失等に見舞われることは多くの場合致命的な状況となる.そのため,運転中に運転者の状態を何らかの方法で管理することが重要課題の一つとなっている.この課題に取り組むために,本研究では,Fig.~\ref{fig:sensorstructure}に示す運転中の運転者の心拍をレーダ非接触型生体センサアレイ(以後,レーダセンサと呼ぶ)を活用した常時モニタリングシステム(以後,振動測定系と呼ぶ)を取り扱う.しかし,この振動測定系においては,目的としている心拍信号以外にも振動測定系自体のノイズ,運転者の体動,呼吸による体表面変動等の信号も同時に観測されてしまう.本稿では,これらのノイズから心拍信号のみを分離するためにブラインド信号源分離(BSS)\cite{originica, ica2}を適用し,心拍推定アルゴリズムを用いた心拍推定について検討する.

%----------------------------------------------
\section{振動測定系と測定条件}
%----------------------------------------------

%----------------------------------------------
\subsection{振動測定系}
%----------------------------------------------

本研究では,Fig.~\ref{fig:sensorstructure}に示す運転中の運転者の心拍をレーダ非接触型生体センサアレイ(以後,レーダセンサと呼ぶ)を活用した常時モニタリングシステムを取り扱う.レーダセンサでは,運転者の体表面の微小変位を測定することができる.またFig.~\ref{fig:sensorimg}に示すように,1つのレーダセンサは4チャネルの異なる指向性レーダで駆動しているため,同時に近傍4点の体表面変位を測定することが可能となっている.

%-%-%-%-%-%-%-%-%
\begin{figure}[tb]
  \centering
  \vspace{0pt} % 図上部の余白調整
  \includegraphics[width=0.8\columnwidth]{sensorstructure.pdf}
  \vspace{0pt} % 図とキャプション間の余白調整
  \caption{Vibration measurement system and driver's seat with radar sensors.}
  \vspace{0pt} % キャプション下部の余白調整
  \label{fig:sensorstructure}
\end{figure}
%-%-%-%-%-%-%-%-%

%-%-%-%-%-%-%-%-%
\begin{figure}[t]
  \centering
  \vspace{0pt} % 図上部の余白調整
  \includegraphics[width=0.8\columnwidth]{sensorimg.pdf}
  \vspace{0pt} % 図とキャプション間の余白調整
  \caption{Beams from radar sensor for measurering (a) back and (b) bottom surfaces of driver's body.}
  \vspace{0pt} % キャプション下部の余白調整
  \label{fig:sensorimg}
\end{figure}
%-%-%-%-%-%-%-%-%

%----------------------------------------------
\subsection{振動測定条件}
%----------------------------------------------

計測時間は420~sで,計測開始から60~sは振動は加えていない状態で計測し,その後の300~sは振動を加えた状態,残りの60~sは再度,振動を加えていない状態で計測している.60~s--360~sの間はFig.~\ref{fig:sensorstructure}に示している振動方向のうち,上下方向に振動を加えている.また振動台の振動の振幅及び周波数は,10~mm及び1.2~Hzとしている.

また本研究では,分離した信号と比較する参照値を得るために,Zephyr Technology社の接触型心電図センサ(以後,ECGセンサと呼ぶ)Bioharness\cite{bioharness}を用いて可能な限りECG信号を取得している.

%----------------------------------------------
\section{}
%----------------------------------------------

%----------------------------------------------
\subsection{$t$-ILRMA}
%----------------------------------------------
近年,複素Student's~$t$分布に基づく非負地行列因子分解($t$-NMF)\cite{tnmf}が提案されており,BSS等のNMFに基づく特定のタスクにおいてItakura--Saitoダイバージェンスに基づくNMF\cite{isnmf}よりも性能が向上することが報告されている.このことに鑑み,複素tudent's~$t$分布を生成モデルとして仮定する独立低ランク行列分析(ILRMA)\cite{ILRMA,Kitamura2018_ilrma}である$t$-ILRMAが提案された\cite{tdist1,tdist2}.

%----------------------------------------------
\subsection{心拍推定アルゴリズム}
%----------------------------------------------
$t$-ILRMAを適用することで得られる分離信号に適用する心拍推定アルゴリズムについて説明する.まず,心拍信号の調波構造を強調しつつ低周波成分を除去するために二階微分フィルタを適用する.もう一度調波構造を強調するために半波整流を適用する.さらに,心拍信号が多く存在する帯域である0.7~Hz--1.4~Hzを通過帯域とする5次楕円IIRバンドパスディジタルフィルタを適用する.最後に,短時間フーリエ変換を適用してスペクトログラムに変換し,振幅スペクトログラムの最大ピークとなる周波数を時間フレーム毎に求めることで,推定心拍を得る.

%----------------------------------------------
\section{実験結果}
%----------------------------------------------
呼吸は他の成分よりも相対的に強い成分であるため,前処理としてカットオフ周波数1.5~Hzとするハイパスフィルタを観測信号に適用する.このハイパスフィルタは,位相歪みが生じない(線形位相特性を満たす)ようにFIRディジタルフィルタとして設計している.またフィルタのタップ長(次数)は170次である.

Fig.~\ref{fig:stilrmaa5}の3番目のスペクトログラムより,心拍の高調波成分が2.5~Hz,3.5~Hz,5~Hz,及び6~Hz付近に分離されている.1番目,2番目,及び4番目のスペクトログラムには心拍の高調波成分が見られず,振動台による振動及び呼吸の高調波成分が分離されていることが分かる.

Fig.~\ref{fig:stilrmaa5}に対して心拍推定アルゴリズムを適用した心拍推定値をFig.~\ref{fig:hrtilrmaa5}に示す.振動が加えられた60~s--100~sでは参照値と合致していないが,時間においては参照値と合致していることがわかる.

%-%-%-%-%-%-%-%-%
\begin{figure}[b]
  \centering
  \vspace{0pt} % 図上部の余白調整
  \includegraphics[width=0.8\columnwidth]{spect_tILRMA_ampdom_dofp5_64_est.pdf}
  \vspace{0pt} % 図とキャプション間の余白調整
  \caption{Spectrograms of separated signal obtained by $t$-ILRMA, where $p=1$ and $\nu=5$ (high-pass-filtered).}
  \vspace{0pt} % キャプション下部の余白調整
  \label{fig:stilrmaa5}
\end{figure}
%-%-%-%-%-%-%-%-%

%-%-%-%-%-%-%-%-%
\begin{figure}[t]
  \centering
  \vspace{0pt} % 図上部の余白調整
  \includegraphics[width=0.8\columnwidth]{hr_tILRMA_ampdom_dofp5_64_ch3.pdf}
  \vspace{0pt} % 図とキャプション間の余白調整
  \caption{Third Estimated (red) and reference (blue) heart rates obtained by $t$-ILRMA, where $p=1$ and $\nu=5$ (high-pass-filtered).}
  \vspace{0pt} % キャプション下部の余白調整
  \label{fig:hrtilrmaa5}
\end{figure}
%-%-%-%-%-%-%-%-%

%%%%% 参考文献 %%%%%
\begin{thebibliography}{99}% 10以上の文献数であれば99とする
%参考文献のフォントサイズを小さくしたい場合は下記2行のコメントアウトを解除
\itemsep 3pt % 項目の間隔微調整用
\fontsize{8pt}{10pt}\selectfont  % 項目のフォントサイズ微調整用

\bibitem{originica}
P. Comon, "Independent component analysis, A new concept?," {\em Signal Processing}, vol. 36, no. 3, pp.287--314, 1994.

\bibitem{ica2}
H. Sawada, N. Ono, H. Kameoka, D. Kitamura, and H. Saruwatari, ``A review of
blind source separation methods: Two converging routes to ILRMA originating from 
ICA and NMF,'' {\em Asia Pacific Signal and Information Processing Association Trans. Signal and Information Processing}, vol. 8, no. e12,
pp. 1–14, 2019.

\bibitem{bioharness}
\url{Zephyr~Technology~Bioharness, https://www.zephyranywhere.com/media/download/bioharness3-user-manual.pdf}. Accessed 4 January 2022.

\bibitem{tnmf}
K.~Yoshii, K.~Itoyama, and M.~Goto, ``Student's $t$ nonnegative matrix factorization and positive semidefinite tensor    
factorization for single-channel audio source separation,''
{\em Proc.~International Conference on Acoustics, Speech, and Signal Processing}, pp.51--55, 2016.

\bibitem{isnmf}
C.~Févotte, N.~Bertin, J.-L.~Durrieu, 
``Nonnegative matrix factorization with the Itakura-Saito divergence. With application to music analysis.'' 
{\em Neural Computation}, vol.~21, 793--830, 2009.

\bibitem{ILRMA}
D.~Kitamura, N.~Ono, H.~Sawada, H.~Kameoka, and H.~Saruwatari,
``Determined blind source separation unifying independent vector analysis and nonnegative matrix factorization,'' 
{\em IEEE/ACM Trans. Audio, Speech, and Language Processing}, vol.~24, no.~9, pp.~1626--1641, 2016.

\bibitem{Kitamura2018_ilrma}
D.~Kitamura, N.~Ono, H.~Sawada, H.~Kameoka, and H.~Saruwatari, ``Determined blind source separation with independent low-rank matrix analysis,'' 
{\em Audio Source Separation}, S.~Makino, Ed., pp. 125--155. Springer, Cham, 2018.

\bibitem{tdist1}
S.~Mogami, D.~Kitamura, Y.~mitsui, N.~Takamune, H.~Saruwatari, and N.~Ono, ``Independent low-rank matrix analysis based on complex Student's t-distribution for blind audio source     separation,''
{\em Proc.~IEEE International Workshop on Machine Learning for Signal Processing}, 2017. 
 
\bibitem{tdist2}
D.~Kitamura, S.~Mogami,~Y.~Mitsui, N.~Takamune, H.~Saruwatari, N.~Ono, Y.~Takahashi, and K.~Kondo, 
``Generalized independent low-rank matrix analysis using heavy-tailed distributions for blind source separation,''
{\em European Association for Speech, Signal and Image Processing Journal on Advances in Signal Processing}, vol. 2018, no. 28, p. 25, 2018. 

\end{thebibliography}

\end{document}