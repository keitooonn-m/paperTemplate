\chapter{緒言}
\label{chap:intro}

%----------------------------------------------
\section{本論文の背景}
%----------------------------------------------
自動車の運転中に運転者が睡眠,突発的な発作,体調の悪化による意識喪失等に見舞われることは多くの場合致命的な状況となる.このような運転者の状況が直接的な要因となって引き起こされる死亡事故は,全死亡事故の内の少なくない割合を占めている.そのため,人々の暮らしに自動車が欠かせない現代社会においては,運転中に運転者の状態を何らかの方法で管理することが重要課題の一つとなっている.

運転者の状態を自動車内の装備で常時モニタリングする方法は,これまで様々なものが提案されている\cite{sensorreview}.その多くは,運転者の様子を車内カメラで常時観測し,その動画データから運転者の睡眠,不注意,倦怠等の状態を検出することで注意を促すシステムである\cite{system1, system2, system3}.一方で,睡眠や倦怠等ではなく健康状態を常時モニタリングする方法としては,\textcolor{black}{心電図(electrocardiogram: ECG)センサ,筋電センサ,呼吸検知センサ,皮膚コンダクタンスセンサ等の接触センサを用いるシステムが提案されている}\cite{contactsensor1, contactsensor2}.これらのシステムでは,測定された生体信号を解析し運転者の心拍を推定することを目的としていることが一般的である.運転者の心拍は,運転中のストレスの検出や健康状態の管理等に役立てられる医学的に重要な情報である.従って,運転中の発作や意識喪失等の致命的な状況への対応方法の一つとして,運転中の心拍を推定し異常が発生した際にはアラートを発令・自動車を安全に停止するシステムが考えられる.

しかしながら,運転中の運転者の心拍を推定することは容易ではない.前述のように接触型生体センサを運転者に装着する方法が最も直接的であるが,これらの接触型生体センサは装着感や圧迫感のような運転者に無用のストレスを与える恐れがある.また,装着にかかる時間や労力等のコストも大きな問題であり,すべての運転者にそのような\textcolor{black}{計測}システムを普及させることはあまり現実的ではないといえる.より簡便なアプローチとして,自動車のステアリング等の操作部分にのみ心拍計測用の接触型センサを設置するシステム\cite{eazyap1, eazyap2, eazyap3}等も検討されるが,センサの設置箇所を運転者が常時把持しておかなければならない煩わしさが残る.さらに,常に運転者の皮膚と接触させる必要がある生体センサは\textcolor{black}{蒸れやかぶれ(接触性皮膚炎)を引き起こす等の衛生面}の問題が生じる恐れもある.本論文では,これらの問題を全て解決しながら運転中の心拍を推定するシステムとして,レーダ非接触型生体センサアレイ(以後,レーダセンサと呼ぶ)を活用したシステムを取り扱う.

%----------------------------------------------
\section{本論文の目的}
%----------------------------------------------
本論文では,運転者の着座するシート内部にレーダセンサを設置するシステムを対象とし,心拍推定を実現するための信号処理について取り扱う.本論文が対象とするシステムの外観図をFig.~\ref{fig:sensorstructure}に示す.図中の赤色の点にレーダセンサが埋め込まれている.自動車の走行中の振動を模擬するため,振動台の上にシートを設置した振動測定系を構築している.また,レーダセンサは放射されたレーダ信号が到達する(反射する)面の微小変位を測定できるものである.従って,Fig. \ref{fig:sensorimg}に示す図のように,シートに着座した運転者の背部(背中)と臀部(尻)の\textcolor{black}{微小}体表面変位が測定される.また,1つのレーダセンサは4チャネルの異なる指向性レーダで駆動しているため,同時に\textcolor{black}{近傍}4点の体表面変位を測定することが可能となっている.

しかし,この\textcolor{black}{振動}測定系においては,次に示す2点の問題がある.
\begin{itemize}
	\item \textcolor{black}{振動}測定系自体の振動
	\item センサと対象の相対的位置の変動
\end{itemize}
1つ目の問題点は,\textcolor{black}{振動}測定系自体が振動してしまうことによるノイズの存在である.レーダセンサはシートに埋め込んでいるため,走行時の車体の振動が観測信号としてそのまま計測されてしまう.2つ目の問題点は,シート内部のレーダセンサと運転者の相対的な位置関係が変動してしまうことである.この問題は運転者が体を動かした場合に顕著であるが,静止していたとしてもなお心拍以外の体表面変動(呼吸による変動等)が同時に計測されてしまう.これらの問題は観測信号の信号対ノイズ比(signal-to-noise ratio: SNR)を著しく低下させるため,高精度な心拍推定\textcolor{black}{を実現するため}には何らかの対処が必要となる.ここで,SNRにおける「信号」は心拍由来の体表面変位を表し,「ノイズ」は心拍以外に起因する体表面変位(\textcolor{black}{振動}測定系自体の指導と呼吸等の体表面変位の両方)を表す.以後,前者及び後者をそれぞれ心拍信号及びノイズ信号と呼ぶ.

本論文の目的は,観測信号が低SNRとなる前述の問題に対して信号処理を適用することで,心拍の推定精度の向上を目指すことである.本論文では,心拍信号とノイズ信号を異なる信号源と仮定して分離することを考える.観測信号から心拍信号を精度よく分離することができれば,その後の心拍推定アルゴリズムの推定精度向上が期待できる.特に,音響信号処理分野で高度に発展してきた歴史を持つ\textcolor{black}{ブラインド信号源分離(blind source separation: BSS)}\cite{originica, ica2}の適用を考える.BSSは観測信号を取得した際のセンサと信号源の空間的な位置関係(即ち,信号源の混合系)が未知\textcolor{black}{(ブラインド)}という条件の下で混合前の信号源を推定する技術である.本研究においては,レーダや体表面の位置関係を事前に仮定しないことから,心拍信号の分離にBSSを用いることは妥当\textcolor{black}{と考えられる}.

%-%-%-%-%-%-%-%-%
\begin{figure}[!t]
\centering
\includegraphics[width=0.6\hsize]{./ch_conventional/fig/sensorstructure.pdf}
\caption{Vibration measurement system and driver's seat with radar sensors.}
\label{fig:sensorstructure}
\end{figure}
%-%-%-%-%-%-%-%-%

%-%-%-%-%-%-%-%-%
\begin{figure}[htbp]
      \begin{minipage}[t]{0.45\hsize}
        \centering
        \includegraphics[keepaspectratio, width=4.5cm]{./ch_conventional/fig/sensorimgback.pdf}
        \subcaption{Back}
        \label{fig:sensorimgback}
      \end{minipage} 
      \begin{minipage}[t]{0.45\hsize}
        \centering
        \includegraphics[keepaspectratio, width=4.5cm]{./ch_conventional/fig/sensorimgbottom.pdf}
        \subcaption{Bottom}
        \label{fig:sensorimgbottom}
      \end{minipage} 
     \caption{Beams from radar sensor for measurering (a) back and (b) bottom surfaces \textcolor{black}{of driver's body}.}
     \label{fig:sensorimg}
  \end{figure}
%-%-%-%-%-%-%-%-%


%----------------------------------------------
\section{本論文の構成}
%----------------------------------------------
\ref{chap:measurementstructsig}章では\textcolor{black}{振動}測定系の\textcolor{black}{各種}条件と測定で得られる信号について述べ,\textcolor{black}{観測信号に短時間フーリエ変換(short-time Fourier transformation: STFT)を適用して得られるスペクトログラムを掲載する.また,\textcolor{black}{心拍信号及びノイズ信号の詳細}心拍や混在している呼吸や振動成分について説明する.}さらに,\textcolor{black}{本論文}で心拍信号\textcolor{black}{の参照値(正解値)と定義している}接触型のECGセンサの説明をする.
\ref{chap:methods}章では\textcolor{black}{まず,本研究においてBSSを用いる動機について説明する.その後に,}BSSの定式化を行い,Itakura--Saitoダイバージェンスに基づくNMF(Itakura-Saito-divergence-based NMF: ISNMF\textcolor{black}{\cite{isnmf}})について説明する.\textcolor{black}{また,本論文で取り扱うBSSの代表的な手法である独立行列分析(independence vector analysis: IVA\textcolor{black}{\cite{Kim2007_iva, auxIVA}}),独立低ランク行列分析(independence low-rank matrix analysis: ILRMA\textcolor{black}{\cite{ILRMA, Kitamura2018_ilrma}})},ILRMAに複素Student's $t$分布を生成モデルに持たせた$t$-ILRMA\textcolor{black}{\cite{tdist1, tdist2},}及び心拍推定アルゴリズムの説明を行う.
\textcolor{black}{\ref{chap:bsshrexp}章では\textcolor{black}{,予備実験として}\ref{chap:methods}章で説明したIVAを用いて観測信号から心拍信号を分離する.また分離信号に対して心拍推定アルゴリズムを心拍推定アルゴリズムを適用して,心拍信号が最も分離されるSTFTの窓長の関係性について調査\textcolor{black}{する}.}
\ref{chap:fbsshrexp}章では\textcolor{black}{,}\textcolor{black}{観測信号に\textcolor{black}{適用する前処理として},呼吸由来の信号成分を落とすハイパスフィルタを適用し,IVA,基底数固定型ILRMA,基底数可変型ILRMA,及び$t$-ILRMA\textcolor{black}{による}心拍信号の分離を行う.また分離信号に対して心拍推定アルゴリズムを適用し\textcolor{black}{,}心拍\textcolor{black}{推定精度}の比較を行う.
\ref{chap:con}章ではすべての章の総括した結言を述べる.}