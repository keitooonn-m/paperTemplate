\chapter{緒言}
\label{chap:intro}

%----------------------------------------------
\section{本論文の背景}
%----------------------------------------------
自動車の運転中の発作や急病などが原因と思われる死亡事故は,死亡事故のうち10\%を占めている.防御策として運転者の心拍を常時モニタリングすることが挙げられる.運転中の運転者の心拍値を解析し,異常時には警告したり,自動車を停止させることが必要となる.この信号を取得して解析する工程が非常に困難である.理由としては,運転中には車体自体の振動や運転者の呼吸や体動による振動など,たくさんの振動ノイズが混在してしまうためである.ハンドルに心拍測定装置を埋め込んで測定することも可能ではあるが,運転者が運転中にハンドルを絶えず握っている保証はない.運転者に要求の少ない心拍測定方法として,本研究で議論する,シートにセンサーを埋め込む心拍測定方法を提案する.この方法では,運転者が意図せずとも心拍を測定することが可能となる.この測定系を用いた心拍推定法が必要となる.

%----------------------------------------------
\section{本論文の目的}
%----------------------------------------------
問題解決として,Fig. \ref{fig:sensorstructure}のようにレーダ非接触生体センサアレイ(以後,レーダセンサと呼ぶ)を車のシートに埋め込んだ振動測定系で観測信号から心拍推定を行う.レーダセンサはFig. \ref{fig:sensorimg}のように送信ビーム2指向性,受信アンテナ2本の合計4指向性における体表面変位を取得している.しかし,この測定系において以下に示す2点の問題点がある.

\begin{itemize}
 \item 測定系自体の外来振動の印加
 \item 測定系と運転者の呼吸や体動による相対的位置関係の変動
\end{itemize}

前述のように,レーダセンサはシートに埋め込んでいるため,走行時の車体振動が観測信号として取得されてしまう.また,体表面変位の変動を取得しているため,運転者の心拍だけでなく,呼吸や体動などの変動も同時に取得してしまう.そのため,本研究ではこれらの振動ノイズから目的信号である心拍信号を取り出すために信号源分離を行う.


%-%-%-%-%-%-%-%-%
\begin{figure}[htbp]
 \begin{minipage}{0.5\hsize}
  \begin{center}
   \includegraphics[width=70mm]{./ch_conventional/fig/sensorstructure.pdf}
  \end{center}
  \caption{Image of shaking table.}
  \label{fig:sensorstructure}
 \end{minipage}
 \begin{minipage}{0.5\hsize}
  \begin{center}
   \includegraphics[width=70mm]{./ch_conventional/fig/sensorimg.pdf}
  \end{center}
  \caption{Image of sensor motion and position.}
  \label{fig:sensorimg}
 \end{minipage}
\end{figure}
%-%-%-%-%-%-%-%-%


%----------------------------------------------
\section{本論文の構成}
%----------------------------------------------
\ref{chap:measurementstructsig}章では測定条件と測定で得られる信号について述べる.
\ref{chap:methods}章では本論文で取り扱うBSSの手法及び心拍推定アルゴリズムの解説を行う.
\ref{chap:bsshrexp}章では\ref{chap:methods}で解説したIVAを用いて分離制度の高いチャネルとFFTサイズの調査を行う.
\ref{chap:fbsshrexp}章では測定信号に前処理としてフィルタを適用し,IVAや基底数固定型ILRMA,基底数可変型ILRMA,$t$-ILRMAを用いて分離制度の比較を行う.
\ref{chap:con}章ではすべての章の総括した結言を述べる.