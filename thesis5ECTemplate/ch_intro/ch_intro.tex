\chapter{緒言}
\label{chap:intro}

%----------------------------------------------
\section{本論文の背景}
%----------------------------------------------
音源分離とは,複数の音響信号が混ざった観測信号から混合前の信号を推定する技術である.
この技術は,Fig. \ref{fig:prac}に示す例のように,様々な場面で用いられている.
例えば,スマートフォンやスマートスピーカーの音声認識機能において入力音声の前段処理として用いられている.
これは端末に入力された信号から外部雑音や残響音を取り除いた目的信号のみを抽出することで認識性能を高めるためである.

音源分離手法は主に音源数及びチャネル数の関係によって手法が大別される.
まず,劣決定条件(音源数$>$チャネル数)の場合について述べる.
一般的にmp3形式やwav形式の音楽信号はチャネル数が1ch(モノラル)もしくは2ch(ステレオ)であり,そこに含まれる音源の数は2つより大きいことの方が多いため,通常の音楽音源分離は劣決定条件の下で行われる.
しかし,劣決定条件の音源分離を音源の性質や定位,残響長などの事前情報が無い状態で実現するのは困難である.
そこで,劣決定条件での音源分離は楽器信号特有の時間周波数構造を手掛かりとして,非負値行列因子分解(nonnegative matrix factorization: NMF)\cite{NMF}を用いた音源分離手法\cite{singlechsep, supNMF, MNMF_oz}が提案されている.
一方で,決定的条件(音源数$=$チャネル数)もしくは優決定条件(音源数$<$チャネル数)では,事前情報が無い状態でも音響信号の統計的性質を利用することで分離を推定するブラインド音源分離(blind source separation: BSS)\cite{bss_review}の手法が盛んに研究されている.
本研究で取り扱う独立ベクトル分析(independent vector analysis: IVA)\cite{Kim2007_iva, auxIVA}や独立低ランク行列分析(independent low-rank matrix analysis: ILRMA)\cite{ILRMA, Kitamura2018_ilrma}はBSSの代表的な手法である.

前述したとおり,音源分離では特定の信号から外部雑音や残響音を取り除いた目的信号のみを抽出することができる.本研究では,これを応用してシートに振動センサを取り付けた車に乗車する被験者から心拍を推定する.振動センサでは座椅子自体の振動と被験者の体動や呼吸といった雑音が混在した信号が得られる.
この信号に対してBSSの代表的な手法を適用し,高精度な心拍推定を行うことが必要とされている.


%-%-%-%-%-%-%-%-%
\begin{figure}[!t]
\centering
\includegraphics[width=0.8\hsize]{./ch_intro/fig/prac.pdf}
\caption{Typical application examples of audio source separation technique.}
\label{fig:prac}
\end{figure}
%-%-%-%-%-%-%-%-%


%----------------------------------------------
\section{本論文の目的}
%----------------------------------------------
本論文では,BSSの代表的手法であるIVAやILRMAを用いて測定信号から混在している振動や体動,呼吸といった雑音を取り除き高精度な心拍推定の実現を目指す.より良い心拍推定が可能である手法を調査するとともに,心拍推定制度がより高い手法の調査も行う.
%----------------------------------------------
\section{本論文の構成}
%----------------------------------------------
\ref{chap:measurementstructsig}章では測定条件と測定で得られる信号について述べる.
\ref{chap:methods}章では本論文で取り扱うBSSの手法及び心拍推定アルゴリズムの解説を行う.
\ref{chap:bsshrexp}章では\ref{chap:methods}で解説したIVAを用いて分離制度の高いチャネルとFFTサイズの調査を行う.
\ref{chap:fbsshrexp}章では測定信号に前処理としてフィルタを適用し,IVAや基底数固定型ILRMA,基底数可変型ILRMA,$t$-ILRMAを用いて分離制度の比較を行う.
\ref{chap:con}章ではすべての章の総括した結言を述べる.