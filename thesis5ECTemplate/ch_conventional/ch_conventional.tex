\chapter{振動測定系と測定信号}
\label{chap:measurementstructsig}

%----------------------------------------------
\section{まえがき}
%----------------------------------------------
本章では,本研究で扱う信号の測定条件と得られる信号について説明する.
まず\ref{sec:conv:measurementcondition}節では,本研究で使用する信号の測定条件について説明する.
\ref{sec:conv:signal}節では,\ref{sec:conv:measurementcondition}節で述べた測定条件から得られる信号について説明する

%----------------------------------------------
\section{測定条件}
\label{sec:conv:measurementcondition}
%----------------------------------------------
本研究で使用する信号の測定系はFig\ref{fig:sensorstructure}に示されているとおりである.振動台の上に赤色で示すレーダを取り付けたシートを設置し,被験者が座った状態で計全体を振動させる.レーダはFig\ref{fig:sensorimg}に示されているようなビームフォーミングセンサを使用している.これを,シートの背中と臀部に取り付け測定を行う.測定系による振動はsin波で単軸加振であり,振動状態はFig\ref{fig:sensorimg}に示されているとおりである.その他の測定条件はTable\ref{tab:vibevacondition}に示されているとおりである.


%-%-%-%-%-%-%-%-%
\begin{table}[b]
  \caption{VIbration evaluation conditions}
  \centering
  \scalebox{0.8}[0.9]{
  \begin{tabular}{|c|c|c|c|c|c|c|} \hline
    Data number & subject & Vibration direction & amplitude & Frequency &  Use of handles, etc. & part \\ \hline \hline
    0 & \multirow{7}{*}{G} & - & - & - & - & \multirow{7}{*}{Back and Bottom}   \\ \cline{1-1} \cline{3-3} \cline{4-4} \cline{5-5} \cline{6-6}
    1 & & \multirow{6}{*}{Up and Down} & 4 & 1.2 & - &  \\ \cline{1-1} \cline{4-4} \cline{5-5} \cline{6-6}
    2 & &  & 10 & 1.2 & - &   \\ \cline{1-1} \cline{4-4} \cline{5-5} \cline{6-6}
    3 & &  & 40 & 1.2 & - &   \\ \cline{1-1} \cline{4-4} \cline{5-5} \cline{6-6}
    4 & &  & 40 & 1.2 & Use &   \\ \cline{1-1} \cline{4-4} \cline{5-5} \cline{6-6}
    5 & &  & 4 & 2.4 & - &   \\ \cline{1-1} \cline{4-4} \cline{5-5} \cline{6-6}
    6 & &  & 10 & 2.4 & - &   \\ \hline
  \end{tabular}
  }
  \label{tab:vibevacondition}
\end{table}
%-%-%-%-%-%-%-%-%

%----------------------------------------------
\section{測定で得られる信号}
\label{sec:conv:signal}
%----------------------------------------------
\ref{sec:conv:measurementcondition}節の測定系によって得られる信号はFig\ref{fig:sensorimg}から分かるように4チャネルである.Fig\ref{fig:obssig}に示すような値が得られる.横軸は測定時間[s],縦軸は体表面の変動[$\mathrm{\mu}$m]となっている.
また,観測信号に短時間フーリエ変換(short-time Fourier transformation: STFT)を行うことで,スペクトログラムを描画する.
Fig. \ref{fig:obsspect}は,Fig. \ref{fig:obssig}にSTFTを行ったものである.60秒から360秒に横一直線に確認できる成分が振動成分,約3Hzに確認できる最も強い成分が呼吸成分,振動成分に隠れるように心拍成分が表れている.このように時間信号をスペクトログラムに描画することで,各成分が確認できるようになる.
Fig. \ref{fig:ecgsensor}は本研究で得る推定信号と比較するために使用する,接触型の心電図(electrocardiogram: ECG)センサ(以後,ECGセンサと呼ぶ)である.

%-%-%-%-%-%-%-%-%
\begin{figure}[b]
\centering
\includegraphics[width=1.0\hsize]{./ch_conventional/fig/obsSig.pdf}
\caption{Observation signal.}
\label{fig:obssig}
\end{figure}
%-%-%-%-%-%-%-%-%

%-%-%-%-%-%-%-%-%
%\begin{figure}[tb]
%\centering
%\includegraphics[width=1.0\hsize]{./ch_conventional/fig/obsSpect.pdf}
%\caption{Spectrogram obtained by observation signal.}
%\label{fig:obsspect}
%\end{figure}
%-%-%-%-%-%-%-%-%

%-%-%-%-%-%-%-%-%
\begin{figure}[tb]
\centering
\includegraphics[width=0.7\hsize]{./ch_conventional/fig/ecgsensor.pdf}
\caption{Contact electrocardiogram sesor.}
\label{fig:ecgsensor}
\end{figure}
%-%-%-%-%-%-%-%-%

%----------------------------------------------
\section{本章のまとめ}
%----------------------------------------------
本章では,測定条件と測定によって得られる信号について説明した.また,測定したデータとの比較を行うために使用した接触型ECGセンサについて説明した.
次章では,測定によって得られる信号に適用する各手法と心拍推定アルゴリズムについて説明する.


\chapter{適用手法}
\label{chap:methods}

%----------------------------------------------
\section{まえがき}
%----------------------------------------------
本章では,本研究で適用した手法について説明する.
まず\ref{sec:conv:researchmotivation}節では,本研究の動機について説明する.
\ref{sec:conv:bssformularization}節では,周波数領域におけるBSSの定式化を行う.
\ref{sec:conv:isnmf}節では,行列分解の手法の一つであるNMFのうちItakura-Saitoダイバージェンスに基づくNMFについて説明する
\ref{sec:conv:ilrma}節では,本研究で扱ったILRMA,\ref{sec:conv:tilrma}節では,本研究で扱った$t$-ILRMAについて,\ref{sec:conv:heartrateestalgo}節では,心拍推定アルゴリズムについて説明する.

%----------------------------------------------
\section{本研究の動機}
\label{sec:conv:researchmotivation}
%----------------------------------------------
本章で述べるBSSの代表的な手法は音響信号処理の分野で広く使われている手法である.本研究で取り扱う信号は生体信号であるが,体動や呼吸などの雑音が混在した信号の中から心拍信号のみを取り出すためにそれらの手法が適用し,分離可能かどうか調査するものである.

%----------------------------------------------
\section{BSSの定式化}
\label{sec:conv:bssformularization}
%----------------------------------------------

周波数領域のBSSの定式化を行う.
複数の音源が混合している観測信号中の音源数及びチャネル数をそれぞれ$N$及び$M$と定義する.
Fig. \ref{fig:bss}に$N=M=2$の場合のBSSの概略を示す.
混合前の音源,観測信号及び分離信号に対して短時間フーリエ変換(short time fourier transformation: STFT)したものをそれぞれ
\begin{align}
\bm{s}_{ij} &= (s_{ij,1}, s_{ij,2}, \cdots, s_{ij,n}, \cdots, s_{ij,N})^{\mathrm{T}} \in \mathbb{C}^{N} \label{eq:s} \\
\bm{x}_{ij} &= (x_{ij,1}, x_{ij,2}, \cdots, x_{ij,m}, \cdots, x_{ij,M})^{\mathrm{T}} \in \mathbb{C}^{M} \label{eq:x} \\
\bm{y}_{ij} &= (y_{ij,1}, y_{ij,2}, \cdots, y_{ij,n}, \cdots, y_{ij,N})^{\mathrm{T}} \in \mathbb{C}^{N} \label{eq:y}
\end{align}
と定義する.
ここで,$n=1, 2,  \cdots, N$及び$m=1, 2, \cdots, M$はそれぞれ音源及びチャネルのインデクスを表す.
式(\ref{eq:s})から式(\ref{eq:y})と周波数ごとの時不変な(時間フレーム$j$に依存しない)混合行列$\bm{A}_i \in \mathbb{C}^{M\times N}$を用いると,混合時の残響長がSTFTの短時間信号長(窓長)$L$よりも十分に短い場合,観測信号は次式で表せる.
\begin{align}
  \bm{x}_{ij} = \bm{A}_i \bm{s}_{ij} \label{eq:xas}
\end{align}
ここで,$M=N$かつ$\bm{A}_i$がフルランクの場合は,分離行列$\bm{W}_{i} = (\bm{w}_{i1}~\bm{w}_{i2}~\cdots~ ~\bm{w}_{iN})^{\mathrm{H}} \in \mathbb{C}^{N \times M} $が存在し,分離信号は次式で表せる.
\begin{align}
  \bm{y}_{ij} = \bm{W}_i \bm{x}_{ij} \label{eq:ywj}
\end{align}
ここで,$^{\mathrm{H}}$は行列またはベクトルのエルミート転置を表す.
BSSは混合行列$\bm{A}_i$が未知の状態で分離行列$\bm{W}_{i}$を全ての周波数$i=1, 2, \cdots, I$において推定する問題である.

%-%-%-%-%-%-%-%-%
\begin{figure}[!t]
\centering
\includegraphics[width=0.75\hsize]{./ch_conventional/fig/bss.pdf}
\caption{Outline of BSS, where $N=M=2$.}
\label{fig:bss}
\end{figure}
%-%-%-%-%-%-%-%-%



%----------------------------------------------
\section{IVA}
\label{sec:conv:iva}
%----------------------------------------------
%-%-%-%-%-%-%-%-%
\begin{figure}[t]
\centering
\includegraphics[width=0.95\hsize]{./ch_conventional/fig/iva.pdf}
\caption{Mixing and demixing model in IVA, where $N=M=2$.}
\label{fig:ivamodel}
\end{figure}
%-%-%-%-%-%-%-%-%

%-%-%-%-%-%-%-%-%
\begin{figure}[t]
\centering
\includegraphics[width=0.95\hsize]{./ch_conventional/fig/laplace_sav.pdf}
\caption{Zero-mean and spherically symmetric Laplace distribution, where $I=2$ and $s^*_{ij,n}$ can be considered as either real or imaginary part of $s_{ij,n}$.}
\label{fig:laplace}
\end{figure}
%-%-%-%-%-%-%-%-%
Iここでは音源,観測信号及び分離信号それぞれについて,全ての周波数ビンに関する成分をまとめたベクトルを
\begin{align}
    \overline{\bm{s}}_{j,n} &= (s_{1j,n}, s_{2j,n}, \cdots, s_{ij,n}, \cdots, s_{Ij,n} )^{\mathrm{T}} \in \mathbb{C}^{I} \\
    \overline{\bm{x}}_{j,m} &= (x_{1j,m}, x_{2j,m}, \cdots, x_{ij,m}, \cdots, x_{Ij,m} )^{\mathrm{T}} \in \mathbb{C}^{I} \\
    \overline{\bm{y}}_{j,n} &= (y_{1j,n}, y_{2j,n}, \cdots, y_{ij,n}, \cdots, y_{Ij,n} )^{\mathrm{T}} \in \mathbb{C}^{I}
\end{align}
と定義する.
Fig. \ref{fig:ivamodel}に$M=N=2$の場合の混合分離モデルを示す.
IVAは周波数ごとに独立した分離行列$\bm{W}_i$を推定する.
ただし,推定の過程で全周波数を含む$I$次元分布を生成モデルとして仮定し,$I$次元ベクトル内の高次相関を仮定している.
このIVAの生成モデルには,Fig. \ref{fig:laplace} で示すような球状対称ラプラス分布が用いられ,式(\ref{eq:iva_model})で表される.
\begin{align}
  p(\overline{\bm{y}}_{jn}) = \frac{1}{\pi \prod_{i} \sigma_{i,n}} \exp \left(  - \sqrt{ \sum_i \left| \frac{y_{ij,n}}{\sigma_{i,n}} \right|^2}  \right)
\label{eq:iva_model}    
\end{align}
ここで,$\sigma_{i,n}$はスケールパラメタである.
式(\ref{eq:iva_model})の多次元ラプラス分布は球対称性を持つため,同一ベクトル内の成分が高次相関を持つ.
したがって,IVAは同時に生起する周波数成分を一つの音源としてまとめるという傾向がある.
つまり,信号の基本周波数とその倍音成分が同一音源として扱われやすいと言える.
音源周波数ベクトル間の独立性$p(\overline{\bm{y}}_{j,1}, \overline{\bm{y}}_{j,2}, \cdots, \overline{\bm{y}}_{j,N}) = \prod_n p(\overline{\bm{y}}_{i,n})$を仮定すると,IVAの観測信号に対する負対数尤度関数は次式で得られる.
\begin{align}
    \mathcal{L} = -2J \sum_i \log |\det \bm{W}_i| + \sum_{j,n} G(\overline{\bm{y}}_{j,n})
    \label{eq:ivalike}
\end{align}
ここで,$G(\overline{\bm{y}}_{j,n})$はコントラスト関数と呼ばれ,$G(\overline{\bm{y}}_{j,n}) = -\log p(\overline{\bm{y}}_{j,n})$で定義される.
IVAの最適化は補助関数法\cite{auxfunc}を用いた手法によって高速かつ安定に行える\cite{auxIVA, stable_auxIVA}.


%----------------------------------------------
\section{Itakura-Saito ダイバージェンスに基づくNMF}
\label{sec:conv:isnmf}
%----------------------------------------------

NMF~\cite{NMF}とは行列分解の方法の一つである.
NMFが他の行列分解の方法であるLU分解や固有値分解と異なるのは非負値行列を対象としている点である.
また,NMFは対象とする非負値行列の低ランク性を仮定して分解することで,行列の潜在パタンを抽出できるアルゴリズムである.
そして,NMFは劣決定音源分離に適用することが可能である\cite{singlechsep, supNMF, MNMF_oz}.
単一チャネルの音響信号をSTFTすることで得られるパワースペクトログラムのNMFによる分解は次式で表される.

\begin{align}
    |\bm{Z}|^{.2} = \bm{TV}
\end{align}
ここで,$| \cdot |$は要素ごとの絶対値を,ドット付きの指数は要素ごとの累乗を示す.よって,$|\bm{Z}|^{.2}$はパワースペクトログラムを表す.
また,$\bm{T} \in \mathbb{R}^{I \times K}_{\geq 0}$を基底行列,$\bm{V} \in \mathbb{R}^{K \times J}_{\geq 0}$をアクティベーション行列という.
$K$はNMFの分解において手動で与えるパラメタであり,基底行列$\bm{T}$の列ベクトルの本数(基底ベクトル数)である.従って,行列積$\bm{TV}$のランクは$K$と一致し,通常は$K \ll \min (I,J)$となるように設定される.
つまり,Fig. \ref{fig:nmf_ps}に示すように単一チャネルの音響信号を対象としたNMFでは,パワースペクトログラムに対して低ランク近似を行うことで,$\bm{T}$が音響信号中の頻出スペクトルパタンとなり,$\bm{V}$が各スペクトルパタンの発音タイミング(アクティベーション)となるような分解が可能である.
また,基底行列$\bm{T}$とアクティベーション行列$\bm{V}$は次式の最小化問題の解として推定される.

\begin{align}
    \nonumber \min_{\bm{T,V}} \mathcal{D}(|\bm{Z}|^{.2} | \bm{TV}) ~~ & \mathrm{s.t.} ~ t_{ik}, v_{kj} \geq 0 \\ &\forall i = 1, 2, ..., I, ~j = 1, 2, ..., J, ~k = 1, 2, ..., K 
\end{align}
ここで,$t_{ik}$及び$v_{kj}$は$\bm{T}$及び$\bm{V}$の要素である.
また,$ \mathcal{D}(|\bm{Z}|^{.2} | \bm{TV})$は2つの行列($|\bm{Z}|^{.2}$及び$\bm{TV}$)間の類似度を測る関数である.
行列の類似度を図る関数には,次式で表される$\beta$-divergence~\cite{psi-div, ext_nmf}がよく利用される\cite{pssnmf}.
\begin{align}
  d_{\beta}(a|b) = \left\{ \begin{array}{lll}
    \displaystyle \frac{a}{b} - \log \frac{a}{b} -1 & (\beta = 0) \\
    a \log \displaystyle\frac{a}{b} + b - a & (\beta =1) \\
    \displaystyle \frac{a^{\beta}}{\beta (\beta -1)} + \frac{b^{\beta}}{\beta} - \frac{ab^{\beta - 1}} {\beta - 1} & (\mathrm{otherwise}) 
  \end{array} \right. \label{eq:betad}
\end{align}
$\beta$-divergenceの中でも特に$\beta= 2,1,0$の場合はそれぞれ二乗Euclid距離,一般化Kullback--Leiblerダイバージェンス及びItakura--Saitoダイバージェンスと呼ばれている.
これらのうち,ダイバージェンスと名のつくものは以下に示す距離の公理
\begin{enumerate}
  \item 非負性:$\mathcal{D}(a|b) \geq 0 ~~\forall a,b \in \mathbb{R}$
  \item 同一性:$\mathcal{D}(a|b) = 0 \Leftrightarrow a = b~~\forall a,b \in \mathbb{R}$
  \item 対称性:$\mathcal{D}(a|b)= \mathcal{D}(a|b) ~~\forall a,b \in \mathbb{R}$
  \item 三角不等式:$\mathcal{D}(a|b) + \mathcal{D}(b|c) \geq \mathcal{D}(a|c) ~~\forall a,b,c \in \mathbb{R}$
\end{enumerate}
のうち,対称性と三角不等式を満たさない.
%-%-%-%-%-%-%-%-%
\begin{figure}[!t]
\centering
\includegraphics[width=0.95\hsize]{./ch_conventional/fig/NMF_ps.pdf}
\caption{NMF decomposition of power spectrogram, where $| \cdot |$  and dotted exponent for matrices denote entrywise absolute value and entrywise exponent, respectively.}
\label{fig:nmf_ps}
\end{figure}
%-%-%-%-%-%-%-%-%

%-%-%-%-%-%-%-%-%
\begin{figure}[!b]
\centering
\includegraphics[width=0.95\hsize]{./ch_conventional/fig/gauss_sav.pdf}
\caption{Circularly symmetric complex Gauss distribution.}
\label{fig:gauss}
\end{figure}
%-%-%-%-%-%-%-%-%

本論文ではItakura--Saitoダイバージェンスに基づくNMF(Itakura--Saito-divergence-based NMF: ISNMF)\cite{isnmf}について述べる.
$\bm{Z}$の要素である複素スペクトル$z_{ij}$が以下の確率モデルに従って生成されていると仮定する.
\begin{align}
    z_{ij} = \sum_k c_{ij,k} \\
    c_{ij,k} \sim \mathcal{N}_{\mathbb{C}} (0, t_{ik}v_{lj})
\end{align}
ここで,$c_{ij,k} \in \mathbb{C}$は全ての$i$,$j$及び$k$に関して互いに独立と仮定する.
また,$c$を複素数の確率変数としたとき,$\mathcal{N}_{\mathbb{C}} (\mu, \sigma^2)$は一次元複素ガウス分布を表し,その確率密度関数は次式で与えられる.
\begin{align}
    p_c(c | \mu, \sigma^2) &= \frac{1}{\pi \sigma^2} \exp \left\{ -\frac{|c-\mu|^2}{\sigma^2} \right\}
\end{align}
ここで,$\mu$及び$\sigma^2$はそれぞれ平均及び分散を示す.
また,$c_1$と$c_2$が独立である場合,ゼロ平均の一次元複素ガウス分布において以下の加法性が成り立つ.
\begin{align}
    c_1 \sim \mathcal{N}_{\mathbb{C}}(0,\sigma_1^2)かつc_2 \sim \mathcal{N}_{\mathbb{C}}(0,\sigma_2^2) ~~ \Longrightarrow ~~ c_1 + c_2 \sim \mathcal{N}_{\mathbb{C}}(0,\sigma_1^2 + \sigma_2^2)
\end{align}
よって,
\begin{align}
    \sum_k c_{ij,k} \sim \mathcal{N}_{\mathbb{C}}\left( 0, \sum_k t_{ik} v_{kj} \right)
\end{align}
を用いて次式が成り立つ.
\begin{align}
    z_{ij} \sim \mathcal{N}_{\mathbb{C}}\left( 0, \sum_k t_{ik} v_{kj} \right) \label{eq:nmfgen}
\end{align}
これがISNMFの生成モデルで,Fig. \ref{fig:gauss}のような球対称複素ガウス分布である.
ここで,観測信号$z_{ij}$が与えられた場合における$t_{ik}$及び$v_{kj}$の最尤推定問題を考える.
このとき,尤度関数は
\begin{align}
    \mathcal{L}(\bm{T}, \bm{V}) = \prod_{i,j} \frac{1}{\pi \sum_k t_{ik} v_{kj} } \exp \left( -\frac{|z_{ij}|^2}{\sum_k t_{ik} v_{kj}} \right)
\end{align}
となり,負対数尤度は
\begin{align}
    -\log\mathcal{L}(\bm{T}, \bm{V}) = \sum_{i,j} \left( \frac{|z_{ij}|^2}{\sum_k t_{ik} v_{kj}} + \log \sum_k t_{ik} v_{kj} + \log \pi \right)
\end{align}
で表される.
これは観測信号のパワースペクトログラム$|z_{ij}|^2$に対するISNMFの目的関数(式(\ref{eq:betad})の$\beta = 0$の場合)と定数部分を除いて一致するので,尤度関数は以下のように書き換えられる.
\begin{align}
    -\log\mathcal{L}(\bm{T}, \bm{V}) = d_{\mathrm{IS}} \left( |z_{ij}|^2 | \sum_k t_{ik} v_{kj} \right) + \mathrm{const.}
    \label{eq:isnmf}
\end{align}
ここで,$d_{\mathrm{IS}}( \cdot | \cdot )$は2つの行列間のItakura-Saitoダイバージェンスを示す.
つまり,ISNMFを観測信号のパワースペクトログラム$|\bm{Z}|^{.2}$に適用したとき,複素スペクトル$z_{ij}$が式(\ref{eq:nmfgen})で表される生成モデルに従い,全時間周波数グリッドに関して互いに独立であると仮定されている.
また,ISNMFの$\bm{T}$及び$\bm{V}$の最適化のための反復更新式は式(\ref{eq:MUT})及び式(\ref{eq:MUV})で表される\cite{MU}.

\begin{align}
    t_{ik} \leftarrow t_{ik} \sqrt \frac{ \sum_j |z_{ij}|^2 v_{kj} \left( \sum_{k'} t_{ik'} v_{k'j} \right)^{-2} }{ \sum_j v_{kj} \left( \sum_{k'} t_{ik'} v_{k'j} \right)^{-1} } \label{eq:MUT} \\
    v_{kj} \leftarrow v_{kj} \sqrt \frac{ \sum_i |z_{ij}|^2 t_{ik} \left( \sum_{k'} t_{ik'} v_{k'j} \right)^{-2} }{ \sum_i t_{ik} \left( \sum_{k'} t_{ik'} v_{k'j} \right)^{-1} } \label{eq:MUV}
\end{align}
この更新式は乗算型反復更新式と呼ばれ,目的関数が単調非増加であることが保証されている.

%----------------------------------------------
\section{ILRMA}
\label{sec:conv:ilrma}
%----------------------------------------------

ILRMAの反復最適化の概要をFig. \ref{fig:ilrma_outline}に示す.

%-%-%-%-%-%-%-%-%
\begin{figure}[!t]
\centering
\includegraphics[width=0.95\hsize]{./ch_conventional/fig/ilrmaoutline.pdf}
\caption{Outline of ILRMA.}
\label{fig:ilrma_outline}
\end{figure}
%-%-%-%-%-%-%-%-%

ここで$\bm{T}_n$及び$\bm{V}_n$は$n$番目の音源のパワースペクトログラムをモデル化する基底行列及びアクティベーション行列を示す.
ILRMAはIVAに対応する分離行列$\bm{W}_i$の反復最適化とISNMFの低ランクモデリングに対応する$\bm{T}_n\bm{V}_n$の反復最適化が交互に行われる.
具体的には,分離行列$\bm{W}_i$により推定された分離信号をNMFによって非負低ランク行列でモデル化し,得られた$\bm{T}_n$及び$\bm{V}_n$の各時間周波数成分を式(\ref{eq:nmfgen})における分散(各音源の生成モデルの推定パラメタ)として用いて分離行列を再度推定する,というプロセスが反復的に行われる.
ILRMAの生成モデルはISNMFと同様に次式の複素ガウス分布で表される.
\begin{align}
    y_{ij,n} &= \sum_k c_{ij,k,n} \\
    c_{ij,k,n} &= \mathcal{N}_{\mathbb{C}}(0, t_{ik,n} v_{kj,n}) \label{eq:ilrma_gen}
\end{align}
ここで,$t_{ik,n}$及び$v_{kj,n}$は$n$番目の音源に関する基底行列$\bm{T}_n$及びアクティベーション行列$\bm{V}_n$の非負要素であり,$k = 1, 2, \cdots, K$は基底インデクスである.
また,$c_{ij,k,n} \in \mathbb{C}$は互いに独立であると仮定する.
このとき,観測$x_{ij,n}$が与えられた場合において$\bm{W}_i$,$\bm{T}_n$及び$\bm{V}_n$を最尤推定する問題を考える.
ISNMFのときと同様に
\begin{align}
    \sum_k c_{ij,k,n} \sim \mathcal{N}_{\mathbb{C}}\left( 0, \sum_k t_{ik,n} v_{kj,n} \right)
\end{align}
より,
\begin{align}
    y_{ij,n} \sim \mathcal{N}_{\mathbb{C}}\left( 0, \sum_k t_{ik,n} v_{kj,n} \right) 
\end{align}
が成り立つので,この生成モデルに基づく観測信号の負対数尤度は次式で表される.
\begin{align}
    \mathcal{L}(\mathsf{W, T, V}) = -2J \sum_i \log | \det \bm{W}_i | + \sum_{i,j,n} \left( \frac{|\bm{w}_{i,n}^{\mathrm{H}}\bm{x}_{ij}|^2}{\sum_k t_{ik,n}v_{kj,n}} + \log \sum_k t_{ik,n}v_{kj,n} \right)
    \label{eq:ilrmalike2}
\end{align}
ここで,$\mathsf{W}=\{ \bm{W}_i \}_{i=1}^I$,$\mathsf{T}=\{ \bm{T}_n \}_{n=1}^N$及び$\mathsf{V}=\{ \bm{V}_n \}_{n=1}^N$は最適化パラメタの集合である.
式(\ref{eq:ilrmalike2})を見ると,第二項と第三項は式(\ref{eq:isnmf})のISNMFの尤度関数に対応していることがわかる.

分離行列$\bm{W}_i$の関する最適化は,分離ベクトル$\bm{w}_{i,n}$の更新を反復射影法(iterative projection: IP)\cite{auxIVA}を用いることで次式で行われる.
\begin{align}
\bm{U}_{i,n} &= \frac{1}{J} \sum_j \frac{1}{\sum_{l}t_{ik,n}v_{kj,n}} \bm{x}_{ij} \bm{x}_{ij}^{\mathrm{H}} \label{eq:ip1} \\
\bm{w}_{i,n} &\leftarrow (\bm{W}_i \bm{U}_{i,n})^{-1} \bm{e}_n \label{eq:ip2} \\
\bm{w}_{i,n} &\leftarrow \bm{w}_{i,n} ( \bm{w}_{i,n}^{\mathrm{H}} \bm{U}_{i,n} \bm{w}_{i,n} )^{-\frac{1}{2}} \label{eq:ip3}
\end{align}
ここで,$\bm{e}_n \in \mathbb{R}_{\{0, 1\}}^{N}$は$n$番目の要素が1,他要素が0のベクトルである.

NMFによる低ランクモデリングのパラメタ$\bm{T}_n\bm{V}_n$の最適化は式(\ref{eq:MUT})及び式(\ref{eq:MUV})の乗算型反復更新式を用いて次式で表される.
\begin{align}
t_{ik, n} &\leftarrow  t_{ik, n} \sqrt{ \frac{ \sum_{j} | \bm{w}_{i,n}^{\mathrm{H}}\bm{x}_{ij} |^2 v_{kj, n} ( \sum_{k'} t_{ik', n} v_{k'j, n} )^{-2} }{ \sum_{j} v_{kj, n} ( \sum_{k'} t_{ik', n} v_{k'j, n} )^{-1} } } \label{eq:t} \\
v_{kj, n} &\leftarrow  v_{kj, n} \sqrt{ \frac{ \sum_{i} | \bm{w}_{i,n}^{\mathrm{H}}\bm{x}_{ij} |^2 t_{ik, n} ( \sum_{k'} t_{ik', n} v_{k'j, n} )^{-2} }{ \sum_{i} t_{ik, n} ( \sum_{k'} t_{ik', n} v_{k'j, n} )^{-1} } } \label{eq:v}
\end{align}
これらの更新式も式(\ref{eq:MUT})及び式(\ref{eq:MUV})と同様に目的関数式(\ref{eq:ilrmalike2})の値が単調非増加であることが保証されている.

%----------------------------------------------
\section{$t$-ILRMA}
\label{sec:conv:tilrma}
%----------------------------------------------

Yoshii et al. が提案した$t$-NMF\cite{tnmf}がISNMFやCauchy NMF\cite{cauchynmf}と比較して性能が向上したことに鑑み,複素Student's$t$分布を生成モデルに持つようなILRMAである$t$-ILRMAが提案された.複素Student's$t$分布二は形状パラメータ$\nu(\textgreater0)$があり,これを$\nu=1$とすると複素Cauchy分布のような安定性は何ため,スペクトログラムの加法の正当性が保証されない.しかし,$\nu=2$付近のときの$t$-NMFは,ISNMFやCaunchy NMFに比べて優れたロバストな結果を示すことが報告されている.$t$-ILRMAの生成モデルは次式で表される.
\begin{align}
     \prod_{i,j}p\left(y_{ij}\right)= \prod_{i,j} \frac{1}{\pi \sigma_{ij}^{2} } \left( 1+\frac{2}{\nu}\frac{|z_{ij}|^2}{\sigma_{ij}^{2}} \right)^{\frac{2+\nu}{2}} \label{eq:tmodel}
\end{align}
\begin{align}
     \sigma_{ij}^{p}=\sum_{l}t_{il}v_{lk}{eq:tsigma}
\end{align}

ここで,$\sigma_{ij}$は時間周波数ごとに時変な非負値であり,振幅スペクトラム$|z_{ij}|$に関係する量である.$p$はNMFモデルのドメインを決定する量であり,$1\leqq p \leqq$を満たしているとする,$p=1$のときは振幅ドメインのNMFモデルを,$p=2$のときはパワードメインのモデルを考えることに相当する.$p=2$,$\nu\rightarrow\infty$のときはISNMFのモデルに一致し,$p=1$,$\nu=1$のときはCauchyNMFのモデルに一致する,この負の対数尤度をとることにより,$t$-ILRMAのコスト関数
\begin{align}
    \mathcal{L}_{t} = \mathrm{const.}-2J\log|\mathrm{det}\bm{W}_{i}|+\sum_{i,j}\left\lbrack\left(1+\frac{\nu}{2}\right)\log\left(1+\frac{2}{\nu}\frac{|y_{ij}^{2}}{\sigma_{ij}^{2}}\right)+2\log\sigma_{ij}\right\rbrack \label{eq:tcost}
\end{align}
を得る.

分離行列$\bm{W}_{i}$は節\ref{sec:conv:ilrma}と同様に分離ベクトル$w_{i,n}$の更新をIPを用いることで次式で行われる.
\begin{align}
    U_{i,n}=\frac{1}{J}\left(\frac{2}{\nu}+1\right)\sum_{j}\frac{1}{\alpha_{ij,n}\sigma_{ij,n}^{2}}x_{ij}x_{ij}^{\mathrm{H}} \label{eq:tip1}
\end{align}
\begin{align}
    w_{i,n}\leftarrow\left(W_{i}U_{i,n}\right)^{-1}e_{n} \label{eq:tip2}
\end{align}
\begin{align}
    w_{i,n}\leftarrow\frac{w_{i,n}}{\sqrt{w_{i,n}^{\mathrm{H}}U_{i,n}w_{i,n}}} \label{eq:tip3}
\end{align}

NMFパラメタ$T_{n}V_{n}$の更新式は次式で表される.
\begin{align}
    t_{il,n}\leftarrow t_{il,n}\left(\frac{\sum_{j}\frac{|y_ij,n|^{2}}{\frac{\nu}{\nu+2}\sigma_{ij,n}^{2}+\frac{2}{\nu+2}|y_{ij,n}|^{2}}\sigma_{ij,n}^{-p}v_{lj,n}^{-\frac{2}{p}}}{\sum_{i}\sigma_{ij,n}^{-p}v_{lj,n}}\right)^{\frac{p}{p+2}} \label{eq:tt}
\end{align}
\begin{align}
    v_{lj,n}\leftarrow v_{lj,n}\left(\frac{\sum_{i}\frac{|y_ij,n|^{2}}{\frac{\nu}{\nu+2}\sigma_{ij,n}^{2}+\frac{2}{\nu+2}|y_{ij,n}|^{2}}\sigma_{ij,n}^{-p}t_{il,n}^{-\frac{2}{p}}}{\sum_{i}\sigma_{ij,n}^{-p}t_{il,n}}\right)^{\frac{p}{p+2}} \label{eq:tv}
\end{align}
これらの更新式も節\ref{sec:conv:ilrma}と同様にコスト関数式(\ref{eq:tcost})の値が単調非増加であり,収束することが保証されている.


%----------------------------------------------
\section{心拍推定アルゴリズム}
\label{sec:conv:heartrateestalgo}
%----------------------------------------------

測定した信号を$\left(x[t]\right)_{t=1}^{T}$とする.ここで,$t=1, 2, \cdots , T$は離散時間インデクスであり,$T$は観測時間長である.今回使用した心拍推定アルゴリズムは以降に示す式の通りである.この信号には0.5Hz付近に呼吸が強く入っているため,次式のような二回微分フィルタをかけることで呼吸を落とすことが可能である.二回微分フィルタは次式で表される.
\begin{align}
  \left\{ \begin{array}{ll}
    x'[t]=x[t+1]-x[t]\quad\quad \forall{t} \\
    x''[t]=x'[t+1]-x'[t]\hspace{12.5pt} \forall{t}
  \end{array} \right. \label{eq:diffilter}
\end{align}
半波整流した後に,心拍の高周波成分である0.7Hz~1.4Hzを通すバンドパスフィルタをかける.半波整流とバンドパスフィルタ二よる処理は次式で表される.
\begin{align}
  x_{\mathrm{r}}[t] = \left\{ \begin{array}{ll}
    x''[t]\quad \mathrm{if}\quad x''[t]\geqq0\quad \forall{t} \\
    0\hspace{25pt} \mathrm{otherwise}\hspace{24pt} \forall{t}
  \end{array} \right. \label{eq:hrec}
\end{align}
\begin{align}
    \left(x_{\mathrm{BPF}}[t]\right)_{t=1}^{T}=\mathrm{BPF}\left[\left(x_{r}\right)_{t=1}^{T}\right]
\end{align} \label{eq:bpf}

次に,$\left(x_{\mathrm{BPF}}[t]\right)_{t=1}^{T}$を短時間区間信号に分割する.STFTの分析窓関数の長さ及びシフト長をそれぞれ$Q$及び$\tau$としたとき,$x_{\mathrm{BPF}}[t]$の$j$番目の短時間区間信号(時間フレーム)は次式で表される.
\begin{align}
  \nonumber \bm{x}^{[j]} &= \left[ x[(j-1)\tau +1], x[(j-1)\tau +2], \cdots, x[(j-1)\tau +Q] \right]^{\mathrm{T}} \\
  &=  [ x^{[j]} [1], x^{[j]} [2], \cdots, x^{[j]} [q], \cdots, x^{[j]} [Q] ]^{\mathrm{T}} \in \mathbb{R}^Q
\end{align}
ここで,$l =1, 2, \cdots, L$,$j=1, 2, \cdots, J$及び$q=1, 2, \cdots, Q$はそれぞれ離散時間のインデクス,時間フレーム及び時間フレーム内のサンプルを示す.
また,時間フレーム数$J$は次式によって与えられる.
\begin{align}
    J= \frac{L}{\tau}
\end{align}
ただし,信号長$L$はセグメント数$J$が整数となるように各時間フレームの信号の両端にゼロを挿入する処理(ゼロパディング)が施される.
そして,信号$\bm{x} = [ x[1], x[2], \cdots, x[L] ]^{\mathrm{T}}  \in \mathbb{R}^L$のSTFTは次式のように表される.
\begin{align}
    \bm{X} = \mathrm{STFT}_{\bm{\omega}}(\bm{x}) \in \mathbb{C}^{I \times J}  \label{eq:tstft}
\end{align}
式(\ref{eq:tstft})によって得られた各要素に対して絶対値を取り,ピーク値のインデクス(周波数)を取得する.この処理は以下のように表される.
\begin{align}
    \hat{x}[j]=\argmax|x[j]|  \label{eq:searchmax}
\end{align}
得られたピーク値のインデクスを降順に並べ替える.これらの値のうち,推定範囲$0.7\leqq f \leqq 1.6$に収まるものだけを扱い,推定心拍は$\hat{h_{o}}[j]=60\hat{x}[j]$となる.

%----------------------------------------------
\section{本章のまとめ}
%----------------------------------------------
本章では,周波数領域SSの定式化を行い,その代表的な手法,そしてILRMAで導入された手法について説明した.また,複素Student's $t$分布を生成モデルに持つようなILRMAである$t$-ILRMAについて説明した.心拍推定値を算出する心拍推定アルゴリズムについても説明した.次章では,\ref{chap:measurementstructsig}節で得られた測定信号に,\ref{sec:conv:iva}節で説明したIVAを適用した結果について実験を行う.

\chapter{観測信号にフィルタを適用しない場合のBSS及び心拍推定実験}
\label{chap:bsshrexp}

%----------------------------------------------
\section{まえがき}
%------------------------------------------
本章では,測定信号に対して,\ref{sec:conv:iva}節IVAを適用した結果をスペクトログラムに描画し,心拍推定アルゴリズムによって心拍推定結果を比較した結果を説明する.
心拍推定の比較には接触型ECGセンサの測定値を使用している.
まず\ref{sec:conv:expcondition4}節では,IVAを適用させる際のパラメタについて述べる.
\ref{sec:conv:expresult4}節では,IVAを適用させて得られたスペクトログラムと心拍推定グラフについての解説を述べる.

%----------------------------------------------
\section{実験条件}
\label{sec:conv:expcondition4}
%----------------------------------------------
STFTの窓長を確定するために一つの測定信号に対し4パターンの窓長に変更しIVAを適用した.実験条件はTable\ref{tab:iva}に示す.
%-%-%-%-%-%-%-%-%
\begin{table}[t]
  \caption{Experimental conditions of IVA}
  \centering
  \begin{tabular}{cc} \hline
    Parameter & Value \\ \hline \hline
    FFT length & 32, 64, 96, 128 samples  \\ \hline
    Shift length & FFt length/16 samples \\ \hline
    Channel number(Channel number: $N$) & 4 \\ \hline
    frequency bins(frequency bins: $I$) & 17, 33, 49, 65 \\ \hline
    Initial value of separating matrix & $N \times N \times I$ identity matrix  \\ \hline
    Initial value per frequency & $N \times N$ identity matrix  \\ \hline
    Number of iterations & 100\\ \hline
  \end{tabular}
  \label{tab:iva}
\end{table}
%-%-%-%-%-%-%-%-%

%----------------------------------------------
\section{実験結果}
\label{sec:conv:expresult4}
%----------------------------------------------
Fig. \ref{fig:siva32obs}は観測信号Back no.~2のスペクトログラムである.Fig. \ref{fig:siva32est} $\sim$ Fig. \ref{fig:siva128est}はIVAを適用した観測信号のスペクトログラムで,FFTサイズを32$\sim$128に変更したものである.Fig. \ref{fig:hrivach}は心拍推定アルゴリズムによって推定された心拍値でチャネルごとの推定結果を比較している.Fig. \ref{fig:hva32ch3}がリファレンスの心拍値を最もとらえていることがわかる.その他のデータからも同様のことがいえるため,以後の実験結果は,チャネル3について述べる.Fig. \ref{fig:hrivafftsize}はFFTサイズを変更させて心拍推定値の比較を行っているものである.$\mathrm{FFT~size}=32,~64$の場合,リファレンスの心拍値をとらえていることがわかる.その他のデータも同様の比較を行った結果,$\mathrm{FFT~size}=64$の場合,リファレンスの心拍値を最もとらえられていたため,以後の実験結果は,$\mathrm{FFT~size}=64$について述べる.

%-%-%-%-%-%-%-%-%
\begin{figure}[tb]
\centering
\includegraphics[width=1.0\hsize]{./ch_conventional/fig/spect_iva_32_obs.pdf}
\caption{Observation back signal no.~2.}
\label{fig:siva32obs}
\end{figure}
%-%-%-%-%-%-%-%-%

%-%-%-%-%-%-%-%-%
\begin{figure}[tb]
\centering
\includegraphics[width=1.0\hsize]{./ch_conventional/fig/spect_iva_32_est.pdf}
\caption{Estimated back signal no.~2, where FFT size is set to 32.}
\label{fig:siva32est}
\end{figure}
%-%-%-%-%-%-%-%-%

%-%-%-%-%-%-%-%-%
\begin{figure}[tb]
\centering
\includegraphics[width=1.0\hsize]{./ch_conventional/fig/spect_iva_64_est.pdf}
\caption{Estimated back signal no.~2, where FFT size is set to 64.}
\label{fig:siva64est}
\end{figure}
%-%-%-%-%-%-%-%-%

%-%-%-%-%-%-%-%-%
\begin{figure}[tb]
\centering
\includegraphics[width=1.0\hsize]{./ch_conventional/fig/spect_iva_96_est.pdf}
\caption{Estimated back signal no.~2, where FFT size is set to 96.}
\label{fig:siva96est}
\end{figure}
%-%-%-%-%-%-%-%-%

%-%-%-%-%-%-%-%-%
\begin{figure}[tb]
\centering
\includegraphics[width=1.0\hsize]{./ch_conventional/fig/spect_iva_128_est.pdf}
\caption{Estimated back signal no.~2, where FFT size is set to 128.}
\label{fig:siva128est}
\end{figure}
%-%-%-%-%-%-%-%-%

%-%-%-%-%-%-%-%-%
\begin{figure}[htbp]
      \begin{minipage}[t]{0.45\hsize}
        \centering
        \includegraphics[keepaspectratio, width=6.5cm]{./ch_conventional/hrfig/hr_iva_32_ch1.pdf}
        \subcaption{Ch1}
        \label{fig:hva32ch1}
      \end{minipage} 
      \begin{minipage}[t]{0.45\hsize}
        \centering
        \includegraphics[keepaspectratio, width=6.5cm]{./ch_conventional/hrfig/hr_iva_32_ch2.pdf}
        \subcaption{Ch2}
        \label{fig:hva32ch2}
      \end{minipage} \\
      \begin{minipage}[t]{0.45\hsize}
        \centering
        \includegraphics[keepaspectratio, width=6.5cm]{./ch_conventional/hrfig/hr_iva_32_ch3.pdf}
        \subcaption{Ch3}
        \label{fig:hva32ch3}
      \end{minipage} 
      \begin{minipage}[t]{0.45\hsize}
        \centering
        \includegraphics[keepaspectratio, width=6.5cm]{./ch_conventional/hrfig/hr_iva_32_ch4.pdf}
        \subcaption{Ch4}
        \label{fig:hva32ch4}
      \end{minipage} 
     \caption{Estimated (red) and reference (blue) heart rates of back signal no.~2 obtained by IVA.}
     \label{fig:hrivach}
  \end{figure}
%-%-%-%-%-%-%-%-%

%-%-%-%-%-%-%-%-%
\begin{figure}[htbp]
      \begin{minipage}[t]{0.45\hsize}
        \centering
        \includegraphics[keepaspectratio, width=6.5cm]{./ch_conventional/hrfig/hr_iva_32_ch3.pdf}
        \subcaption{FFT size is set to 32}
        \label{fig:hva32ch3}
      \end{minipage} 
      \begin{minipage}[t]{0.45\hsize}
        \centering
        \includegraphics[keepaspectratio, width=6.5cm]{./ch_conventional/hrfig/hr_iva_64_ch3.pdf}
        \subcaption{FFT size is set to 64}
        \label{fig:hva64ch3}
      \end{minipage} \\
   
      \begin{minipage}[t]{0.45\hsize}
        \centering
        \includegraphics[keepaspectratio, width=6.5cm]{./ch_conventional/hrfig/hr_iva_96_ch3.pdf}
        \subcaption{FFT size is set to 96}
        \label{fig:hva96ch3}
      \end{minipage} 
      \begin{minipage}[t]{0.45\hsize}
        \centering
        \includegraphics[keepaspectratio, width=6.5cm]{./ch_conventional/hrfig/hr_iva_128_ch3.pdf}
        \subcaption{FFT size is set to 128}
        \label{fig:hva128ch3}
      \end{minipage} 
     \caption{Estimated (red) and reference (blue) heart rates of back channel 3 signal no.~2 obtained by IVA.}
     \label{fig:hrivafftsize}
  \end{figure}
%-%-%-%-%-%-%-%-%

%----------------------------------------------
\section{本章のまとめ}
%----------------------------------------------
本章では,測定信号に対してIVAを適用した,心拍推定グラフを得た.まず,4チャネルある信号のうち最も分離されているチャネルを確認した.そして,FFTサイズを変更しIVAを適用することで,最も分離制度の高い心拍推定値を得られるFFTサイズを確認した.次章では,スペクトログラムの約0.3Hzに確認できる呼吸をカットするようなフィルタを設計する.また,前処理を施した測定信号に\ref{chap:methods}章で述べた各手法を適用した結果について述べる.


\chapter{観測信号にフィルタを適用した場合のBSS及び心拍推定実験}
\label{chap:fbsshrexp}

%----------------------------------------------
\section{まえがき}
%----------------------------------------------
本章では,測定信号に対して,フィルタをかけたIVA,基底数固定型ILRMA,基底数可変型ILRMA,$t$-ILRMAを適用した結果をスペクトログラムに描画し,心拍推定アルゴリズムによって心拍推定結果を比較した結果を説明する.
心拍推定の比較には接触型ECGセンサの測定値を使用している.
まず\ref{sec:conv:motivation}節では,\ref{chap:bsshrexp}章で得られた結果からフィルタリングを行った動機について述べる.
\ref{sec:conv:filteroutline}節では,測定信号にかけるフィルタの設計について述べる.
\ref{sec:conv:expcond5}節では,IVA,基底数固定型ILRMA,基底数可変型ILRMA,$t$-ILRMAを適用させる際のパラメタについて述べる.

%----------------------------------------------
\section{動機}
\label{sec:conv:motivation}
%----------------------------------------------
Fig. \ref{fig:siva32obs} $\sim$ Fig. \ref{fig:siva128est}から約0.3Hzに確認できる呼吸が心拍推定値に影響を及ぼしていると考えたため,次節で述べるようなフィルタを前処理として測定信号に適用した.次節で適用したフィルタについて述べる.

%----------------------------------------------
\section{設計したフィルタ}
\label{sec:conv:filteroutline}
%----------------------------------------------
本研究では,等リップルハイパスFIRディジタルフィルタを用いた.カットオフ周波数は1.5Hz及びフィルタ次数は170としている.このフィルタの振幅応答と位相応答をFig.\ref{fig:ampres}とFig.\ref{fig:phaseres}に示す.

%-%-%-%-%-%-%-%-%
\begin{figure}[!t]
\centering
\includegraphics[width=1.0\hsize]{./ch_conventional/fig/ampprop.pdf}
\caption{Amplitude responses.}
\label{fig:ampres}
\end{figure}
%-%-%-%-%-%-%-%-%

%-%-%-%-%-%-%-%-%
\begin{figure}[!t]
\centering
\includegraphics[width=1.0\hsize]{./ch_conventional/fig/phaseprop.pdf}
\caption{Phase responses.}
\label{fig:phaseres}
\end{figure}
%-%-%-%-%-%-%-%-%

%----------------------------------------------
\section{実験条件}
\label{sec:conv:expcond5}
%----------------------------------------------

基底数固定型ILRMAは反復毎にプロジェクションバックで正規化し,反復毎にプロジェクションバックで正規化しない.基底数固定型ILRMA,基底数可変型ILRMA及び$t$-ILRMAの実験条件はそれぞれ表(\ref{tab:fixedilrma}),表(\ref{tab:variableilrma})及び表(\ref{tab:tilrma})に示す.
%-%-%-%-%-%-%-%-%
\begin{table}[t]
  \caption{Experimental conditions of ILRMA without partitioning function}
  \centering
  \begin{tabular}{cc} \hline
    Parameter & Value \\ \hline \hline
    FFT length & 64 samples  \\ \hline
    Shift length & 4 samples \\ \hline
    Channel number(Channel number: $N$) & 4 \\ \hline
    frequency bins(frequency bins: $I$) & 33 \\ \hline
    Initial value of separating matrix & $N \times N \times I$ identity matrix  \\ \hline
    Initial value per frequency & $N \times N$ identity matrix  \\ \hline
    Number of iterations & 100\\ \hline
    Basis number & 3 \\ \hline \hline
  \end{tabular}
  \label{tab:fixedilrma}
\end{table}
%-%-%-%-%-%-%-%-%
%-%-%-%-%-%-%-%-%
\begin{table}[t]
  \caption{Experimental conditions of ILRMA with partitioning function}
  \centering
  \begin{tabular}{cc} \hline
    Parameter & Value \\ \hline \hline
    FFT length & 64 samples  \\ \hline
    Shift length & 4 samples \\ \hline
    Channel number(Channel number: $N$) & 4 \\ \hline
    frequency bins(frequency bins: $I$) & 33 \\ \hline 
    Initial value of separating matrix & $N \times N \times I$ identity matrix  \\ \hline
    Initial value per frequency & $N \times N$ identity matrix  \\ \hline
    Number of iterations & 100\\ \hline
    Basis number(Basis number: $K$) & 12 \\ \hline
    Basis matrix(Basis matrix: $T$) & Uniform random values $(0,1)$ \\ \hline
    Activation matrix(Activartion matrix: $V$) & Uniform random values $(0,1)$ \\ \hline 
    Partition function matrix(Partition function matrix: $Z$) & Uniform random values $(0,1)$ \\ \hline \hline
  \end{tabular}
  \label{tab:variableilrma}
\end{table}
%-%-%-%-%-%-%-%-%
%-%-%-%-%-%-%-%-%
\begin{table}[t] 
  \caption{Experimental conditions of $t$-ILRMA}
  \centering
  \begin{tabular}{cc}\hline
    Parameter & Value \\ \hline \hline
    FFT length & 64 samples  \\ \hline
    Shift length & 4 samples \\ \hline
    Channel number(Channel number: $N$) & 4 \\ \hline
    frequency bins(frequency bins: $I$) & 33 \\ \hline
    Initial value of separating matrix & $N \times N \times I$ identity matrix  \\ \hline
    Initial value per frequency & $N \times N$ identity matrix  \\ \hline
    Number of iterations & 100\\ \hline
    Freedom of student's $t$-distribution & 1, 2, 5\\ \hline
    Domains in low-rank modeling of signal source & 1(Amplitude domain), 2(Power dmain)\\ \hline
  \end{tabular}
  \label{tab:tilrma}
\end{table}
%-%-%-%-%-%-%-%-%

%----------------------------------------------
\section{実験結果}
\label{sec:conv:expresult5}
%----------------------------------------------

%----------------------------------------------
\subsection{IVAを適用した結果}
\label{sec:conv:resultiva}
%----------------------------------------------
Fig. \ref{fig:sfiva64obs}は\ref{sec:conv:filteroutline}節のフィルタを前処理として適用した測定信号である.Fig. \ref{fig:sfiva64est}はFig. \ref{fig:sfiva64obs}にIVAを適用した結果である.Fig. \ref{fig:hriva64ch3}はフィルタリングしていない推定信号に対して心拍推定アルゴリズムを適用した推定心拍値である.Fig. \ref{fig:fhriva64ch3}はフィルタリングした推定信号に対して心拍推定アルゴリズムを適用した結果である.Fig. \ref{fig:fhriva64ch3}はよりリファレンスデータの推定心拍値に近い値を示していることが分かる.そのためフィルタリングを行うことで推定精度が向上するといえる.しかし,振動の開始時からの約2分間と終了時は体動が大きくなることから,推定心拍値がレファレンスデータに対して大きな差が表れている.

%-%-%-%-%-%-%-%-%
\begin{figure}[tb]
\centering
\includegraphics[width=1.0\hsize]{./ch_conventional/fig/spect_iva_filter_64_obs.pdf}
\caption{Observated back signal no.~2 with filter.}
\label{fig:sfiva64obs}
\end{figure}
%-%-%-%-%-%-%-%-%

%-%-%-%-%-%-%-%-%
\begin{figure}[tb]
\centering
\includegraphics[width=1.0\hsize]{./ch_conventional/fig/spect_iva_filter_64_est.pdf}
\caption{Estimated back signal no~2 with filter}
\label{fig:sfiva64est}
\end{figure}
%-%-%-%-%-%-%-%-%

%-%-%-%-%-%-%-%-%
\begin{figure}[htbp]
 \begin{minipage}{0.5\hsize}
  \begin{center}
   \includegraphics[width=70mm]{./ch_conventional/hrfig/hr_iva_64_ch3.pdf}
  \end{center}
  \caption{Estimated heart rate obtained by IVA.}
  \label{fig:hriva64ch3}
 \end{minipage}
 \begin{minipage}{0.5\hsize}
  \begin{center}
   \includegraphics[width=70mm]{./ch_conventional/hrfig/hr_iva_filter_64_ch3.pdf}
  \end{center}
  \caption{Estimated heart rate obtained by IVA with filter.}
  \label{fig:fhriva64ch3}
 \end{minipage}
\end{figure}
%-%-%-%-%-%-%-%-%

%----------------------------------------------
\subsection{基底数固定型ILRMAを適用した結果}
\label{sec:conv:resultilrma1}
%----------------------------------------------
Fig. \ref{fig:silrma1}は基底数固定型ILRMAを適用した推定信号である.チャネル3のスペクトログラムに心拍成分が多く分離されているが,振動開始時と終了時の振動成分が強力であるため,分離しきれず残留している.その他のチャネルには振動成分が分離されている.また,チャネル2の3$\sim$5Hzにも心拍成分が分離してしまっている.Fig. \ref{fig:hrilrma1}は推定信号に対して心拍推定アルゴリズムを適用した推定心拍値である.IVAを適用したFig. \ref{fig:fhriva64ch3}と比較すると,非常にリファレンスデータに近い心拍推定値となっている.しかし,振動の開始時と終了時はリファレンスデータに対して心拍推定値に差が表れている.

%-%-%-%-%-%-%-%-%
\begin{figure}[tb]
\centering
\includegraphics[width=1.0\hsize]{./ch_conventional/fig/spect_ILRMA1_64_est.pdf}
\caption{Estimated signal obtained by ILRMA without partitioning function.}
\label{fig:silrma1}
\end{figure}
%-%-%-%-%-%-%-%-%

%-%-%-%-%-%-%-%-%
\begin{figure}[tb]
\centering
\includegraphics[width=1.0\hsize]{./ch_conventional/hrfig/hr_ILRMAtype1_64_ch3.pdf}
\caption{Estimated heart rate obrtained by ILRMA without partitioning function.}
\label{fig:hrilrma1}
\end{figure}
%-%-%-%-%-%-%-%-%

%----------------------------------------------
\subsection{基底数可変型LRMAを適用した結果}
\label{sec:conv:resultilrma2}
%----------------------------------------------
Fig. \ref{fig:silrma2}は基底数可変型ILRMAを適用した推定信号である.チャネル3のスペクトログラムに心拍成分が多く分離されている.また,Fig. \ref{fig:silrma1}と比較するとチャネル2に分離されていた心拍成分がチャネル3におおよそ分離できている.Fig. \ref{fig:hrilrma2}は推定信号に対して心拍推定アルゴリズムを適用した推定心拍値である.基底数固定型ILRMAを適用したFig. \ref{fig:heilrma1}と比較すると,振動開始時に大きく差が表れていた心拍値がよりリファレンスデータの心拍値に近い値を示している.

%-%-%-%-%-%-%-%-%
\begin{figure}[tb]
\centering
\includegraphics[width=1.0\hsize]{./ch_conventional/fig/spect_ILRMA2_64_est.pdf}
\caption{Estimated signal obtained by ILRMA with partitioning function.}
\label{fig:silrma2}
\end{figure}
%-%-%-%-%-%-%-%-%

%-%-%-%-%-%-%-%-%
\begin{figure}[tb]
\centering
\includegraphics[width=1.0\hsize]{./ch_conventional/hrfig/hr_ILRMAtype2_64_ch3.pdf}
\caption{Estimated heart rate obtained by ILRMA with partitioning function.}
\label{fig:hrilrma2}
\end{figure}
%-%-%-%-%-%-%-%-%

%----------------------------------------------
\subsection{$t$-ILRMAを適用した結果}
\label{sec:conv:resulttilrma}
%----------------------------------------------
Fig. \ref{fig:stilrmaa5}は$p=1$, $\nu = 5$,\ref{fig:stilrmap5}は$p=2$, $\nu = 5$で$t$-ILRMAを適用した推定信号である.Fig. \ref{fig:silrma1}, Fig. \ref{fig:silrma2}よりチャネル3に品p九成分が分離できている.Fig. \ref{fig:hrtilrmaa5}, Fig. \ref{fig:hrtilrmap5}はFig. \ref{fig:stilrmaa5},\ref{fig:stilrmap5}に心拍推定アルゴリズムを適用した推定心拍値である.$p$の値を変更したことによる心拍推定結果の差はほとんど見られない.Fig. \ref{fig:hrilrma1}に比べて,振動開始時の心拍推定値のリファレンスデータの心拍値に対する差が小さい.Fig. \ref{fig:hrilrma2}では振動終了時にリファレンスデータの心拍値と差があったが

%-%-%-%-%-%-%-%-%
\begin{figure}[tb]
\centering
\includegraphics[width=1.0\hsize]{./ch_conventional/fig/spect_tILRMA_ampdom_dofp5_64_est.pdf}
\caption{Estimated signal obtained by $t$-ILRMA, where $p=1$ and $\nu=5$.}
\label{fig:stilrmaa5}
\end{figure}
%-%-%-%-%-%-%-%-%

%-%-%-%-%-%-%-%-%
\begin{figure}[tb]
\centering
\includegraphics[width=1.0\hsize]{./ch_conventional/fig/spect_tILRMA_powdom_dofp5_64_est.pdf}
\caption{Estimated signal obtained by $t$-ILRMA, where $p=2$ and $\nu=5$.}
\label{fig:stilrmap5}
\end{figure}
%-%-%-%-%-%-%-%-%

%-%-%-%-%-%-%-%-%
\begin{figure}[tb]
\centering
\includegraphics[width=1.0\hsize]{./ch_conventional/hrfig/hr_tILRMA_ampdom_dofp5_64_ch3.pdf}
\caption{Estimated heart rate obtained by $t$-ILRMA, where $p=1$ and $\nu=5$.}
\label{fig:hrtilrmaa5}
\end{figure}
%-%-%-%-%-%-%-%-%

%-%-%-%-%-%-%-%-%
\begin{figure}[tb]
\centering
\includegraphics[width=1.0\hsize]{./ch_conventional/hrfig/hr_tILRMA_powdom_dofp5_64_ch3.pdf}
\caption{Estimated heart rate obtained by $t$-ILRMA, where $p=2$ and $\nu=5$.}
\label{fig:hrtilrmap5}
\end{figure}
%-%-%-%-%-%-%-%-%

%----------------------------------------------
\section{本章のまとめ}
%----------------------------------------------
本章では,\ref{chap:bsshrexp}章の測定信号に混在している呼吸をカットするためのフィルタについて説明した.また,フィルタリングした測定信号に対して\ref{chap:methods}章で述べた各手法を適用し,スペクトログラムと心拍推定グラフを出力し結果を説明した.$t$-ILRMAが最も分離制度が高いことを示した.