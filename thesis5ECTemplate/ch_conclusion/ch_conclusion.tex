\chapter{結言}
\label{chap:con}

本論文では,高精度な心拍推定が可能であるBSSの手法について調査した.

\ref{chap:measurementstructsig}章では,測定条件と振動測定系からで得られる信号について述べた.また本研究の正解値として利用した接触型ECGセンサについて説明した.

\ref{chap:methods}章では,前章で説明した振動測定系から得られる信号にBSSを適用する妥当性及び動機について述べた.また時間周波数領域BSSの定式化を行い,BSSの代表的な手法であるIVAやILRMAおよびILRMAに複素Student's $t$分布を生成モデルとして持たせた$t$-ILRMAの解説を行った.また,前述のBSSの手法によって
分離された信号から心拍値を推定するための心拍推定アルゴリズムの説明を行った.

\ref{chap:bsshrexp}章では,観測信号にIVAを適用し,心拍信号が最も良く分離されるSTFTの窓長の調査を行った.また,すべてのスペクトログラムに呼吸に起因するノイズ信号が強く残留しているため,前処理として呼吸に起因するノイズ信号を落とすようなハイパスフィルタを設計する必要があることが分かった.

\ref{chap:fbsshrexp}章では,前章の結果から呼吸に金するノイズ信号を落とすような170次FIRディジタルハイパスフィルタを設計した.前処理を適用した観測信号にIVA,ILRMA(基底数固定型及び基底数可変型を含む),及び$t$-ILRMAを適用した.これらの手法の中で$t$-ILRMAが最も心拍信号の分離精度が高いことが確認できた.

最後に今後の課題を述べる.\ref{chap:measurementstructsig}章で述べた測定信号は振動台によるsin波の単軸加振であるため,実際の車の走行で得られる振動とは異なるものである.本論文では,雑音が混在した信号から心拍信号を取り出すことが可能であることを明らかにした.今後は実測データに対しても同等以上の分離精度を持つ手法についての調査を行う.