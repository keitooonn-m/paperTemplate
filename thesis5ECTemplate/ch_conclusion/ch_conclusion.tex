\chapter{結言}
\label{chap:con}

本論文では,高精度な心拍推定が可能であるBSSの手法について調査した.

\ref{chap:measurementstructsig}章では,測定条件と測定で得られる信号について述べた.

\ref{chap:methods}章では,BSSの定式化を行い,BSSの代表的な手法であるIVAやILRMAおよびILRMAに複素Student's $t$分布を生成モデルとして持たせた$t$-ILRMAの解説を行った.また心拍推定アルゴリズムの解説を行った.

\ref{chap:bsshrexp}章では,測定信号にIVAを適用し,分離制度が最も高い測定信号のチャネルとFFTサイズの調査を行った.

\ref{chap:fbsshrexp}章では,呼吸成分をカットするためのフィルタの解説を行った.また,設計したフィルタを前処理として測定信号に適用し,IVAや基底数固定ILRMA, 基底数可変型ILRMA,$t$-ILRMAの分離制度を比較した.

最後に今後の課題を述べる.\ref{chap:measurementstructsig}章で述べた測定信号は振動台によるsin波の短軸家臣であるため,実際の車の走行で得られる振動とは異なるものである.本論文では,雑音が混在した信号から生体信号を取り出すことが可能であることを明らかにした.今後は実測データに対しても同等以上の分離制度を持つ手法についての調査を行う.