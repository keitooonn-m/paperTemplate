%#!pdfpLaTeX
%
% 北村研究室用卒業論文・特別論文のTeXテンプレートファイル
% 本ファイルは非公式であり,表紙とアブストに関しては下記で公開されているワードの
% テンプレートを利用して作成したものが公式であるので,表紙とアブストはPDFにして
% 差し替えること.
% https://www.kagawa-nct.ac.jp/EE/local/index.html (学内限定アクセス)
%
% 2020年1月17日 北村大地作成
%

%%%%%%%%%%%%%%%%%%%%%%%%%%% 論文情報 %%%%%%%%%%%%%%%%%%%%%%%%%%%
%%%%% テンプレート選択 %%%%%
\documentclass[honka]{nitkagawathesis}%卒論(本科5年)日本語用
%\documentclass[honka,english]{nitkagawathesis}%卒論(本科5年)英語用
%\documentclass[senkouka]{nitkagawathesis}%特論(専攻科2年)日本語用
%\documentclass[senkouka,english]{nitkagawathesis}%特論(専攻科2年)英語用

%%%%% タイトル %%%%%
\title{ブラインド信号源分離に基づく\\心拍推定}
%\titlewidth{}% タイトル幅 (指定するときは単位つきで)

%%%%% 著者 %%%%%
\author{村田 佳斗}
\eauthor{Keito Muarta}% Copyright表示で使われる

%%%%% 指導教員名 %%%%%
\supervisor{北村 大地 講師}% 1つ引数をとる (役職まで含めて書く)

%%%%% 副査教員名 %%%%%
\reviewer{柿元 健 准教授}% 1つ引数をとる (役職まで含めて書く)

%%%%% 学科長教員名 %%%%%
\chairperson{辻 正敏 教授}% 1つ引数をとる (役職まで含めて書く)

%%%%% 提出年月 %%%%%
\date{令和X年X月X日} % 和暦表示(卒論はこっちが正しい)
%\handin{2020}{2} % 西暦表示


\renewcommand{\figurename}{Fig.} % 図中の文字や図キャプションを日本語で書く場合はこの行をコメントアウト(原則英語とする)
\renewcommand{\tablename}{Table} % 表中の文字や表キャプションを日本語で書く場合はこの行をコメントアウト(原則英語とする)
\usepackage{tabularx}
\usepackage{array}
\usepackage{subcaption}

%%%%% \usepackage等のプリアンブル宣言(macros.texに記載) %%%%%
\input{macros.tex}

\begin{document}
\bstctlcite{IEEEexample:BSTcontrol} % BibTeXのIEEEtranで同一著者の横線表示を防止

\maketitle% タイトル生成

%%%%%%%%%%%%%%%%%%%%%%%%%%% 前文 %%%%%%%%%%%%%%%%%%%%%%%%%%%
\frontmatter

%%%%% English title %%%%%
\etitle{Heart Rate Estimation Based on \\
Blind Source Separation}

%%%%% Abstract %%%%%
\eabstract{ % これ単体で複数ページにまたがる場合はエラーが出るので注意,アブスト内で改段落は禁止
While driving a motor vehicle, the driver may suffer from sleep, sudden seizure, or loss of consciousness due to deterioration of physical condition, which is often fatal. Fatal accidents caused directly by the driver's condition account for a large percentage of all fatal accidents. Therefore, in modern society, where automobiles are indispensable to people's lives, one of the important issues is to manage the driver's condition in some way while driving.
In order to solve this problem, we are dealing with a system that constantly monitors the driver's heartbeat during driving using a radar non-contact biometric sensor array (hereafter referred to as a radar sensor). In this research, a vibration measurement system with a seat on a vibrating plate is constructed to simulate the vibration during driving. The radar sensor is installed on the seat where the driver is seated, and measures the driver's body surface micro-displacement.
However, in addition to the driver's heart rate signal, which is the target of the measurement, the vibration of the measurement system itself and the relative position variation between the sensor and the target are measured, so noise signals other than the heart rate signal are measured at the same time.
A technique to extract only the target sound source from a signal source with mixed noise has been widely studied in the field of acoustic signal processing. In this thesis, we consider the application of blind source separation (BSS), which has a particularly long history of advanced development. We apply independent vector analysis (IVA) and independent low-rank matrix analysis (ILRMA), which are typical methods of BSS, and $t$-ILRMA, which has complex Student's $t$ distribution as a generating model, to ILRMA. We also applied each method. In order to compare the separation accuracy of the estimated signals using each method, we utilize a heart rate estimation algorithm and compare the results with heart rate values measured by a contact heart rate monitor.
}

%%%%% 概要 %%%%%
\abstract{ % これ単体で複数ページにまたがる場合はエラーが出るので注意,アブスト内で改段落は禁止
自動車の運転中に運転者が睡眠,突発的な発作,体調の悪化による意識喪失等に見舞われることは多くの場合致命的な状況となる.このような運転者の状況が直接的な要因となって引き起こされる死亡事故は,全死亡事故の内の少なくない割合を占めている.そのため,人々の暮らしに自動車が欠かせない現代社会においては,運転中に運転者の状態を何らかの方法で管理することが重要課題の一つとなっている.
この問題を解決するために,運転中の運転者の心拍をレーダ非接触型生体センサアレイ(以後,レーダセンサと呼ぶ)を活用した常時モニタリングシステムを取り扱う.本研究では運転中の振動を模擬するために,振動版の上にシートを設置した振動測定系を構築している.このレーダセンサは運転者の着座するシートに設置し,運転者の体表面微小変位を測定する.
しかしながら,測定時に目的としている運転者の心拍信号以外にも測定系自体の振動とセンサと対象の相対的位置の変動が測定しまうため心拍信号以外の雑音信号も同時に測定されてしまう.
雑音が混在した信号源から目的音源のみを抽出する技術は音響信号処理の分野で広く研究されている.本論文ではその中でも特に高度に発展してきた歴史を持つブラインド信号源分離(blind source separation: BSS)の適用を考える.BSSは信号源の混合系に関する事前情報を全く用いずに信号源を分離する技術である.BSSの代表的手法である独立ベクトル分析(independent vector analysis: IVA)と独立低ランク行列分析(independent low-rank matrix analysis: ILRMA),ILRMAに複素Student's $t$分布を生成モデルに持つような$t$-ILRMAを適用する.また,各手法を適用した推定信号の分離精度を比較するために心拍推定アルゴリズムを活用し,接触型心拍計によって計測される心拍値との比較を行う.
実験結果より,呼吸に起因する雑音信号が分離精度に影響していると考え呼吸成分をカットするフィルタを設計し適用した.前処理としてフィルタリングを行い再度実験を行った結果,$t$-ILRMAの心拍制度が最も良いことを確認した.
本研究では,振動測定系を活用して測定を行ったため自動車の走行中の振動に対しても頑健な分離手法を提案することが今後の課題となる.
}

%\keywords{Keyword1, Keyword2, Keyword3}

\makeseparatedabstract % 英語アブストと日本語アブストをそれぞれ独立したページとする
%\makeabstract % 英語アブストと日本語アブスト合わせて1枚に収まる場合はこちらを使ってもよい,ただし2枚になる場合はエラーが出る

%%%%% 目次 %%%%%
% \setcounter{tocdepth}{3} \tableofcontents % ページ番号を削除しない目次
%----- ページ番号を削除した目次 -----%
{\makeatletter
  \let\ps@jpl@in\ps@empty
  \makeatother
  \pagestyle{empty}
  \setcounter{tocdepth}{3}
  \tableofcontents
  \clearpage}
%---------------------------------%

%%%%%%%%%%%%%%%%%%%%%%%%%%% 本文 %%%%%%%%%%%%%%%%%%%%%%%%%%%
\mainmatter

%%%%% 第1章 %%%%%
\chapter{緒言}
\label{chap:intro}

%----------------------------------------------
\section{本論文の背景}
%----------------------------------------------
自動車の運転中の発作や急病などが原因と思われる死亡事故は,死亡事故のうち10\%を占めている.防御策として運転者の心拍を常時モニタリングすることが挙げられる.運転中の運転者の心拍値を解析し,異常時には警告したり,自動車を停止させることが必要となる.この信号を取得して解析する工程が非常に困難である.理由としては,運転中には車体自体の振動や運転者の呼吸や体動による振動など,たくさんの振動ノイズが混在してしまうためである.ハンドルに心拍測定装置を埋め込んで測定することも可能ではあるが,運転者が運転中にハンドルを絶えず握っている保証はない.運転者に要求の少ない心拍測定方法として,本研究で議論する,シートにセンサーを埋め込む心拍測定方法を提案する.この方法では,運転者が意図せずとも心拍を測定することが可能となる.この測定系を用いた心拍推定法が必要となる.

%----------------------------------------------
\section{本論文の目的}
%----------------------------------------------
問題解決として,Fig. \ref{fig:sensorstructure}のようにレーダ非接触生体センサアレイ(以後,レーダセンサと呼ぶ)を車のシートに埋め込んだ振動測定系で観測信号から心拍推定を行う.レーダセンサはFig. \ref{fig:sensorimg}のように送信ビーム2指向性,受信アンテナ2本の合計4指向性における体表面変位を取得している.しかし,この測定系において以下に示す2点の問題点がある.

\begin{itemize}
 \item 測定系自体の外来振動の印加
 \item 測定系と運転者の呼吸や体動による相対的位置関係の変動
\end{itemize}

前述のように,レーダセンサはシートに埋め込んでいるため,走行時の車体振動が観測信号として取得されてしまう.また,体表面変位の変動を取得しているため,運転者の心拍だけでなく,呼吸や体動などの変動も同時に取得してしまう.そのため,本研究ではこれらの振動ノイズから目的信号である心拍信号を取り出すために信号源分離を行う.


%-%-%-%-%-%-%-%-%
\begin{figure}[htbp]
 \begin{minipage}{0.5\hsize}
  \begin{center}
   \includegraphics[width=70mm]{./ch_conventional/fig/sensorstructure.pdf}
  \end{center}
  \caption{Image of shaking table.}
  \label{fig:sensorstructure}
 \end{minipage}
 \begin{minipage}{0.5\hsize}
  \begin{center}
   \includegraphics[width=70mm]{./ch_conventional/fig/sensorimg.pdf}
  \end{center}
  \caption{Image of sensor motion and position.}
  \label{fig:sensorimg}
 \end{minipage}
\end{figure}
%-%-%-%-%-%-%-%-%


%----------------------------------------------
\section{本論文の構成}
%----------------------------------------------
\ref{chap:measurementstructsig}章では測定条件と測定で得られる信号について述べる.
\ref{chap:methods}章では本論文で取り扱うBSSの手法及び心拍推定アルゴリズムの解説を行う.
\ref{chap:bsshrexp}章では\ref{chap:methods}で解説したIVAを用いて分離制度の高いチャネルとFFTサイズの調査を行う.
\ref{chap:fbsshrexp}章では測定信号に前処理としてフィルタを適用し,IVAや基底数固定型ILRMA,基底数可変型ILRMA,$t$-ILRMAを用いて分離制度の比較を行う.
\ref{chap:con}章ではすべての章の総括した結言を述べる.

%%%%% 第2章 %%%%%
\chapter{振動測定系と測定信号}
\label{chap:measurementstructsig}

%----------------------------------------------
\section{まえがき}
%----------------------------------------------
本章では,本研究で扱う信号の測定条件と得られる信号について説明する.
{まず\ref{sec:conv:measurementcondition}節では,本研究で使用する観測信号の測定条件,及び心拍\textcolor{black}{の参照値}として用いる接触型ECGセンサについて説明する.
\ref{sec:conv:signal}節では,\ref{sec:conv:measurementcondition}節で述べた測定条件\textcolor{black}{とレーダセンサ}から得られる時間波形,及び時間波形にSTFTを適用して得られるパワースペクトログラム\textcolor{black}{を掲載し,考察を述べる}.また,接触型ECGセンサから得られる信号のパワースペクトログラムと比較することで\textcolor{black}{,}振動台の成分,心拍,呼吸,及び体動等のその他の成分を確認する.
\ref{sec:conv:conclusion2}節では,本章のまとめを述べる.}

%----------------------------------------------
\section{測定条件}
\label{sec:conv:measurementcondition}
%----------------------------------------------
		本研究で使用する信号の{振動測定系}はFig.~\ref{fig:sensorstructure}に示すとおりである.{運転者を模した}被験者が座った状態で{振動測定系}全体を振動させる.Fig.~\ref{fig:sensorimg}に示されている{レーダセンサ}を,{背部と臀部にあたるシートの内部に埋め込み},\textcolor{black}{計測}を行う.レーダセンサのサンプリング周波数は40~Hzである.{振動測定系の振動はsin波の単軸加振であり,Fig.~\ref{fig:sensorstructure}に示す通り,前後・上下・左右方向に振動\textcolor{black}{させることができる}.本研究で使用したデータはTable~\ref{tab:vibevacondition}に示すように\textcolor{black}{運転中の振動を模擬したData nos.~0--6の7種類の条件である}.Data no.~0は振動を加えずに運転者を模した被験者を計測しており,このデータのみ\textcolor{black}{計測}時間が\textcolor{black}{300~s}となっている.\textcolor{black}{Data nos.~0--6}のデータは\textcolor{black}{計測}時間が\textcolor{black}{420~s}である.\textcolor{black}{計測}開始から\textcolor{black}{60~s}は振動を加えていない状態で\textcolor{black}{計測}し,その後の\textcolor{black}{300~s}は振動を加えた状態,残りの\textcolor{black}{60~s}は再度,振動を加えていない状態で計測している.\textcolor{black}{振動台の振動はData nos.~1--6のいずれの条件も}上下方向に加えている.\textcolor{black}{Data nos.~1--4}の\textcolor{black}{振動台の振動の振幅}はそれぞれ,4~mm, 10~mm, 40~mm, \textcolor{black}{及び}40~mm\textcolor{black}{であり,振動台の振動の周波数は}1.2~Hz\textcolor{black}{である.また,Data nos.~5及び6}の\textcolor{black}{振動台の振動の振幅}はそれぞれ4~mm\textcolor{black}{及び}10~mm\textcolor{black}{であり,振動台の振動の周波数は}2.4~Hzとなっている.\textcolor{black}{なお},\textcolor{black}{Data no.~4に関してのみ振動台の振動}の\textcolor{black}{影響がどの程度軽減するか観測する目的で,}被験者はハンドルを握りシートベルトを着用している.}

{\textcolor{black}{本論文}では,分離した信号と比較する\textcolor{black}{参照値を得るために,}Fig.~\ref{fig:ecgsensor}に示す\textcolor{black}{Zephyr Technology}社の接触型ECGセンサBioharness~\cite{bioharness}を用いて\textcolor{black}{可能な限り正確なECG信号を取得している.心拍の参照値の計算には,このBioharness内部で実装されている推定アルゴリズムを用いる.Bioharnessの技術的な資料は公開されておらず原理は不明であるが,恐らく一般的な心拍推定アルゴリズムであるR-R間隔(R-R interval: RRI)推定に基づくものと予想される.}接触型ECG\textcolor{black}{センサ}であるため,振動台の振動が加えられても高精度な心拍を得ることが可能である.\textcolor{black}{本論文では,}この\textcolor{black}{接触型ECG}センサから得られる心拍と同程度の\textcolor{black}{精度で}レーダセンサ\textcolor{black}{の信号から心拍を推定することが目的となる}.接触型ECGセンサはTable~\ref{tab:ecgdata}に\textcolor{black}{例示するように},ECGセンサ\textcolor{black}{の出力データ},ピエゾセンサ\textcolor{black}{の出力データ},RRI),心拍数,呼吸数,及び呼吸の深さを測定する.また,サンプリング周波数は250~Hzである.}


%-%-%-%-%-%-%-%-%
\begin{table}[tb]
  \caption{\textcolor{black}{Evaluation conditions using vibration measurement sysytem.}}
  \centering
  \scalebox{1.0}[1.0]{
  \begin{tabular}{|c|c|c|c|c|c|} \hline
    \begin{tabular}{c} 
    Data \\
    no.~
    \end{tabular} &
    \begin{tabular}{c} 
    Vibration \\
    type
    \end{tabular} &
    \begin{tabular}{c} 
    Amplitude \\
    \relax [mm]
    \end{tabular} &
    \begin{tabular}{c}
    Frequency \\
    \relax [Hz]
    \end{tabular} &
    \begin{tabular}{c} 
    Stearing \\
    wheel
    \end{tabular} &
    Seatbelt \\ \hline \hline
    0 & No vibration & - & - & Not hold & Not fasten \\ \hline
    1 & & 4 & 1.2 & Not hold & Not fasten  \\ \cline{1-1} \cline{3-6}
    2 & & 10 & 1.2 & Not hold & Not fasten \\ \cline{1-1} \cline{3-6}
    3 & Up and Down & 40 & 1.2 & Not hold & Not fasten \\ \cline{1-1} \cline{3-6}
    4 & for 60--360~s & 40 & 1.2 & Hold & Fasten \\ \cline{1-1} \cline{3-6}
    5 & & 4 & 2.4 & Not hold & Not fasten \\ \cline{1-1} \cline{3-6}
    6 & & 10 & 2.4 & Not hold & Not fasten \\ \hline
  \end{tabular}
  }
  \label{tab:vibevacondition}
\end{table}
%-%-%-%-%-%-%-%-% 


%-%-%-%-%-%-%-%-%
\begin{figure}[tb]
\centering
\includegraphics[width=0.45\hsize]{./ch_conventional/fig/ecgsensor.pdf}
\caption{\textcolor{black}{Contact\textcolor{black}{-type} ECG sensor \cite{bioharness} \textcolor{black}{used to obtain driver's reference heart rate}.}}
\label{fig:ecgsensor}
\end{figure}
%-%-%-%-%-%-%-%-%

%-%-%-%-%-%-%-%-%
\begin{table}[tb]
  \caption{\textcolor{black}{\textcolor{black}{Example of output data} obtained by contact\textcolor{black}{-type} ECG sensor \textcolor{black}{(Bioharness)}.}}
  \centering
  \scalebox{1.0}[1.0]{
  \begin{tabular}{|c|c|c|c|c|c|} \hline
    \begin{tabular}{c} 
    Raw data \\
    obtained by \\
    ECG sensor \\
    \relax [mV]
    \end{tabular} &
    \begin{tabular}{c} 
    Raw data \\
    obtained by \\
    piezo sensor \\
    \relax [mV]
    \end{tabular} &
    \begin{tabular}{c}
    RRI \\
    \relax [s]
    \end{tabular} &
    \begin{tabular}{c}
    Heart \\
    rate \\
    \relax [bpm]
    \end{tabular} &
    \begin{tabular}{c}
    Breathing \\
    rate \\
    \relax [bpm]
    \end{tabular} &
    \begin{tabular}{c}
    Depth \\
    of \\
    breathing
    \end{tabular} \\ \hline \hline
    54.5852 & 0.0574969 & 0.759 & 91 & 18 & 11.864	\\ \hline
    54.4779 & 0.0574969 & 0.759 & 91 & 18 & 11.864	\\ \hline
    54.4779 & 0.0573748 & 0.759 & 91 & 18 & 11.864	\\ \hline
    \vdots & \vdots & \vdots & \vdots & \vdots & \vdots \\ \hline	
  \end{tabular}
  }
  \label{tab:ecgdata}
\end{table}
%-%-%-%-%-%-%-%-%

%----------------------------------------------
\section{レーダセンサの観測信号}
\label{sec:conv:signal}
%----------------------------------------------
\ref{sec:conv:measurementcondition}節の{振動測定系}によって得られる\textcolor{black}{レーダセンサの観測信号はFig.~\ref{fig:sensorimg}に示すように4チャネルの時間波形である.一例として,Tabel~\ref{tab:vibevacondition}に示すData no.~1の時間波形をFig.~\ref{fig:obssig}に示す.このとき,横軸及び縦軸はそれぞれ測定時間及び体表面の変位量となっている.\ref{sec:conv:measurementcondition}節で述べた通り,Fig.~\ref{fig:obssig}の\textcolor{black}{60~s--360~s}の間で\textcolor{black}{振動台の振動}が加えられているため時間波形が他の時間の時間波形とは異なっていることがわかる.\textcolor{black}{振動台の振動}が加えられていない時間での時間波形には,心拍,呼吸,及び体動\textcolor{black}{に由来する信号の}成分が現れている.また,レーダセンサはチャネル間で\textcolor{black}{比較的近傍の体表面}にビームを当てて\textcolor{black}{計測}しているため,完全に一致した時間波形ではないが,チャネル間に強い相関を持っていることも確認できる.}

{また,\textcolor{black}{Fig.~\ref{fig:obssig}}の観測信号にSTFTを適用して得られる,パワースペクトログラムをFig.~\ref{fig:4chobsspect}に\textcolor{black}{示す.ここで,STFTにおける短時間区間長及びシフト長はそれぞれ25.6~s及び0.8~sに設定し,窓関数にはハミング窓を用いている}.Fig.~\ref{fig:4chobsspect}は4チャネルのパワースペクトログラム\textcolor{black}{であり,各図の}カラーマップ(dB値と色の対応関係)は共通化している.Fig.~\ref{fig:1chobsspect}はFig.~\ref{fig:4chobsspect}のCh.~1のパワースペクトログラムに解説を加えたものである.60~sから360~sまでの間に確認できる約1.2~Hz及びその整数倍の2.4~Hz,3.6~Hz,4.8~Hz,6.0~Hz,及び7.2~Hzの直線の成分は振動台の振動成分である.また,呼吸の基本周波数成分が約0.3~Hzに存在し,その高周波成分が1.5~Hz,3.2~Hz,及び4.0~Hzに確認できる.本論文で分離・抽出する目的となる心拍信号は,基本周波数成分が約0.7~Hzに確認でき,さらにその高調波成分が2.3~Hz,3.6~Hz,5.0~Hz,及び6.0~Hz付近に確認できる.このように時間波形のスペクトログラムから,振動台の振動成分,心拍や呼吸由来の成分,及び体動等のその他の成分が確認できる.1章で述べた通り,観測信号としてはSNRが低く,この観測から直接心拍を推定することは困難であるため,BSSにより心拍信号を分離・抽出することの妥当性が伺える.}

%-%-%-%-%-%-%-%-%
\begin{figure}[b]
\centering
\includegraphics[width=1.0\hsize]{./ch_conventional/fig/obsSig.pdf}
\caption{\textcolor{black}{Four-channel observed signals obtained by radar sensor.}}
\label{fig:obssig}
\end{figure}
%-%-%-%-%-%-%-%-%

%-%-%-%-%-%-%-%-%
\begin{figure}[tb]
\centering
\includegraphics[width=1.0\hsize]{./ch_conventional/fig/obsSpect.pdf}
\caption{\textcolor{black}{Spectrograms of four-channel observed signals calculated by STFT.}}
\label{fig:4chobsspect}
\end{figure}
%-%-%-%-%-%-%-%

%-%-%-%-%-%-%-%-%
\begin{figure}[tb]
\centering
\includegraphics[width=1.0\hsize]{./ch_conventional/fig/1chobsspect.pdf}
\caption{\textcolor{black}{Each component in first channel spectrogram.}}
\label{fig:1chobsspect}
\end{figure}
%-%-%-%-%-%-%-%

\clearpage
%----------------------------------------------
\section{本章のまとめ}
\label{sec:conv:conclusion2}
%----------------------------------------------
\textcolor{black}{本章では,振動測定系の測定条件と測定によって得られる時間波形について説明した.また,観測信号から心拍信号を分離した際に比較を行うために使用した接触型ECGセンサから得られる信号ついて説明した.レーダセンサから得られる信号にSTFTを適用して得られたパワースペクトログラムと時間波形のパワースペクトログラムを比較し,振動台の振動成分,心拍,呼吸,及び体動等のその他の成分を確認した.
次章では,本研究でBSSを適用する動機と時間周波数領域BSSの定式化を行い,IVA,ILRMA,$t$-ILRMA,及び心拍推定アルゴリズムについて説明する.}


\chapter{適用手法}
\label{chap:methods}

%----------------------------------------------
\section{まえがき}
%----------------------------------------------
本章では,本研究で適用したBSSについて説明する.
まず\ref{sec:conv:researchmotivation}節では,{\ref{chap:measurementstructsig}章の振動測定系において運転者の体表面とレーダセンサの相対的な位置関係が未知であることから,BSSを観測信号に適用することの妥当性及び動機について述べる.}
\ref{sec:conv:bssformularization}節では,時間周波数領域におけるBSSの定式化を行う.
{\ref{sec:conv:iva}節では,音響信号処理分野で一般的な周波数領域BSSの一つであるIVAについて説明する.}
\ref{sec:conv:isnmf}節では,行列分解の手法の一つであるNMFのうちISNMFについて説明する
{\ref{sec:conv:ilrma}節及び\ref{sec:conv:tilrma}節では,本論文で適用するもう一つのBSSのILRMAで用いられる基礎理論として,ILRMA(基底数固定型及び基底数可変型)及び複素Student's $t$分布を生成モデルと仮定する$t$-ILRMAについてそれぞれ説明する.
\ref{sec:conv:heartrateestalgo}節では,\ref{sec:conv:iva}節,\ref{sec:conv:ilrma}節及び\ref{sec:conv:tilrma}節で説明する三つの手法によって得られる分離信号の心拍推定精度を確認するために用いる心拍推定アルゴリズムについて説明し,\ref{sec:conv:conclusion3}節で本章をまとめる.}

%----------------------------------------------
\section{本研究の動機}
\label{sec:conv:researchmotivation}
%----------------------------------------------
{1章で述べた運転者の車内での心拍推定という目的においては,運転者の体表面とレーダセンサの相対的な位置関係は未知としたまま心拍推定を行えることが望ましい.従って,混合系が未知であっても観測信号に含まれる各信号源を推定できる.BSSを\ref{chap:measurementstructsig}章で述べたレーダセンサの観測信号に適用することは妥当であると考えられる.BSSはとりわけ音響信号処理分野で高度に発展してきた歴史を持つ.音響信号処理分野では,信号源(音源)の混合が部屋の残響の影響を受けて,時間領域での畳み込み混合となってしまう.そのため,時間領域の信号ではなく時間周波数領域の信号(観測信号に対してSTFTを適用した信号)を対象とする時間周波数領域BSSが長く研究されており,近年においては,IVAやILRMA等の高度に発展したアルゴリズムが提案されている.本論文では,IVAやILRMA等の時間周波数領域BSSをレーダセンサの4チャネル観測信号に適用し,心拍信号及びノイズ信号の分離が可能であるか,実験的に調査する.}

%-%-%-%-%-%-%-%-%
\begin{figure}[!t]
\centering
\includegraphics[width=0.9\hsize]{./ch_conventional/fig/bss.pdf}
\caption{Process of BSS.}
\label{fig:bss}
\end{figure}
%-%-%-%-%-%-%-%-%


%----------------------------------------------
\section{BSSの定式化}
\label{sec:conv:bssformularization}
%----------------------------------------------
今,観測信号中に混合している信号源の数を$N$,観測信号のチャネル数(センサ数)を$M$と定義するとき,時間領域の信号源,観測信号,及び分離信号はそれぞれ次のよう表される.
\begin{align}
  \tilde{\bm{s}}[l] &= [ \tilde{s}_{1}[l], \tilde{s}_{1}[l], \cdots, \tilde{s}_{n}[l], \cdots, \tilde{s}_{N}[l] ]^\mathrm{T} \in \mathbb{R}^{N} \\
  \tilde{\bm{x}}[l] &= [ \tilde{x}_{1}[l], \tilde{x}_{1}[l], \cdots, \tilde{x}_{m}[l], \cdots, \tilde{x}_{M}[l] ]^\mathrm{T} \in \mathbb{R}^{M} \\
  \tilde{\bm{y}}[l] &= [ \tilde{y}_{1}[l], \tilde{y}_{1}[l], \cdots, \tilde{y}_{n}[l], \cdots, \tilde{y}_{N}[l] ]^\mathrm{T} \in \mathbb{R}^{N} 
\end{align}
ここで,$l = 1, 2, \cdots, L$,$n = 1, 2, \cdots, N$,及び$m = 1, 2, \cdots, M$は離散時間インデクス,信号源インデクス,及びチャネルインデクスをそれぞれ表す.
また,$\cdot^\mathrm{T}$はベクトル及び行列の転置である.
このとき,時間領域での時不変な(離散時間$l$に依存しない)瞬時混合系は次式で表される.
\begin{align}
  \tilde{\bm{x}}[l] = \tilde{\bm{A}} \tilde{\bm{s}}[l] \label{eq:mixtureica}
\end{align}
ここで,$\tilde{\bm{A}} \in \mathbb{R}^{M\times N}$は混合行列と呼ばれる.一方,分離信号は分離行列$\tilde{\bm{W}} \in \mathbb{R}^{N\times M}$を用いて次式で表せる.
\begin{align}
  \tilde{\bm{y}}[l] = \tilde{\bm{W}} \tilde{\bm{x}}[l] 
\end{align}
従って,もし混合行列が正方行列(即ち$M=N$)でかつ可逆であれば,$\tilde{\bm{W}} = \tilde{\bm{A}}^{-1}$なる分離行列を推定することで観測信号$( \tilde{\bm{x}}[l] )_{l=1}^L$から分離信号$( \tilde{\bm{y}}[l] )_{l=1}^L$を求めることができる.この様子をFig.~\ref{fig:bss}に示している.但し,通常は混合系に対応する$\tilde{\bm{A}}$は未知(ブラインド)であるため,信号源間の統計的独立性を仮定して$\tilde{\bm{W}}$を求めるアプローチが主流である.このような信号源間の独立性に基づくBSSには,独立成分分析(independent component analysis: ICA)\cite{ica}を起源としてこれまで長く研究されてきた歴史がある.

{音響信号処理分野におけるBSSでは,録音時の部屋の残響や音速の遅さ等に起因して,音源の混合現象は式\eqref{eq:mixtureica}時間領域での時不変瞬時混合ではなく,畳み混み混合になる.ICAでは畳み込み混合された信号源を分離できないため,観測信号$( \tilde{\bm{x}}[l] )_{l=1}^L$にSTFTを適用し,時間領域から時間周波数領域の信号に変換することが一般的である.この場合,時間領域での畳み込み混合は,時間周波数領域での瞬時混合とみなせるため,Fig.~\ref{fig:bss}に示すような混合系を周波数毎に考えればよい.
このような混合系のモデル化に基づくBSSを時間周波数領域BSSと呼び,音響信号処理においては効果的な音源分離を達成するための重要なモデルとなっている.本論文で取り扱うBSSであるIVA及びILRMAはいずれも時間周波数領域BSSの一つである.}

以下に,時間周波数領域BSSの定式化を行う.
信号源$( \tilde{\bm{s}}[l] )_{l=1}^L$,観測信号$( \tilde{\bm{x}}[l] )_{l=1}^L$,及び分離信号$( \tilde{\bm{y}}[l] )_{l=1}^L$のそれぞれに対して{STFTを適用したものを次式で定義する.}
\begin{align}
\bm{s}_{ij} &= [s_{ij1}, s_{ij2}, \cdots, s_{ijn}, \cdots, s_{ijN}]^{\mathrm{T}} \in \mathbb{C}^{N} \label{eq:s} \\
\bm{x}_{ij} &= [x_{ij1}, x_{ij2}, \cdots, x_{ijm}, \cdots, x_{ijM}]^{\mathrm{T}} \in \mathbb{C}^{M} \label{eq:x} \\
\bm{y}_{ij} &= [y_{ij1}, y_{ij2}, \cdots, y_{ijn}, \cdots, y_{ijN}]^{\mathrm{T}} \in \mathbb{C}^{N} \label{eq:y}
\end{align}
ここで,$i=1, 2,  \cdots, I$及び$j=1, 2,  \cdots, J$はそれぞれ周波数ビンインデクス及び時間フレームインデクスを表す.また,式\eqref{eq:s}--\eqref{eq:y}の各信号においては,時間周波数行列としての表記もそれぞれ$\bm{S}_n\in\mathbb{C}^{{I\times J}}$,$\bm{X}_m\in\mathbb{C}^{{I\times J}}$,及び$\bm{Y}_n\in\mathbb{C}^{{I\times J}}$として定義しておく.
{式(\ref{eq:s})--(\ref{eq:y})において},周波数毎の時不変な(時間フレーム$j$に依存しない)瞬時混合は,混合行列$\bm{A}_i \in \mathbb{C}^{M\times N}$を用いて次式で表せる.
\begin{align}
  \bm{x}_{ij} = \bm{A}_i \bm{s}_{ij} \label{eq:xas}
\end{align}
時間領域の場合と同様に$M=N$かつ$\bm{A}_i$がフルランクの場合は,分離行列$\bm{W}_{i} = [\bm{w}_{i1}~\bm{w}_{i2}~\cdots~ ~\bm{w}_{iN}]^{\mathrm{H}} \in \mathbb{C}^{N \times M} $が存在し,分離信号は次式で表せる.
\begin{align}
  \bm{y}_{ij} = \bm{W}_i \bm{x}_{ij} \label{eq:ywj}
\end{align}
ここで,$^{\mathrm{H}}$は行列及びベクトルのエルミート転置を表す.従って,
{時間周波数領域BSSは周波数毎の混合行列$\bm{A}_i$が未知の状態で周波数毎の分離行列$\bm{W}_{i} \approx \bm{A}_i^{-1}$}を全ての周波数$i=1, 2, \cdots, I$において推定する問題である.但し,そのままでは周波数毎に分離された信号源の順序($\bm{y}_{ij}$のベクトルの要素の順序)が周波数間で統一されない問題が生じる.これは時間周波数領域BSSにおけるパーミュテーション問題\cite{permute}と呼ばれ,これまで様々な解決法が提案されてきた(例えば\cite{persolve1,persolve2,persolve3,persolve4}等).近年では,信号源間の統計的独立性だけでなく,各信号源の時間周波数表現に何らかの構造を仮定することで,パーミュテーション問題を回避するBSSが主流となっている.特に,次節以降で述べるIVA及びILRMAが良く用いられる代表的な時間周波数領域BSSとなっている.


%----------------------------------------------
\section{IVA}
\label{sec:conv:iva}
%----------------------------------------------
%-%-%-%-%-%-%-%-%
\begin{figure}[htbp]
      \begin{minipage}[t]{0.45\hsize}
        \centering
        \includegraphics[keepaspectratio, width=6cm]{./ch_conventional/fig/mixingiva.pdf}
        \subcaption{Mixing model}
        \label{fig:mixingiva}
      \end{minipage} 
      \begin{minipage}[t]{0.45\hsize}
        \centering
        \includegraphics[keepaspectratio, width=6cm]{./ch_conventional/fig/demixingiva.pdf}
        \subcaption{Demixing model}
        \label{fig:demixingiva}
      \end{minipage} 
     \caption{Mixing and demixing models assumed in IVA, where $M=N=2$.}
     \label{fig:ivamodel}
  \end{figure}
%-%-%-%-%-%-%-%-%

%-%-%-%-%-%-%-%-%
%\begin{figure}[t]
%\centering
%\includegraphics[width=0.95\hsize]{./ch_conventional/fig/iva.pdf}
%\caption{Mixing and demixing models in IVA, where $M=N=2$.}
%\label{fig:ivamodel}
%\end{figure}
%-%-%-%-%-%-%-%-%

%-%-%-%-%-%-%-%-%
\begin{figure}[t]
\centering
\includegraphics[width=0.8\hsize]{./ch_conventional/fig/laplace_sav.pdf}
\caption{Zero-mean and spherically symmetric Laplace distribution, where $I=2$ and $s^*_{ijn}$ can be considered as either real or imaginary part of $s_{ijn}$.}
\label{fig:laplace}
\end{figure}
%-%-%-%-%-%-%-%-%
{本節では,音響信号処理分野で有名な周波数領域のBSSの一つであるIVAについて概説する.信号源,観測信号,及び分離信号のそれぞれについて,全ての周波数ビンに関する成分をまとめたベクトルを次のように定義する.}
\begin{align}
    \overline{\bm{s}}_{jn} &= [s_{1jn}, s_{2jn}, \cdots, s_{ijn}, \cdots, s_{Ijn} ]^{\mathrm{T}} \in \mathbb{C}^{I} \\
    \overline{\bm{x}}_{jm} &= [x_{1jm}, x_{2jm}, \cdots, x_{ijm}, \cdots, x_{Ijm} ]^{\mathrm{T}} \in \mathbb{C}^{I} \\
    \overline{\bm{y}}_{jn} &= [y_{1jn}, y_{2jn}, \cdots, y_{ijn}, \cdots, y_{Ijn} ]^{\mathrm{T}} \in \mathbb{C}^{I}
\end{align}
Fig. \ref{fig:ivamodel}に$M=N=2$の場合\textcolor{black}{におけるIVAの混合系及び分離系のモデル図を示す.}
IVAは時間周波数領域BSSであるため,全周波数の分離行列$( \bm{W}_i )_{i=1}^I$を推定する.
ただし,推定の過程{において,}全周波数を含む$I$次元{複素}分布を各信号源{$\overline{\bm{s}}_{jn}$の生成モデルと仮定している.さらに,この}$I$次元ベクトル内には高次相関(ベクトル内の要素が共起するという性質)があることを仮定している.
このIVAの生成モデルには,Fig. \ref{fig:laplace} で示すような\textcolor{black}{$I$次元球対称複素Laplace分布}が用いられ,式(\ref{eq:iva_model})で表される.
\begin{align}
  \nonumber p(\overline{\bm{s}}_{jn}) &= p(\overline{\bm{y}}_{jn}) \\
&= \frac{1}{\pi \prod_{i} \sigma_{in}} \exp \left(  - \sqrt{ \sum_i \left| \frac{y_{ijn}}{\sigma_{in}} \right|^2}  \right)
\label{eq:iva_model}    
\end{align}
ここで,$\sigma_{in}>0$はスケールパラメタである.
式(\ref{eq:iva_model})の分布は球対称性を持つため,同一ベクトル内の成分が高次相関を持つ\cite{Kim2007_iva}.
したがって,IVAは同時に生起する周波数成分を一つの\textcolor{black}{信号源}としてまとめる傾向がある.\textcolor{black}{この性質は,信号源が基本周波数成分とその整数倍の周波数成分からなる場合(調波構造を持つ場合)に,それらの成分を同一の信号源の成分とみなすことに対応する.そして,そのような信号源が出力されるように分離行列$( \bm{W}_i )_{i=1}^I$が推定される.即ち,もし各信号源が実際に周波数方向の共起性を持っていれば,IVAはパーミュテーション問題を回避しながら分離行列$( \bm{W}_i )_{i=1}^I$を推定することができ,BSSが達成される.本論文で扱う信号源においては,Fig.~\ref{fig:4chobsspect}に示されるように,心拍由来の成分や振動台で加振される成分などに顕著な調波構造(異なる周波数の共起性)がみられる.そのため,IVAが仮定する周波数間の高次相関は,本論文が対象とする観測信号や信号源に対しても有効であると考えられる.}

信号源周波数ベクトル間の独立性$p(\overline{\bm{y}}_{j1}, \overline{\bm{y}}_{j2}, \cdots, \overline{\bm{y}}_{jN}) = \prod_n p(\overline{\bm{y}}_{in})$を仮定すると,IVAの観測信号に対する負対数尤度関数は次式で得られる.
\begin{align}
    \mathcal{L} = -2J \sum_i \log |\det \bm{W}_i| + \sum_{j,n} G(\overline{\bm{y}}_{jn})
    \label{eq:ivalike}
\end{align}
ここで,$G(\overline{\bm{y}}_{jn})$はコントラスト関数と呼ばれ,次式で定義される.
\begin{align}
  \nonumber G(\overline{\bm{y}}_{jn}) &= -\log p(\overline{\bm{y}}_{jn}) \\
  \nonumber  &= -\log \frac{1}{\pi \prod_{i} \sigma_{in}} \exp \left(  - \sqrt{ \sum_i \left| \frac{y_{ijn}}{\sigma_{in}} \right|^2}  \right) \\
  &= \log \pi + \sum_i \log \sigma_{in} + \frac{1}{2} \log \sum_i \left| \frac{y_{ijn}}{\sigma_{in}}\right|^2
\end{align}
IVAにおける分離行列$( \bm{W}_i )_{i=1}^I$の推定は,最尤推定(式\eqref{eq:ivalike}の最小化)として定式化される.この最適化問題は, 補助関数法\cite{auxfunc}{及び反復射影法(iterative projection: IP)\cite{auxIVA}を用いた最適化アルゴリズムによって,高速かつ安定に解くことができる\cite{stable_auxIVA}.この反復最適化更新則は下記の通りである.}
\begin{align}
\bm{G}_{in} &= \frac{1}{J} \sum_j \frac{1}{\sqrt{\sum_{i} |\bm{w}_{in}^\mathrm{H}\bm{x}_{ij}|^{2}}} \bm{x}_{ij} \bm{x}_{ij}^{\mathrm{H}} \label{ep:auxIVAip1} \\
\bm{w}_{in} &\leftarrow (\bm{W}_i \bm{G}_{in})^{-1} \bm{e}_n \label{ep:auxIVAip2} \\
\bm{w}_{in} &\leftarrow \bm{w}_{in} ( \bm{w}_{in}^{\mathrm{H}} \bm{G}_{in} \bm{w}_{in} )^{-\frac{1}{2}} \label{ep:auxIVAip3}
\end{align}
ここで,$\bm{e}_{n} \in \mathbb{R}^{N}_{\{ 0, 1 \}^N}$は$n$番目の要素のみが1,他要素が0のベクトルである.
{式\eqref{ep:auxIVAip1}--\eqref{ep:auxIVAip3}を反復計算して分離行列$( \bm{W}_i )_{i=1}^I$を求めることができる.なお,この反復最適化アルゴリズムは1回の更新で負対数尤度関数\eqref{eq:ivalike}の値が減少する又は変動しないこと(単調非増加)が理論的に保証されている.}


%----------------------------------------------
\section{ISNMF}
\label{sec:conv:isnmf}
%----------------------------------------------

{本節では,次節で取り扱う周波数領域BSSであるILRMAを説明するために必要なNMF~\cite{NMF}の理論(特にISNMFの生成モデル\cite{isnmf})について概説する.}
NMFとは,{非負行列を低ランク近似する数理アルゴリズムの一つである.特に,少数の非負の基底ベクトルを教師なし学習で推定できるため,観測された非負行列に含まれる非負潜在パターンを抽出することが可能である.音響信号処理分野では,NMFに基づくBSSも数多く提案されている\cite{singlechsep, supNMF, MNMF_oz, iwase}.}

単一チャネルの{観測信号(時間波形)にSTFTを適用して得られる複素スペクトログラムを$\bm{B}\in\mathbb{C}^{I\times J}$とする.この$\bm{B}$に対するパワースペクトログラムを$|\bm{B}|^{.2}$と表記する.ここで,行列に対する$|\cdot|$及びドット付きの指数乗はそれぞれ要素毎の絶対値及び要素毎の指数乗を表す.パワースペクトログラムは非負行列であるため,NMFで次式のように低ランク近似することが可能である.}

\begin{align}
    |\bm{B}|^{.2} = \bm{TV} \label{eq:powerspect}
\end{align}
ここで,$\bm{T} \in \mathbb{R}^{I \times K}_{\geq 0}$を基底行列,$\bm{V} \in \mathbb{R}^{K \times J}_{\geq 0}$をアクティベーション行列という.
{また,$\bm{T}$の列数及び$\bm{V}$の行数である基底数$K$は事前に設定するパラメタであり,式(\ref{eq:powerspect})の右辺(モデル行列$\bm{TV}$)のランクに対応する.}通常は$K \ll \min (I,J)$となるように設定される.
{一例として,音響信号のパワースペクトログラムを$K=2$のNMFで低ランク近似した様子をFig. \ref{fig:nmf_ps}に示す.このような場合,パワースペクトログラム中に頻繁に出現するスペクトルパターンが2本の基底ベクトル(基底行列$\bm{T}$の列ベクトル)として抽出され,またそれらの時間的な強度変化がアクティベーションベクトル(アクティベーション行列$\bm{V}$の行ベクトル)に現れる.}

{特に,NMFのコスト関数にItakura--Saitoダイバージェンスを用いたISNMF\cite{isnmf}は,複素分布に基づく生成モデル\cite{GGD}を有しており,次節で説明する時間周波数領域BSSのILRMAに用いられている.ISNMFでは,}基底行列$\bm{T}$とアクティベーション行列$\bm{V}$は次式の最小化問題の解として推定される.
\begin{align}
    \min_{\bm{T,V}} \mathcal{D}(|\bm{B}|^{.2} | \bm{TV}) ~~\mathrm{s.t.} ~ t_{ik}, v_{kj} \geq 0 ~~\forall i,~j,~k
\end{align}
\begin{align}
    \mathcal{D}(|\bm{B}|^{.2}|\bm{TV}) = \sum_{i,j}\left( \frac{ |b_{ij}|^2 }{ \sum_{k}t_{ik}v_{kj} } - \log{\frac{ |b_{ij}|^2 }{ \sum_{k}t_{ik}v_{kj}}} -1 \right) \label{eq:matrixisdiv}
\end{align}
ここで,{$b_{ij}$,$t_{ik}$,及び$v_{kj}$はそれぞれ$\bm{B}$,$\bm{T}$,及び$\bm{V}$}の要素である.
また,{$k= 1, 2, \cdots, K$は基底ベクトルのインデクスを表す.さらに,式(\ref{eq:matrixisdiv})}は2つの行列($|\bm{B}|^{.2}$及び$\bm{TV}$)間の類似度を測る関数で\textcolor{black}{あり,Itakura--Saitoダイバージェンスと呼ばれる.}ISNMFの$\bm{T}$及び$\bm{V}$の最適化のための反復更新式は式(\ref{eq:MUT})及び式(\ref{eq:MUV})で表される\cite{MU}.

\begin{align}
    t_{ik} \leftarrow t_{ik} \sqrt \frac{ \sum_j |b_{ij}|^2 v_{kj} \left( \sum_{k'} t_{ik'} v_{k'j} \right)^{-2} }{ \sum_j v_{kj} \left( \sum_{k'} t_{ik'} v_{k'j} \right)^{-1} } \label{eq:MUT} \\
    v_{kj} \leftarrow v_{kj} \sqrt \frac{ \sum_i |b_{ij}|^2 t_{ik} \left( \sum_{k'} t_{ik'} v_{k'j} \right)^{-2} }{ \sum_i t_{ik} \left( \sum_{k'} t_{ik'} v_{k'j} \right)^{-1} } \label{eq:MUV}
\end{align}
この更新式は乗算型反復更新式と呼ばれ,目的関数が単調非増加\textcolor{black}{となる}ことが保証されている.

%-%-%-%-%-%-%-%-%
\begin{figure}[!t]
\centering
\includegraphics[width=0.95\hsize]{./ch_conventional/fig/NMF_ps.pdf}
\caption{NMF decomposition of power spectrogram of audio signal, where $K=2$.}
\label{fig:nmf_ps}
\end{figure}
%-%-%-%-%-%-%-%-%

%-%-%-%-%-%-%-%-%
\begin{figure}[!b]
\centering
\includegraphics[width=0.8\hsize]{./ch_conventional/fig/gauss_sav.pdf}
\caption{Circularly symmetric complex Gaussian distribution.}
\label{fig:gauss}
\end{figure}
%-%-%-%-%-%-%-%-%

\textcolor{black}{ISNMFは,複素スペクトログラムに対する生成モデルを仮定した最尤推定問題と等価である~\cite{isnmf}.次節で述べるBSSアルゴリズムのILRMAの説明には,ISNMFの生成モデルが必要であるため,本節で述べる.}
\textcolor{black}{複素スペクトログラムである$\bm{B}$の要素である$b_{ij}$が,次式のように複素部分成分$c_{ijk}$に分解されることを仮定する.}
\begin{align}
    b_{ij} = \sum_k c_{ijk} \label{eq:zijmodel}
\end{align}
\textcolor{black}{このとき,複素部分成分$c_{ijk}$に対して次式の生成モデルを仮定する.}
\begin{align}
    c_{ij,k} \sim \mathcal{N}_{\mathbb{C}} (c_{ijk}; 0, t_{ik}v_{kj})
\end{align}
ここで,$c_{ijk}\in \mathbb{C}$は全ての$i$,$j$,及び$k$に関して互いに独立と仮定する.
また,$c$を複素数の確率変数としたとき,\textcolor{black}{$\mathcal{N}_{\mathbb{C}} (c; \mu, \sigma^2)$は原点対称複素Gauss分布を表し,その確率密度関数は次式で与えられる.}
\begin{align}
    \mathcal{N}_{\mathbb{C}} (c; \mu, \sigma^2) &= \frac{1}{\pi \sigma^2} \exp \left( -\frac{|c-\mu|^2}{\sigma^2} \right) \label{eq:compGausiandist}
\end{align}
ここで,$\mu$及び$\sigma^2$はそれぞれ平均及び分散を示す.\textcolor{black}{この分布の外形をFig. \ref{fig:gauss}に示す.式(\ref{eq:compGausiandist})は再生性を持つ分布であるため,式(\ref{eq:zijmodel})で構成される$b_{ij}$もまた次式の生成モデルを持つことになる.}
\begin{align}
    b_{ij} \sim \mathcal{N}_{\mathbb{C}}\left( b_{ij}; 0, \sum_k t_{ik} v_{kj} \right) \label{eq:nmfgen}
\end{align}
\textcolor{black}{ISNMFでは},観測信号$b_{ij}$が与えられた場合における$t_{ik}$及び$v_{kj}$の最尤推定問題を考える.
このとき,尤度関数は
\begin{align}
    \mathcal{L}(\bm{T}, \bm{V}) = \prod_{i,j} \frac{1}{\pi \sum_k t_{ik} v_{kj} } \exp \left( -\frac{|b_{ij}|^2}{\sum_k t_{ik} v_{kj}} \right)
\end{align}
となり,負対数尤度\textcolor{black}{関数}は
\begin{align}
    -\log\mathcal{L}(\bm{T}, \bm{V}) = \sum_{i,j} \left( \frac{|b_{ij}|^2}{\sum_k t_{ik} v_{kj}} + \log \sum_k t_{ik} v_{kj} + \log \pi \right) \label{eq:negativeloglikelihood}
\end{align}
で表される.
\textcolor{black}{式(\ref{eq:negativeloglikelihood})は,パワースペクトログラム$|\bm{B}_{ij}|^{.2}$に対するISNMFの目的関数(\ref{eq:matrixisdiv})と定数部分の違いを除いて一致する.そのため,負対数尤度関数は以下のように書き換えられる.}
\begin{align}
    -\log\mathcal{L}(\bm{T}, \bm{V}) = \mathcal{D} \left( |\bm{B}|^{.2} | \bm{TV} \right) + \mathrm{const.}
    \label{eq:isnmf}
\end{align}
\textcolor{black}{この事実は,パワースペクトログラム$|\bm{B}|^{.2}$にISNMFを適用することが,式(\ref{eq:nmfgen})の生成モデルを仮定した分散の最尤推定問題と等価であることを示している\cite{GGD}.}

%----------------------------------------------
\section{ILRMA}
\label{sec:conv:ilrma}
%----------------------------------------------

\textcolor{black}{IVAよりも高精度なBSSを達成できるアルゴリズムとしてILRMA~\cite{ILRMA, Kitamura2018_ilrma}が提案されている.本論文においても,ILRMAに基づく心拍信号のBSSを検討するため,本節ではILRMAの生成モデル及び反復更新式について概説する.}

%-%-%-%-%-%-%-%-%
\begin{figure}[!t]
\centering
\includegraphics[width=0.95\hsize]{./ch_conventional/fig/ilrmaoutline.pdf}
\caption{Parameter updates of ILRMA.}
\label{fig:ilrma_outline}
\end{figure}
%-%-%-%-%-%-%-%-%

ILRMAの反復最適化の概要をFig. \ref{fig:ilrma_outline}に示す.
{図中の}$\bm{T}_n\in\mathbb{R}_{\geq 0}^{I\times K}$及び$\bm{V}_n\in\mathbb{R}_{\geq 0}^{K\times J}$は,$n$番目の信号源のパワースペクトログラム$|\bm{Y}_n|^{.2}$をISNMFで低ランク近似したモデル$\bm{T}_n\bm{V}_n$の基底行列及びアクティベーション行列である.
ILRMAは,IVAに\textcolor{black}{基づく}分離行列$\bm{W}_i$の反復最適化とISNMFによる分離信号源$( |\bm{Y}_n|^{.2} )_{n=1}^N$の低ランクモデリングに対応する$( \bm{T}_n\bm{V}_n )_{n=1}^N$の反復最適化が交互に行われる.
具体的には,分離行列$( \bm{W}_i )_{i=1}^I$により推定された分離信号$( |\bm{Y}_n|^{.2} )_{n=1}^N$をISNMFで非負低ランク行列としてモデル化し,得られた$( \bm{T}_n )_{n=1}^N$及び$( \bm{V}_n )_{n=1}^N$の各時間周波数成分を式(\ref{eq:nmfgen})における分散(各音源の生成モデルの推定パラメタ)に用いて分離行列$( \bm{W}_i )_{i=1}^I$を再度推定する,というプロセスが反復的に行われる.

ILRMAの生成モデルはISNMFと同様に次式の複素ガウス分布\textcolor{black}{が仮定されている.}
\begin{align}
    y_{ijn} &= \sum_k c_{ijnk} \\
    c_{ijnk} &= \mathcal{N}_{\mathbb{C}}(c_{ijnk}; 0, t_{ikn} v_{kjn}) \label{eq:ilrma_gen}
\end{align}
\textcolor{black}{ここで,$t_{ikn}$及び$v_{kjn}$は$\bm{T}_n$及び$\bm{V}_n$の要素である.}
また,$c_{ijnk} \in \mathbb{C}$は\textcolor{black}{$i, j, k, $及び$n$に関して}互いに独立であると仮定する.
このとき,観測\textcolor{black}{信号$(\bm{X}_n)_{n=1}^N$}が与えられた場合において, $\bm{W}_i$,$\bm{T}_n$, 及び$\bm{V}_n$を最尤推定する問題を考える.
ISNMFの\textcolor{black}{生成モデルより,}
\begin{align}
    y_{ijn} \sim \mathcal{N}_{\mathbb{C}}\left(y_{ijn};  0, \sum_k t_{ikn} v_{kjn} \right) 
  \label{eq:ISNMFmodel}
\end{align}
が成り立つので,\textcolor{black}{全分離信号の結合分布は次式で表される.}
\textcolor{black}{
\begin{align}
    \nonumber p(\bm{Y}_1, \bm{Y}_2, \cdots, \bm{Y}_N) &= \prod_n p(\bm{Y}_n) \\
\nonumber &= \prod_{n, i, j} p(y_{ijn}) \\
\nonumber &= \prod_{n, i, j} \mathcal{N}_{\mathbb{C}} \left(y_{ijn}; 0, \sum_k t_{ikn}v_{kjn} \right) \\
&= \prod_{n, i, j} \frac{1}{\pi \sum_k t_{ikn}v_{kjn}} \exp \left( -\frac{|y_{ijn}|^2}{\sum_k t_{ikn}v_{kjn}} \right) \label{eq:combineddist}
\end{align}}
\textcolor{black}{式(\ref{eq:combineddist})を用いて観測信号の負対数尤度関数を求めると次式となる.}
\textcolor{black}{
\begin{align}
  \nonumber \mathcal{L}(\mathsf{W, T, V}) &= - \log p(\bm{X}_1, \bm{X}_2, \cdots, \bm{X}_M) \\
    \nonumber &= -\log \left[p(\bm{Y}_1, \bm{Y}_2, \cdots, \bm{Y}_N) \cdot \prod_{i, j} |\det \bm{W}_i|^2\right] \\
    \nonumber &= -\log \left \{ \left[\prod_n p(\bm{Y}_n) \right] \cdot \prod_{i, j} |\det \bm{W}_i|^2\right \} \\
    \nonumber &= \mathrm{const.}- 2J \sum_{i}  \log |\det \bm{W}_i| +\sum_{i,j,n} \left( \frac{|y_{ij}|^2}{\sum_k t_{ikn} v_{kjn}} + \log \sum_k t_{ikn} v_{kjn}\right) \\
    &= \mathrm{const.}-2J \sum_i \log | \det \bm{W}_i | + \sum_{i,j,n} \left( \frac{|\bm{w}_{in}^{\mathrm{H}}\bm{x}_{ij}|^2}{\sum_k t_{ikn}v_{kjn}} + \log \sum_k t_{ikn}v_{kjn} \right)
    \label{eq:ilrmalike2}
\end{align}}
ここで,$\mathsf{W}=\{ \bm{W}_i \}_{i=1}^I$,$\mathsf{T}=\{ \bm{T}_n \}_{n=1}^N$, 及び$\mathsf{V}=\{ \bm{V}_n \}_{n=1}^N$は最適化パラメタの集合である.

式\eqref{eq:ilrmalike2}を最小化するパラメタは,以下に示す反復最適化アルゴリズムで推定される.まず,式(\ref{eq:ilrmalike2})を見ると,第二項と第三項は式\eqref{eq:negativeloglikelihood}のISNMFの尤度関数と等価であることがわかる.従って,式\eqref{eq:ilrmalike2}を最小化する$\mathsf{T}$及び$\mathsf{V}$はISNMFと同様に,次式の更新式で反復最適化できる.
\begin{align}
    t_{ikn} \leftarrow t_{ikn} \sqrt \frac{ \sum_j |\bm{w}_{in}^{\mathrm{H}}\bm{x}_{ij}|^2 v_{kjn} \left( \sum_{k'} t_{ik'n} v_{k'jn} \right)^{-2} }{ \sum_j v_{kjn} \left( \sum_{k'} t_{ik'n} v_{k'jn} \right)^{-1} } \label{eq:MUTilrma} \\
    v_{kjn} \leftarrow v_{kjn} \sqrt \frac{ \sum_i |\bm{w}_{in}^{\mathrm{H}}\bm{x}_{ij}|^2 t_{ikn} \left( \sum_{k'} t_{ik'n} v_{k'jn} \right)^{-2} }{ \sum_i t_{ikn} \left( \sum_{k'} t_{ik'n} v_{k'jn} \right)^{-1} } \label{eq:MUVilrma}
\end{align}
一方,分離行列$\bm{W}_i$に関する最適化は,分離ベクトル$\bm{w}_{in}$をIPで更新することで達成される.
\begin{align}
\bm{U}_{in} &= \frac{1}{J} \sum_j \frac{1}{\sum_{k}t_{ikn}v_{kjn}} \bm{x}_{ij} \bm{x}_{ij}^{\mathrm{H}} \label{eq:ip1} \\
\bm{w}_{in} &\leftarrow (\bm{W}_i \bm{U}_{in})^{-1} \bm{e}_n \label{eq:ip2} \\
\bm{w}_{in} &\leftarrow \bm{w}_{in} ( \bm{w}_{in}^{\mathrm{H}} \bm{U}_{in} \bm{w}_{in} )^{-\frac{1}{2}} \label{eq:ip3}
\end{align}
\textcolor{black}{これらの更新式も, IVAやISNMFと同様に,目的関数\eqref{eq:ilrmalike2}の値が単調非増加となることが保証されている.}

\textcolor{black}{上記のILRMAは基底数$K$が信号源毎に同じ数を設定した場合の定式化であり,以後,基底数固定型ILRMAと呼ぶ.一方で,信号源毎に適切な基底数は未知であるため,全信号源に用いる基底数を$r=1, 2, \cdots, R$と設定したうえで,適応的に各信号源に基底数を割り振る定式化も提案されている.以後,この手法を基底数可変型ILRMAと呼ぶ.以下に,基底数可変型ILRMAの定式化及び反復更新式を記載する.}

\textcolor{black}{基底数可変型ILRMAでは,式\eqref{eq:ISNMFmodel}の生成モデルにおける分散$\sigma_{ijn}^2$を次式で定式化している.}
\begin{align}
\sigma_{ijn}^2 = \sum_{r} z_{nr}t_{ir}v_{ri}
\end{align}ここで,$z_{nr}\in [0, 1]$は分割関数と呼ばれ,$r$番目の基底を$n$番目の音源に分配する潜在変数である.従って,$\sum_{n} z_{nr} = 1$という制約がある.この分割関数を含めた基底数可変型ILRMAの更新式は次式となる.
\begin{align}
t_{ir}&\leftarrow t_{ir}\sqrt{\dfrac{\sum _{j,n}\left| b_{ijn}\right| ^{2}z_{nr}v_{rj}\left( \sum _{r'}z_{nr'}t_{ir'}v_{r'j}\right) ^{-2}}{\sum _{j,n}z_{nr}v_{rj}\left( \sum _{r'}z_{nr'}t_{ir'}v_{r'j}\right) ^{-1}}} \label{eq:ilrma2ip1} \\
v_{ir}&\leftarrow v_{ir}\sqrt{\dfrac{\sum _{i,n}\left| b_{ijn}\right| ^{2}z_{nr}t_{ir}\left( \sum _{r'}z_{nr'}t_{ir'}v_{r'j}\right) ^{-2}}{\sum _{i,n}z_{nr}t_{ir}\left( \sum _{r'}z_{nr'}t_{ir'}v_{r'j}\right) ^{-1}}} \label{eq:ilrma2ip2} \\
z_{nr}&\leftarrow z_{nr}\sqrt{\dfrac{\sum _{i,j}\left| b_{ijn}\right| ^{2}t_{ir}v_{rj}\left( \sum _{r'}z_{nr'}t_{ir'}v_{r'j}\right) ^{-2}}{\sum _{i,j}t_{ir}v_{rj}\left( \sum _{r'}z_{nr'}t_{ir'}v_{r'j}\right) ^{-1}}} \label{eq:ilrma2ip3} \\
z_{nr}&\leftarrow \dfrac{z_{nr}}{\sum _{n'}z_{nr}} \label{eq:normalizez}
\end{align}

式\eqref{eq:ilrma2ip1}--\eqref{eq:ilrma2ip3}で音源モデルを更新した後は,基底数固定型ILRMAと同様に式\eqref{eq:ip1}--\eqref{eq:ip3}で分離行列を更新する.この手続きを反復する.また,式\eqref{eq:normalizez}は前述した通り,$\sum_{n} z_{nr} = 1$を保証するために計算される.

\textcolor{black}{本論文では,本節で説明したILRMAに関しては基底数固定型及び基底数可変型の2つのアルゴリズムを適用する.但し,実験の章で述べる通り,基底数の固定・可変に対して顕著な差は現れないため,次節で述べる$t$-ILRMAにおいては,基底数固定型のみ適用する.}

%----------------------------------------------
\section{$t$-ILRMA}
\label{sec:conv:tilrma}
%----------------------------------------------

\textcolor{black}{近年,ISNMFの複素Gauss分布に基づくに基づく生成モデルを一般化したNMFが数多く提案されている.例えば,複素一般化Gauss分布に基づくNMF~\cite{GGD}や複素Student's $t$分布に基づくNMF($t$-NMF)\cite{tnmf}等が提案されており,NMFに基づく特定のタスク(BSSやスパースノイズ除去等)においてISNMFよりも性能が向上することが報告されている.このことに鑑み,}複素Student's~$t$分布を生成モデルとして仮定するILRMA($t$-ILRMA)が提案された\cite{tdist1, tdist2}.
複素Student's~$t$分布には形状パラメタ$\nu~\textgreater~0$がある.$\nu=1$とすると複素Cauchy分布に一致し,$\nu\rightarrow \infty$とすると複素Gauss分布に一致する.

$t$-ILRMAの生成モデルは次式で表される.
\begin{align}
     \prod_{i,j}\left(\bm{Y}_{n}\right)= \prod_{i,j}\frac{1}{\pi \sigma_{ijn}^{2} } \left( 1+\frac{2}{\nu}\frac{|y_{ijn}|^2}{\sigma_{ijn}^{2}} \right)^{\frac{2+\nu}{2}} \label{eq:tmodel}
\end{align}
\begin{align}
     \sigma_{ijn}^{p}=\left(\sum_{k}t_{ikn}v_{kjn}\right)^{p} \label{eq:tsigma}
\end{align}
ここで,$\sigma_{ijn}>0$は時間周波数ごとに変動する非負パラメタであり,分離信号$\bm{Y}_{n}$の振幅スペクトログラム$|\bm{Y}_{n}|$のNMFモデルである.また,$p$はNMFモデルの信号領域を決定するドメインパラメタであり,$p=1$のときは振幅ドメイン,$p=2$のときはパワードメインをモデル化することに相当する.従って,$p=2$かつ$\nu\rightarrow\infty$のときは,式\eqref{eq:tmodel}はISNMF及びILRMAの生成モデルに一致する.この生成モデルの負対数尤度関数が$t$-ILRMAの目的関数であり,次式で与えられる.
\textcolor{black}{
\begin{align}
    \nonumber \mathcal{L}_{t}(\mathsf{W, T, V}) &= \mathrm{const.}-2J \sum_{i}\log|\mathrm{det}\bm{W}_{i}| \\
&+ \sum_{i,j}\left\lbrack\left(1+\frac{\nu}{2}\right)\log\left(1+\frac{2}{\nu}\frac{|\bm{w}_{in}^{\mathrm{H}}\bm{x}_{ij}|^2}{\left(\sum_{k}t_{ikn}v_{kjn}\right)^2}\right)+2\log\left(\sum_{k}t_{ikn}v_{kjn}\right)\right\rbrack \label{eq:tcost}
\end{align}}

式\eqref{eq:tcost}を最小化するNMFの最適化パラメタ$\mathsf{T}$及び$\mathsf{V}$の更新式は次式で表される.
\begin{align}
    t_{ikn}&\leftarrow t_{ikn}\left\lbrack\frac{\sum_{j}\frac{|\bm{w}_{in}^{\mathrm{H}}\bm{x}_{ij}|^2}{\frac{\nu}{\nu+2}\left(\sum_{k}t_{ikn}v_{kjn}\right)^{2}+\frac{2}{\nu+2}|\bm{w}_{in}^{\mathrm{H}}\bm{x}_{ij}|^2}\left(\sum_{k}t_{ikn}v_{kjn}\right)^{-p}v_{kjn}^{-\frac{2}{p}}}{\sum_{i}\left(\sum_{k}t_{ikn}v_{kjn}\right)^{-p}v_{kjn}}\right\rbrack^{\frac{p}{p+2}} \label{eq:tt} \\
    v_{lj,n}&\leftarrow v_{lj,n}\left\lbrack\frac{\sum_{i}\frac{|\bm{w}_{in}^{\mathrm{H}}\bm{x}_{ij}|^2}{\frac{\nu}{\nu+2}\left(\sum_{k}t_{ikn}v_{kjn}\right)^{2}+\frac{2}{\nu+2}|\bm{w}_{in}^{\mathrm{H}}\bm{x}_{ij}|^2}\left(\sum_{k}t_{ikn}v_{kjn}\right)^{-p}t_{iln}^{-\frac{2}{p}}}{\sum_{i}\left(\sum_{k}t_{ikn}v_{kjn}\right)^{-p}t_{iln}}\right\rbrack^{\frac{p}{p+2}} \label{eq:tv}
\end{align}

一方,分離行列$\mathsf{W}$は節\ref{sec:conv:ilrma}と同様に分離ベクトル$\bm{w}_{in}$をIPで更新することで達成される.
\begin{align}
    \bm{U}_{in}&=\frac{1}{J}\left(\frac{2}{\nu}+1\right)\sum_{j}\frac{1}{\left\lbrack1+\frac{2}{\nu}\frac{|\bm{w}_{in}^{\mathrm{H}}\bm{x}_{ij}|^2}{\left(\sum_{k}t_{ikn}v_{kjn}\right)^{2}}\right\rbrack\left(\sum_{k}t_{ikn}v_{kjn}\right)^{2}}\bm{x}_{ij}\bm{x}_{ij}^{\mathrm{H}} \label{eq:tip1} \\
    \bm{w}_{in}&\leftarrow\left(\bm{W}_{i}\bm{U}_{in}\right)^{-1}\bm{e}_{n} \label{eq:tip2} \\
    \bm{w}_{in}&\leftarrow\frac{\bm{w}_{in}}{\sqrt{\bm{w}_{in}^{\mathrm{H}}\bm{U}_{in}\bm{w}_{in}}} \label{eq:tip3}
\end{align}
\textcolor{black}{ここで,$\alpha_{ijn}>0$は補助変数である.}
これらの更新式も,\textcolor{black}{IVA, ISNMF, 及びILRMAと同様に,目的関数}(\ref{eq:tcost})の値が単調非増加となることが保証されている.


%----------------------------------------------
\section{心拍推定アルゴリズム}
\label{sec:conv:heartrateestalgo}
%----------------------------------------------

本節では,レーダセンサの観測信号やBSSで推定された心拍信号等に対して適用する心拍推定アルゴリズムについて説明する.なお,本節ではBSSにより得られる推定心拍の時間領域の分離信号を処理の対象としている.この分離信号を$( \tilde{y}_\mathrm{HR}[l] )_{l=1}^L$と表記する.本論文で使用する心拍推定アルゴリズムを下記に順を追って示す.また,心拍推定アルゴリズムの処理の流れを以下にFig.~\ref{fig:hralgo}として示す.

%-%-%-%-%-%-%-%-%
\begin{figure}[tb]
\centering
\includegraphics[width=1.0\hsize]{./ch_conventional/fig/hralgo.pdf}
\caption{Process of heart rate estimation algorithm.}
\label{fig:hralgo}
\end{figure}
%-%-%-%-%-%-%-%-%

\textcolor{black}{まず,信号の調波構造を強調しつつ低周波成分を除去するため,次式の二階微分フィルタを信号に適用する.}
\begin{equation} 
    \begin{cases}
        \begin{split}
        & \dot{y}_\mathrm{HR}[l]=\tilde{y}_\mathrm{HR}[l+1]-\tilde{y}_\mathrm{HR}[l] \\
        & \ddot{y}_\mathrm{HR}[l]=\dot{y}_\mathrm{HR}[l+1]-\dot{y}_\mathrm{HR}[l]
        \end{split}                         &   \forall{l}
    \end{cases}
\end{equation}
次に,もう一度調波構造を強調する目的で,信号に半波整流を適用する.
\begin{equation} 
\hat{y}_{\mathrm{HR}}[l] =
    \begin{cases}
        \begin{split}
        & \ddot{y}_\mathrm{HR}[l]\quad \mathrm{if}\quad \ddot{y}_\mathrm{HR}[l]\geqq0 \\
        & 0\hspace{25pt} \mathrm{otherwise}
        \end{split}                         &   \forall{l}
    \end{cases}
\end{equation}
\textcolor{black}{さらに,心拍信号が多く存在する帯域である0.7--1.4~Hzを通過域とする5次楕円IIRバンドパスディジタルフィルタを適用する.この処理を次式で表す.}
\begin{align}
    \left(\breve{y}_{\mathrm{HR}}[l]\right)_{l=1}^{L}=\mathrm{BPF}\left[\left(\hat{y}_{\mathrm{HR}}[l]\right)_{l=1}^{L}\right]
\end{align} \label{eq:bpf}
最後に,$( \breve{y}_{\mathrm{HR}}[l] )_{l=1}^L$にSTFTを適用してスペクトログラムに変換し,振幅スペクトログラムの最大ピークとなる周波数を時間フレーム毎に求めることで,推定心拍を得る.

%----------------------------------------------
\section{本章のまとめ}
\label{sec:conv:conclusion3}
%----------------------------------------------
\textcolor{black}{本章では,前章で得られた観測信号に対してBSSを適用する動機について述べた.振動測定系と体表面の相対的な位置関係が未知であることと,BSSは音響信号処理分野で高度に発展していることから,本論文の目的に対してBSSを適用しその効果を検証することは重要と考えられる.また,時間周波数領域BSSの定式化を行い,その代表的な手法であるIVAについて説明した.行列分解の手法の一つであるISNMFについて説明し,ILRMAの生成モデル及び反復更新式について概説した.さらに,複素Student's $t$分布を生成モデルに持つようなILRMAである$t$-ILRMAについて説明した.振動測定系から得られる観測信号に前述した手法を適用して得られる分離信号に対して適用する心拍推定アルゴリズムについても説明した.次章では予備実験として,\ref{chap:measurementstructsig}章の振動測定系から得られる観測信号に,\ref{sec:conv:iva}節で説明したIVAを適用し,実験結果の解説を行う.}

\chapter{観測信号にフィルタを適用しない場合のBSS及び心拍推定実験}
\label{chap:bsshrexp}

%----------------------------------------------
\section{まえがき}
%------------------------------------------
本章では,\ref{chap:measurementstructsig}章の振動測定系で観測された4チャネルのレーダアレイの観測信号に対して,\ref{sec:conv:iva}節で述べたIVAを適用した結果を示し,心拍推定アルゴリズムの精度向上について実験的に検討する.但し,本節では,レーダセンサから得られる信号に前処理を加えず,そのままIVAを適用した結果を予備実験として示している.より詳細な実験については次章にまとめて示すが,\ref{chap:bsshrexp}章と\ref{chap:fbsshrexp}章の違いは,呼吸の体動に由来するノイズを除去するためのハイパスフィルタを前処理として観測信号に適用するか否かである.
なお,心拍推定の精度を確認するために必要な心拍の参考値については,\ref{chap:measurementstructsig}章で述べた接触型ECGセンサの観測信号を心拍推定アルゴリズムに適用した推定値を用いている.

まず\ref{sec:conv:expcondition4}節では,IVAを適用させる際のパラメタについて述べる.
\ref{sec:conv:expresult4}節では,IVAを適用させて得られたスペクトログラムと心拍推定グラフについての解説を述べる.\ref{sec:conv:conclusion4}節で本章をまとめる.

%----------------------------------------------
\section{本実験の動機}
\label{sec:conv:expmotivation}
%----------------------------------------------
まず,本章では\ref{chap:measurementstructsig}章で述べた振動測定系から得られる時間波形として,背部及び臀部のData no. 0からData no. 6を使用する.本論文で取り扱う振動測定系の混合系は未知であるため,最適なSTFTの窓長を定めることができない.STFTの窓長はBSSの精度に強く影響するパラメタである\cite{winlengthBSS}ため,本実験における最適な窓長は実験的に調査する必要がある.
本論文で取り扱う観測信号は音響信号ではないが,依然としてSTFTの窓長がBSSの性能に与える影響は大きいと予想される.そのため,本章では予備実験として4パターンのSTFTの窓長を試し,それぞれのBSSの結果及び心拍推定アルゴリズムを適用した結果の精度について調査する.

%----------------------------------------------
\section{実験条件}
\label{sec:conv:expcondition4}
%----------------------------------------------
\textcolor{black}{前節で述べた通り,本章の実験ではSTFTの窓長を32点(0.8~s),64点(1.6~s),96点(2.4~s), 及び128点(3.2~s), シフト長をそれぞれ,2点(0.05~s),4点(0.1~s),6点(0.15~s), 及び8点(0.2~s)に設定した4条件のSTFTを適用する.分離行列$\bm{W}_{i}$の初期値は全ての周波数に対して単位行列とした.また,IVAの式\eqref{ep:auxIVAip1}--\eqref{ep:auxIVAip3}に示した反復更新式の反復回数は100回に設定した.100回の反復回数でコスト関数値が十分に収束することを実験的に確認したため,全ての実験において100回で統一した.}

%----------------------------------------------
\section{実験結果}
\label{sec:conv:expresult4}
%----------------------------------------------

\textcolor{black}{本節以降では,\ref{chap:measurementstructsig}章のTable~\ref{tab:vibevacondition}で示した振動条件(Data no.~0からData no.~6)と2種類の測定部位(背部又は臀部)を組み合わせて実験データを参照する.例えば,Data no.~0の背部のレーダセンサから得られる観測信号を「Back no.~0」,Data no.~2の臀部のレーダセンサから得られる観測信号を「Bottom no.~2」等と呼称する.また,本節以降に多くのスペクトログラムを掲載するが,これらの図は見やすさのために窓長を1024点(25.6~s),シフト長を32点(0.8~s)に設定したSTFTで計算されるものに統一されている.前節の条件で述べたように,IVAやILRMA等の時間周波数領域BSSで処理の対象となる信号($\bm{X}_m$等)とは,窓長及びシフト長が異なる点に注意されたい.}

Fig. \ref{fig:siva32obs}は観測信号Back no.~2のスペクトログラムである.Fig.~\ref{fig:ecgspect}に\ref{chap:measurementstructsig}章で述べた接触型ECGセンサから得られる心拍信号のスペクトログラムを示す.従って,BSSで得られた分離信号の中の心拍信号がFig.~\ref{fig:ecgspect}にどの程度近いかが重要となる.Figs. \ref{fig:siva32est}--\ref{fig:siva128est}は,Back no.~2の観測信号に対して,前節で述べた4種類の窓長及びシフト長でSTFTしIVAを適用することで得られる分離信号のスペクトログラムである.観測信号は4チャネルであるため,BSSによる分離信号も4つ得られるが,理想的にはこの内のいずれかが心拍信号に対応することを想定している.本章以降で示すBSSの結果において,心拍が最も強く含まれる信号は全て3番目の分離信号となっている.BSSは分離信号の順序が任意であるため,何番目の分離信号に心拍信号が現れるかは最適化変数(分離行列$\bm{W}_i$及びNMF変数$\bm{T}_n\bm{V}_n$)の初期値に依存する,本論文の実験において心拍の分離信号が常に3番目の分離信号に現れた現象は,実験条件として分離行列$\bm{W}_i$の初期値をどの手法においても単位行列と設定したことに起因すると思われる.なお,4つの分離信号から心拍信号を選択する方法については今後の課題とし,本論文では取り扱わない.これらの結果を見るとFig. \ref{fig:siva32est}では,3番目のスペクトログラムに強く心拍の高調波成分が3.5~Hz,5~Hz,及び6~Hz付近に見られる.一方で,0.4~Hz付近に呼吸による体動に起因するノイズが強く残留しており,心拍成分のみを完全に分離できていない.また,振動の高調波成分はほとんど他のスペクトログラムに分離されているが,1.2~Hzの基本周波数成分がよく残留している.1番目, 2番目, 及び4番目のスペクトログラムを見ると,3番目のスペクトログラムでは見られなかった成分が分離されているが,どれも呼吸又は振動台由来のノイズとなっている.Fig.~\ref{fig:siva64est}についても同様のことがいえる.呼吸の成分は4つの分離信号全てに残留してしまうことが多い.依然として3番目のスペクトログラムに心拍由来の調波構造がよく分離されていることが分かる.Figs. \ref{fig:siva96est}及び\ref{fig:siva128est}についても概ね同様の傾向であるが,Figs. \ref{fig:siva32est}--\ref{fig:siva128est}の3番目のスペクトログラムのみに注目して窓長の違いを比較してみると,窓長が長くなるほど振動台由来の3.5~Hz付近の3次高調波成分が強く現れてくる.Fig. \ref{fig:siva32est}では,ノイズの高調波成分は分離されているが,呼吸成分も同時に僅かに分離されているのに対してFig. \ref{fig:siva64est}では心拍成分があまり分離されずに残留している.これらより,最もバランスよく心拍成分が分離されている窓長は64点であるといえる.

次に,窓長を64点に設定したIVAによって分離された4つの分離信号それぞれに心拍推定アルゴリズムを適用した結果をFig.~\ref{fig:hriva}に示す.赤色のグラフは分離信号に心拍推定アルゴリズムを適用した結果を示し,青色のグラフは接触型ECGセンサから得られる心拍信号に心拍推定アルゴリズムを適用した推定心拍値(参照値)を示している.そのため,参照値のグラフは4つの心拍推定グラフ全てにおいて共通である.Fig.~\ref{fig:hriva}を見ると,1番目及び4番目の心拍推定グラフに振動台由来の振動成分が現れていることが分かる.特に4番目の心拍推定グラフでは振動が加えられている60~sから360~sまでの間の振動成分が顕著に表れている.一方で,3番目の心拍推定グラフでは,分離された心拍信号が参照値と概ね合致していることが分かる.さらに,推定心拍の窓長に関する違いをFig.~\ref{fig:hrivawinlen}に示す.但し,すべての窓長において3番目の分離信号に心拍推定アルゴリズムを適用した結果を掲載している.4つの心拍推定グラフを比較すると,窓長が96点および128点の心拍推定信号は前述した通り,振動の3次高調波成分現れてくることから参照値とあまり合致していない.窓長が32点及び64点の心拍推定信号はどちらも概ね参照値と合致している.この2つの窓長における違いはどちらが良いか判断が難しいが,背部及び臀部の他の観測信号についても同様の比較を行った結果,窓長が64点の場合安定して参照値に近い心拍推定値が得られたため,以後の実験ではSTFTの窓長を64点に設定して調査を行う.


%-%-%-%-%-%-%-%-%
\begin{figure}[tb]
\centering
\includegraphics[width=1.0\hsize]{./ch_conventional/fig/spect_iva_32_obs.pdf}
\caption{Spectrograms of observed signal of Back no.~2.}
\label{fig:siva32obs}
\end{figure}
%-%-%-%-%-%-%-%-%

%-%-%-%-%-%-%-%-%
\begin{figure}[tb]
\centering
\includegraphics[width=0.7\hsize]{./ch_conventional/fig/ecgspect.pdf}
\caption{Spectrogram Spectrogram of observed signal obtained by contact-type ECG sensor for Data no.~2.}
\label{fig:ecgspect}
\end{figure}
%-%-%-%-%-%-%-%-%

%-%-%-%-%-%-%-%-%
\begin{figure}[tb]
\centering
\includegraphics[width=1.0\hsize]{./ch_conventional/fig/spect_iva_32_est.pdf}
\caption{Spectrograms of separated signals obtained by IVA (Back no.~2), where window and shift lengths were respectively set to 32 and 2 points.}
\label{fig:siva32est}
\end{figure}
%-%-%-%-%-%-%-%-%

%-%-%-%-%-%-%-%-%
\begin{figure}[tb]
\centering
\includegraphics[width=1.0\hsize]{./ch_conventional/fig/spect_iva_64_est.pdf}
\caption{Spectrograms of separated signals obtained by IVA (Back no.~2), where window and shift lengths were respectively set to 64 and 4 points.}
\label{fig:siva64est}
\end{figure}
%-%-%-%-%-%-%-%-%

%-%-%-%-%-%-%-%-%
\begin{figure}[tb]
\centering
\includegraphics[width=1.0\hsize]{./ch_conventional/fig/spect_iva_96_est.pdf}
\caption{Spectrograms of separated signals obtained by IVA (Back no.~2), where window and shift lengths were respectively set to 96 and 6 points.}
\label{fig:siva96est}
\end{figure}
%-%-%-%-%-%-%-%-%

%-%-%-%-%-%-%-%-%
\begin{figure}[tb]
\centering
\includegraphics[width=1.0\hsize]{./ch_conventional/fig/spect_iva_128_est.pdf}
\caption{Spectrograms of separated signals obtained by IVA (Back no.~2), where window and shift lengths were respectively set to 128 and 8 points.}
\label{fig:siva128est}
\end{figure}
%-%-%-%-%-%-%-%-%

%-%-%-%-%-%-%-%-%
\begin{figure}[htbp]
     \begin{minipage}[t]{0.45\hsize}
        \centering
        \includegraphics[keepaspectratio, width=6.5cm]{./ch_conventional/hrfig/hr_iva_64_ch1.pdf}
        \subcaption{First estimated signal}
        \label{fig:hriva32ch1}
      \end{minipage} 
      \begin{minipage}[t]{0.45\hsize}
        \centering
        \includegraphics[keepaspectratio, width=6.5cm]{./ch_conventional/hrfig/hr_iva_64_ch2.pdf}
        \subcaption{Second estimated signal}
        \label{fig:hriva32ch2}
      \end{minipage} \\
      \begin{minipage}[t]{0.45\hsize}
        \centering
        \includegraphics[keepaspectratio, width=6.5cm]{./ch_conventional/hrfig/hr_iva_64_ch3.pdf}
        \subcaption{Third estimated signal}
        \label{fig:hriva32ch3}
      \end{minipage} 
      \begin{minipage}[t]{0.45\hsize}
        \centering
        \includegraphics[keepaspectratio, width=6.5cm]{./ch_conventional/hrfig/hr_iva_64_ch4.pdf}
        \subcaption{Fourth estimated signal}
        \label{fig:hriva32ch4}
      \end{minipage} 
     \caption{Estimated (red) and reference (blue) heart rates obtained by IVA (Back no.~2) for each separated signal.}
     \label{fig:hriva}
  \end{figure}
%-%-%-%-%-%-%-%-%

%-%-%-%-%-%-%-%-%
\begin{figure}[htbp]
      \begin{minipage}[t]{0.45\hsize}
        \centering
        \includegraphics[keepaspectratio, width=6.5cm]{./ch_conventional/hrfig/hr_iva_32_ch3.pdf}
        \subcaption{Windw length: 32 points}
        \label{fig:hriva32ch3}
      \end{minipage} 
      \begin{minipage}[t]{0.45\hsize}
        \centering
        \includegraphics[keepaspectratio, width=6.5cm]{./ch_conventional/hrfig/hr_iva_64_ch3.pdf}
        \subcaption{Windw length: 64 points}
        \label{fig:hriva64ch3}
      \end{minipage} \\
   
      \begin{minipage}[t]{0.45\hsize}
        \centering
        \includegraphics[keepaspectratio, width=6.5cm]{./ch_conventional/hrfig/hr_iva_96_ch3.pdf}
        \subcaption{Windw length: 96 points}
        \label{fig:hriva96ch3}
      \end{minipage} 
      \begin{minipage}[t]{0.45\hsize}
        \centering
        \includegraphics[keepaspectratio, width=6.5cm]{./ch_conventional/hrfig/hr_iva_128_ch3.pdf}
        \subcaption{Windw length: 128 points}
        \label{fig:hriva128ch3}
      \end{minipage} 
     \caption{Estimated (red) and reference (blue) heart rates of separated heartbeat signals obtained by IVA (Back no.~2) with various window lengths: (a)~32 points, (b)~64 points, (c)~96, and (d)~128 points.}
     \label{fig:hrivawinlen}
  \end{figure}
%-%-%-%-%-%-%-%-%

%----------------------------------------------
\section{本章のまとめ}
\label{sec:conv:conclusion4}
%----------------------------------------------
本章では,\ref{chap:measurementstructsig}章で得られた観測信号に対してIVAを適用した.但し,STFTの窓長が分離精度に影響を与えるパラメタであるため,窓長とシフト長の組み合わせを4種類に設定して最適な窓長の調査を行った.また,接触型ECGセンサから得られる心拍信号のスペクトログラムと比較も行った.さらに,これらの分離信号に心拍推定アルゴリズムを適用し,心拍推定値と参照値との比較も行った.これらの実験の結果,STFTの最適な窓長は64点であることを確認した.しかしながら,呼吸による体動に起因するノイズは全ての分離信号に強く残留しているため,完全に心拍信号のみを分離することは困難であった.そのため次章では,前処理として観測信号から呼吸の基本周波数成分を落とすハイパスフィルタを適用してから各種BSSを適用する手法について検討する.


\chapter{観測信号にフィルタを適用した場合のBSS及び心拍推定実験}
\label{chap:fbsshrexp}

%----------------------------------------------
\section{まえがき}
%----------------------------------------------
\textcolor{black}{前章の実験結果より,観測信号に含まれる呼吸の基本周波数成分(Fig.~\ref{fig:siva32obs}における0.4~Hz付近の成分)は相対的に他の信号よりも強く,IVAではあまり除去できないことが確認された.この呼吸のノイズ信号はあまり明確な調波構造を持っておらず,基本周波数のみが顕著な周波数特性であることがFig.~\ref{fig:siva32obs}から確認される.そこで,本章では,時間周波数領域BSSを適用する前の処理として,呼吸基本周波数成分を除去するハイパスフィルタを適用することを提案し,この効果について実験的に調査する.特に本章では,前述の前処理の適用に加えて,IVA,基底数固定型ILRMA, 基底数可変型LRMA, 及び$t$-ILRMAの4手法を適用して比較することにより,どの手法が高精度な心拍推定を実現できるか検討する.
まず\ref{sec:conv:filteroutline}節では,前処理として新たに観測信号の各チャネルに適用するディジタルフィルタの設計仕様について述べる.\ref{sec:conv:expcond5}節では,IVA,基底数固定型ILRMA,基底数可変型ILRMA,及び$t$-ILRMAの4手法の実験条件について説明する.\ref{sec:conv:expresult5}節では,各項に手法毎の結果をまとめて考察し,本章の結論を\ref{sec:conv:conclusion5}節で述べる.}

%----------------------------------------------
\section{\textcolor{black}{前処理として適用するフィルタの設計}}
\label{sec:conv:filteroutline}
%----------------------------------------------
\textcolor{black}{本章の実験で適用するハイパスフィルタの詳細について説明する.観測信号のスペクトログラムであるFig.~\ref{fig:siva32obs}より,呼吸の基本周波数成分はおよそ0.4~Hz付近に存在することが確認できる.そこで,本論文では,前処理として,カットオフ周波数を1.5~Hzとするハイパスフィルタを観測信号の各チャネルに対して適用する.このハイパスフィルタは,位相歪みが生じない(線形位相特性を満たす)ようにFIRディジタルフィルタとして設計している.フィルタのタップ長(次数)は170次である.このフィルタの振幅応答と位相応答をFigs.~\ref{fig:ampres}及び\ref{fig:phaseres}にそれぞれ示す.次節では,このハイパスフィルタをチャネル毎に適用した信号を周波数領域BSSの観測信号$( \tilde{\bm{x}}[l] )_{l=1}^L$とみなし,心拍信号の分離を行う.}

%-%-%-%-%-%-%-%-%
\begin{figure}[!t]
\centering
\includegraphics[width=1.0\hsize]{./ch_conventional/fig/ampprop.pdf}
\caption{Amplitude response of high-pass digital filter used in preprocessing for observed signals.}
\label{fig:ampres}
\end{figure}
%-%-%-%-%-%-%-%-%

%-%-%-%-%-%-%-%-%
\begin{figure}[!t]
\centering
\includegraphics[width=1.0\hsize]{./ch_conventional/fig/phaseprop.pdf}
\caption{Phase response of high-pass digital filter used in preprocessing for observed signals.}
\label{fig:phaseres}
\end{figure}
%-%-%-%-%-%-%-%-%

%----------------------------------------------
\section{実験条件}
\label{sec:conv:expcond5}
%----------------------------------------------

前章の実験において,最もバランスよく心拍信号が分離されたSTFTの窓長及びシフト長はそれぞれ,64点(1.6~s)及び4点(0.1~s)であることが分かったため,これらをIVA,ILRMA(基底数固定型と基底数可変型を含む),及び$t$-ILRMAのSTFTの窓長及びシフト長として設定した.IVA,ILRMA,及び$t$-ILRMAのその他の実験条件を以下に示す.

まず,IVAの実験条件を示す.分離行列$\bm{W}_{i}$の初期値は全ての周波数ビンに対して単位行列とした.また,IVAの式\eqref{ep:auxIVAip1}--\eqref{ep:auxIVAip3}に示した反復更新式の反復回数は100回に設定した.

次に,ILRMAの実験条件(基底数固定型と基底数可変型を含む)を示す.\ref{sec:conv:ilrma}節で示した式\eqref{eq:ip1}--\eqref{eq:ip3}の反復更新式の反復回数は100回に設定した.IVAと同様に100回の反復更新でコスト関数値が十分に収束することを実験的に確認したため,全ての実験において100回に統一した.また基底数固定型ILRMAの基底数を信号源毎に$K=3$とした.基底数可変型ILRMAでは全信号源の基底数を$R=12$とした.基底数固定型ILRMAは反復毎にプロジェクションバックで正規化を行い,基底数可変型ILRMAは反復毎にプロジェクションバックで正規化を行わない.基底行列及びアクティベーション行列の初期値は,基底数固定型及び基底数可変型のいずれにおいても,区間$(0,1)$の一様乱数に設定した.その他の実験条件はIVAと同様である.

最後に,$t$-ILRMAの実験条件を示す.複素Studen's~$t$分布の形状パラメタ$\nu$は1, 2, 及び5に設定し,NMFの信号領域を決定するドメインパラメタ$p$は1(振幅ドメイン)及び2(パワードメイン)と設定した.また,\ref{sec:conv:tilrma}節で示した式\eqref{eq:tip1}--\eqref{eq:tip3}の反復更新式の反復回数は100回に設定した.IVA及びILRMAと同様に100回の反復更新でコスト関数値が十分に収束することを実験的に確認したため,全ての実験において100回で統一した.その他の実験条件はIVA及びILRMAと同様である.


%----------------------------------------------
\section{実験結果}
\label{sec:conv:expresult5}
%----------------------------------------------

%----------------------------------------------
\subsection{フィルタを適用した観測信号}
\label{sec:conv:resultiva}
%----------------------------------------------
Fig. \ref{fig:sfiva64obs}は\ref{sec:conv:filteroutline}節のフィルタを前処理として適用したBack no.~2の観測信号である.1.5~Hz以下の信号がフィルタによって落とされているが,依然としてパワーの小さい領域において, 呼吸信号がみられる0.4~Hzの周波数成分及び振動台による振動の信号が卓越して残留している.

%-%-%-%-%-%-%-%-%
\begin{figure}[tb]
\centering
\includegraphics[width=1.0\hsize]{./ch_conventional/fig/spect_iva_filter_64_obs.pdf}
\caption{Spectrograms of high-pass-filtered observed signal (Back no.~2).}
\label{fig:sfiva64obs}
\end{figure}
%-%-%-%-%-%-%-%-%

%----------------------------------------------
\subsection{IVAを適用した結果}
\label{sec:conv:resultiva}
%----------------------------------------------
Fig. \ref{fig:sfiva64est}はFig. \ref{fig:sfiva64obs}にIVAを適用した結果である.
\textcolor{black}{前章のFig. \ref{fig:sfiva64obs}で見られた3.5~Hz付近の振動の3次高調波がほとんど見られず,心拍の高調波が3.5~Hz,5~Hz,及び6~Hz付近に強く現れている.1番目,2番目,及び4番目のスペクトログラムには心拍の高調波成分が見られず,振動及び呼吸の高調波成分が分離されていることが分かる.}

また,Fig. \ref{fig:sfiva64est}に対して心拍推定アルゴリズムを適用した心拍推定値をFig. \ref{fig:fhriva64ch3}に示す.前章のFig.~\ref{fig:hrivawinlen}~(b)の150~sから360~sでは合致していなかった参照値と合致していることが分かる.但し,200~s付近の推定心拍値は参照値と合致していない.この結果より,前処理を適用することで心拍推定精度が向上するといえる.しかし,加振時に参照値と合致していない箇所がまだ多く存在するため,次節以降で,より心拍推定精度が高くなるBSSの手法について調査する.

%-%-%-%-%-%-%-%-%
\begin{figure}[tb]
\centering
\includegraphics[width=1.0\hsize]{./ch_conventional/fig/spect_iva_filter_64_est.pdf}
\caption{Spectrograms of separated signal obtained by IVA (Back no.~2, high-pass-filtered).}
\label{fig:sfiva64est}
\end{figure}
%-%-%-%-%-%-%-%-%

%-%-%-%-%-%-%-%-%
\begin{figure}[tb]
\centering
\includegraphics[width=70mm]{./ch_conventional/hrfig/hr_iva_filter_64_ch3.pdf}
  \caption{Third Estimated (red) and reference (blue) heart rates obtained by IVA (Back no.~2, high-pass-filtered).}
  \label{fig:fhriva64ch3}
\end{figure}
%-%-%-%-%-%-%-%-%


%----------------------------------------------
\subsection{基底数固定型ILRMAを適用した結果}
\label{sec:conv:resultilrma1}
%----------------------------------------------
Fig. \ref{fig:silrma1}はFig. \ref{fig:sfiva64obs}に基底数固定型ILRMAを適用した推定信号である.IVAの分離結果であるFig. \ref{fig:sfiva64est}では,1番目,2番目,及び4番目に心拍成分以外の成分が分離されていたが,Fig. \ref{fig:silrma1}では,振動成分が1番目及び4番目のスペクトログラムに分離され,2番目のスペクトログラムの4.5~Hz及び5.5~Hz付近に心拍成分がみられる.ここで,Fig. \ref{fig:sfiva64est}及びFig. \ref{fig:silrma1}の3番目のスペクトログラムを比較すると心拍成分のパワーに差がみられないことから,Fig. \ref{fig:sfiva64est}の3番目以外のスペクトログラムにも心拍成分が混在していたと考えられる.

また,Fig. \ref{fig:silrma1}に対して心拍推定アルゴリズムを適用した心拍推定値をFig. \ref{fig:hrilrma1}に示す.IVAを用いた際の推定心拍であるFig. \ref{fig:fhriva64ch3}では60~sから150~s及び360~sから400~sの推定心拍値が参照値と合致していなかったが,基底数固定型ILRMAでは60~sから120~s以外の時間で参照値とよく合致している.不一致の時間に関しては振動台の信号が加えられた初期であり,大きく体が動いたことが参照値と外れた原因と考えられる.

%-%-%-%-%-%-%-%-%
\begin{figure}[tb]
\centering
\includegraphics[width=1.0\hsize]{./ch_conventional/fig/spect_ILRMA1_64_est.pdf}
\caption{Spectrograms of separated signal obtained by ILRMA without partitioning function (Back no.~2, high-pass-filtered).}
\label{fig:silrma1}
\end{figure}
%-%-%-%-%-%-%-%-%

%-%-%-%-%-%-%-%-%
\begin{figure}[tb]
\centering
\includegraphics[width=70mm]{./ch_conventional/hrfig/hr_ILRMAtype1_64_ch3.pdf}
\caption{Third Estimated (red) and reference (blue) heart rates obtained by ILRMA without partitioning function (Back no.~2, high-pass-filtered).}
\label{fig:hrilrma1}
\end{figure}
%-%-%-%-%-%-%-%-%

%----------------------------------------------
\subsection{基底数可変型ILRMAを適用した結果}
\label{sec:conv:resultilrma2}
%----------------------------------------------
Fig. \ref{fig:silrma2}はFig. \ref{fig:sfiva64obs}に基底数可変型ILRMAを適用した推定信号である.前節の基底数固定型ILRMAの結果であるFig.~\ref{fig:silrma1}と非常に類似しており,スペクトログラムについては大きな差異は確認できない.

また,Fig. \ref{fig:silrma2}に対して心拍推定アルゴリズムを適用した心拍推定値をFig. \ref{fig:hrilrma2}に示す.Fig. \ref{fig:hrilrma1}と比較してFig. \ref{fig:hrilrma2}では60~s及び120~sにおいて,一時的にリファレンスデータを捉えられていないが,その他の時間においては概ね参照値と合致している.

%-%-%-%-%-%-%-%-%
\begin{figure}[tb]
\centering
\includegraphics[width=1.0\hsize]{./ch_conventional/fig/spect_ILRMA2_64_est.pdf}
\caption{Spectrograms of separated signal obtained by ILRMA with partitioning function (Back no.~2, high-pass-filtered).}
\label{fig:silrma2}
\end{figure}
%-%-%-%-%-%-%-%-%

%-%-%-%-%-%-%-%-%
\begin{figure}[tb]
\centering
\includegraphics[width=70mm]{./ch_conventional/hrfig/hr_ILRMAtype2_64_ch3.pdf}
\caption{Third Estimated (red) and reference (blue) heart rates obtained by ILRMA with partitioning function (Back no.~2, high-pass-filtered).}
\label{fig:hrilrma2}
\end{figure}
%-%-%-%-%-%-%-%-%

%----------------------------------------------
\subsection{$t$-ILRMAを適用した結果}
\label{sec:conv:resulttilrma}
%----------------------------------------------
\textcolor{black}{Figs. \ref{fig:stilrmaa5}及びは\ref{fig:stilrmap5}はそれぞれ$p=1$, $\nu = 5$及び$p=2$, $\nu = 5$でFig. \ref{fig:sfiva64obs}に$t$-ILRMAを適用した推定信号である.前節のFigs. \ref{fig:silrma1}及び\ref{fig:silrma2}の2番目のスペクトログラムの4.5~Hz及び5.5~Hz付近に見られた心拍成分がFigs. \ref{fig:stilrmaa5}及びは\ref{fig:stilrmap5}では見られない.さらに,Figs. \ref{fig:stilrmaa5}及び\ref{fig:stilrmap5}の3番目のスペクトログラムに見られる心拍成分のパワーがFigs. \ref{fig:sfiva64est}, \ref{fig:silrma1}及び\ref{fig:silrma2}に比べて僅かではあるが強くなっていることが分かる.また,Figs. \ref{fig:sfiva64est}, \ref{fig:silrma1}及び\ref{fig:silrma2}の3番目のスペクトログラムの3.5~Hzに見られた振動の高調波成分がFigs. \ref{fig:stilrmaa5}及びは\ref{fig:stilrmap5}では見られない.}

\textcolor{black}{Figs. \ref{fig:stilrmaa5}及びは\ref{fig:stilrmap5}に対して心拍推定アルゴリズムを適用した心拍推定値をFigs. \ref{fig:hrtilrmaa5}及び\ref{fig:hrtilrmap5}に示す.Figs. \ref{fig:hrtilrmaa5}及び\ref{fig:hrtilrmap5}より,NMFモデルの信号領域の違いによる心拍推定精度の差はほとんどない.Figs. \ref{fig:fhriva64ch3}, \ref{fig:hrilrma1}, 及び\ref{fig:hrilrma2}よりも心拍推定精度が良いことが分かる.振動が加えられた60秒付近でのみ参照値と合致していない.これは,振動が加えられた瞬間の体動は大きくなるためであると考えられる.}

%-%-%-%-%-%-%-%-%
\begin{figure}[tb]
\centering
\includegraphics[width=1.0\hsize]{./ch_conventional/fig/spect_tILRMA_ampdom_dofp5_64_est.pdf}
\caption{Spectrograms of separated signal obtained by $t$-ILRMA, where $p=1$ and $\nu=5$ (Back no.~2, high-pass-filtered).}
\label{fig:stilrmaa5}
\end{figure}
%-%-%-%-%-%-%-%-%

%-%-%-%-%-%-%-%-%
\begin{figure}[tb]
\centering
\includegraphics[width=1.0\hsize]{./ch_conventional/fig/spect_tILRMA_powdom_dofp5_64_est.pdf}
\caption{Spectrograms of separated signal obtained by $t$-ILRMA, where $p=2$ and $\nu=5$ (Back no.~2, high-pass-filtered).}
\label{fig:stilrmap5}
\end{figure}
%-%-%-%-%-%-%-%-%

%-%-%-%-%-%-%-%-%
\begin{figure}[tb]
\centering
\includegraphics[width=70mm]{./ch_conventional/hrfig/hr_tILRMA_ampdom_dofp5_64_ch3.pdf}
\caption{Third Estimated (red) and reference (blue) heart rates obtained by $t$-ILRMA, where $p=1$ and $\nu=5$ (Back no.~2, high-pass-filtered).}
\label{fig:hrtilrmaa5}
\end{figure}
%-%-%-%-%-%-%-%-%

%-%-%-%-%-%-%-%-%
\begin{figure}[tb]
\centering
\includegraphics[width=70mm]{./ch_conventional/hrfig/hr_tILRMA_powdom_dofp5_64_ch3.pdf}
\caption{Third Estimated (red) and reference (blue) heart rates obtained by $t$-ILRMA, where $p=2$ and $\nu=5$ (Back no.~2, high-pass-filtered).}
\label{fig:hrtilrmap5}
\end{figure}
%-%-%-%-%-%-%-%-%

%----------------------------------------------
\section{本章のまとめ}
\label{sec:conv:conclusion5}
%----------------------------------------------
\textcolor{black}{本章では,前章の実験結果より,観測信号に含まれる呼吸による体動に起因するノイズ信号が周波数領域BSSであまり除去されていないことから,このノイズ信号を除去するために,カットオフ周波数を1.5~Hzとする170次FIRディジタルハイパスフィルタを設計し,観測信号に適用した.前処理を適用した観測信号にIVA, ILRMA(基底数固定型及び基底数可変型を含む),及び$t$-ILRMAを適用し,得られたスペクトログラムから実験結果の解説を行った.また,各分離信号に心拍推定アルゴリズムを適用し,参照値に対してどの程度の心拍推定精度が得られたかを確認した.心拍推定アルゴリズムを適用した結果,$t$=ILRMAが最も参照値と合致していた.$t$-ILRMAのNMFモデルの信号領域による違いは見られなかった.}

%%%%% 第3章 %%%%%
\chapter{結言}
\label{chap:con}

本論文では,高精度な心拍推定が可能であるBSSの手法について調査した.

\ref{chap:measurementstructsig}章では,測定条件と測定で得られる信号について述べた.

\ref{chap:methods}章では,BSSの定式化を行い,BSSの代表的な手法であるIVAやILRMAおよびILRMAに複素Student's $t$分布を生成モデルとして持たせた$t$-ILRMAの解説を行った.また心拍推定アルゴリズムの解説を行った.

\ref{chap:bsshrexp}章では,測定信号にIVAを適用し,分離制度が最も高い測定信号のチャネルとFFTサイズの調査を行った.

\ref{chap:fbsshrexp}章では,呼吸成分をカットするためのフィルタの解説を行った.また,設計したフィルタを前処理として測定信号に適用し,IVAや基底数固定ILRMA, 基底数可変型ILRMA,$t$-ILRMAの分離制度を比較した.

最後に今後の課題を述べる.\ref{chap:measurementstructsig}章で述べた測定信号は振動台によるsin波の短軸家臣であるため,実際の車の走行で得られる振動とは異なるものである.本論文では,雑音が混在した信号から生体信号を取り出すことが可能であることを明らかにした.今後は実測データに対しても同等以上の分離制度を持つ手法についての調査を行う.

%%%%%%%%%%%%%%%%%%%%%%%%%%% 後付 %%%%%%%%%%%%%%%%%%%%%%%%%%%
\backmatter

%%%%% 謝辞 %%%%%
\chapter{謝辞}

本論文は,香川高等専門学校電気情報工学科北村研究室にて行われた研究に基づくものです.

まず,本研究を進めるにあたり,ご多忙のところ熱心に
ご指導くださいました指導教員の北村大地講師に心より感謝申し上げます.
北村大地講師には,論文執筆や研究に関する議論など,細部にわたるまで
丁寧にご指導いただきました.また,北村大地講師のご趣味であるギターのお話をはじめとして,多くのディスカッションをしていただきました.
心よりありがたくお礼申し上げます.

本論文の副査である柿元健准教授には,論文の構成や記述に関して有益な助言を頂き大変お世話になりました.ここに厚く御礼申し上げます.

株式会社村田製作所の市ノ瀬吏様,石原尚様,植木大地様,齋藤諒様には,共同研究ミーティングを通じ,数多くの有益な
ご指摘のほか,様々なご支援をいただきました.
心より感謝申し上げます.

北村研究室の先輩である専攻科2年の大藪宗一郎氏,梶谷奈未氏には,
関連研究のご説明をはじめ,参考文献のご提供など数々のご支援をいただきました.
また,北村研究室の先輩である専攻科2年の岩瀬雄太佑太氏,渡辺瑠伊氏,同期の川口翔也氏・蓮池郁也氏・細谷泰稚氏,溝渕悠朔氏にはゼミや日頃のディスカッションのほか,プログラミングに際しての多くの助言をいただき,研究室生活を様々な面で支えていただきました.
ここに感謝申し上げます.

最後になりますが,現在に至るまで私の学生生活を金銭的,精神的に支え,
毎日早朝からお弁当を作り笑顔で見送ってくれたご両親には感謝の念に堪えません.
これまで本当にありがとうございました.


%%%%% 参考文献(直接書く場合) %%%%%
\addcontentsline{toc}{chapter}{\bibname} % 参考文献を目次に表示
\begin{thebibliography}{99}
    \bibitem {sensorreview}
  Y. Dong, Z. Hu, K. Uchimura, and N. Murayama, ``Driver inattention monitoring system for intelligent vehicles: A review,'' {\em IEEE Trans. Intelligent Transportation 
  Systems}, vol.           
  12, no. 2, pp. 596--614, 2011.

    \bibitem{system1}
  L. M. Bergasa, J. Nuevo, M. A. Sotelo, R. Barea, and M. E. Lopez, ``Real-time system for monitoring driver vigilance,'' {\em IEEE Trans. Intelligent Transportation 
  Systems}, vol. 7, no. 
  1, pp. 63--77, 2006.

    \bibitem{system2}
  V. E. Dahiphale and S. R. Rao, ``A review paper on portable driver monitoring system for real time fatigue,'' {\em Proc. International Conference on Computing 
  Communication 
  Control and Automation}, pp. 558--560, 2015.

    \bibitem{system3}
  L. Ma\v{s}anovi\'{c}, M. Vranje\v{s}, R. D\v{z}akula, and \v{Z}. Luka\v{c}, ``Driver monitoring using the in-vehicle camera,'' {\em Proc. Zooming Innovation in 
  Consumer Technologies Conference}, pp. 
  33--38, 2019.

    \bibitem{contactsensor1}
  J. A. Healey and R. W. Picard, ``Detecting stress during real-world driving tasks using physiological sensors,'' {\em IEEE Trans. Intelligent Transportation Systems}, 
  vol. 6, no. 2, pp. 156--166, 2005.

    \bibitem{contactsensor2}
  M. Patel, S. K. L. Lal, D. Kavanagh, and P. Rossiter, ``Applying neural network analysis on heart rate variability data to assess driver fatigue,'' {\em Expert Systems 
  with Applications}, vol. 38, no. 6, pp. 7235--7242, 2011.

    \bibitem{eazyap1}
  M. Osaka, H. Murata, Y. Fuwamoto, S. Nanba, K. Sakai, and T. Katoh, ``Application of heart rate variability analysis to electrocardiogram recorded outside the driver's 
  awareness from an automobile steering wheel,'' {\em Circulation Journal}, vol. 72, no. 11, pp. 1867--1873, 2008.

    \bibitem{eazyap2}
  R. Cassani, A. Horai, L. A. Gheorghe, and T. H. Falk, ``Predicting driver stress levels with a sensor-equipped steering wheel and a quality-aware heart rate 
  measurement algorithm,'' {\em Proc. Annual International Conference of the IEEE Engineering Medicine and Biology Society}, pp. 6822--6825, 2021.

    \bibitem{eazyap3}
  B. Babusiak, A. Hajducik, S. Medvecky, M. Lukac, and J. Klarak, ``Design of smart steering wheel for unobtrusive health and drowsiness monitoring,'' {\em Sensors}, 
  vol. 21, no. 16, 5285, 2021.

    \bibitem{originica}
  P. Comon, "Independent component analysis, A new concept?," {\em Signal Processing}, vol. 36, no. 3, pp.287--314, 1994.

    \bibitem{ica2}
  H. Sawada, N. Ono, H. Kameoka, D. Kitamura, and H. Saruwatari, ``A review of
  blind source separation methods: Two converging routes to ILRMA originating from
  ICA and NMF,'' {\em APSIPA Trans. Signal and Information Processing}, vol. 8, no. e12,
  pp. 1–14, 2019.

    \bibitem{NMF}
  D.~D.~Lee and H.~S.~Seung,
  ``Learning the parts of objects by non-negative matrix factorization,''
  {\em Nature}, vol.~401, no.~6755, pp.~788--791, 1999.
  
  \bibitem{singlechsep}
  P.~Smaragdis, B.~Raj, and M.~Shashanka, 
  ``Supervised and semi-supervised separation of sounds from single-channel mixtures,''  
  {\em Proc. International Conference on Independent Component Analysis and Signal Separation}, no. 1, 2007, pp. 414–421.
  
  \bibitem{supNMF}
  D.~Kitamura, H.~Saruwatari, Y.~Kosuke, K.~Shikano, Y.~Takahashi, and K.~Kondo, 
  ``Music signal separation based on supervised nonnegative matrix factorization with orthogonality and maximum-divergence penalties,'' 
  {\em IEICE Trans. Fundamentals}, vol. 97-A, no. 5, pp. 1113–1118, 2014.
  
  \bibitem{MNMF_oz}
  A.~Ozerov and C.~Févotte,
  ``Multichannel nonnegative matrix factorization in convolutive mixtures for audio source separation,'' 
  {\em IEEE Trans. Audio, Speech, and Language Processing,} vol.~18, no.~3, pp. 550--563, 2010.
  
  \bibitem{bss_review}
  H.~Sawada, N.~Ono, H.~Kameoka, D.~Kitamura, and H.~Saruwatari, ``A review of blind source separation methods: Two converging routes to ILRMA originating from ICA and NMF,'' 
  {\em APSIPA Trans. Signal and Information Processing,} vol.~8, no.~e12, pp.~1--14, 2019.

  \bibitem{Kim2007_iva}
  T.~Kim, H.~T.~Attias, S.-Y.~Lee, and T.-W.~Lee, ``Blind source separation exploiting higher-order frequency dependencies,'' 
  {\em IEEE Trans. Audio, Speech, and Language Processing}, vol.~15, no.~1, pp.~70--79, 2007.

  \bibitem{auxIVA}
  N.~Ono and S.~Miyabe, ``Auxiliary-function-based independent component analysis for super-Gaussian sources,'' 
  {\em Proc. International Conference on Latent Variable Analysis and Signal Separation,} pp.165--172, 2010.
    
  \bibitem{ILRMA}
  D.~Kitamura, N.~Ono, H.~Sawada, H.~Kameoka, and H.~Saruwatari,
  ``Determined blind source separation unifying independent vector analysis and nonnegative matrix factorization,'' 
  {\em IEE/ACM Trans. Audio, Speech, and Language Processing}, vol.~24, no.~9, pp.~1626--1641, 2016.

  \bibitem{Kitamura2018_ilrma}
  D.~Kitamura, N.~Ono, H.~Sawada, H.~Kameoka, and H.~Saruwatari, ``Determined blind source separation with independent low-rank matrix analysis,'' 
  {\em Audio Source Separation}, S.~Makino, Ed., pp. 125--155. Springer, Cham, 2018.

  \bibitem{PB}
  K.~Matsuoka and S.~Nakashima,
  ``Minimal distortion principle for blind source separation,''
  {\em Proc. SICE Annual Conference}, pp.~722--727, 2001.
  
  \bibitem{stable_auxIVA}
  N.~Ono,
  ``Stable and fast update rules for independent vector analysis based on auxiliary function technique,''
  {\em Proc. IEEE Workshop on Applications of Signal Processing to Audio and Acoustics}, pp.189--192, 2011.
  
%   \bibitem{undnmf}
%   S.~Arberet, A.~Ozerov, N.Q.~Duong, E.~Vincent, R.~Gribonval, F.~Bimbot, P.~Vandergheynst,
%   ``Nonnegative matrix factorization and spatial covariance model for under-determined reverberant audio source separation,''
%   in {\em 10th International Conference on Information Sciences Signal Processing and their Applications (ISSPA)}, pp.~1--4, 2010.
  
  \bibitem{psi-div}
  S.~Eguchi and Y.~Kano, 
  ``Robustifing maximum likelihood estimation by psi- divergence,'' 
  {\em ISM Research Memorandum}, 2001.
  
  \bibitem{ext_nmf}
  A.~Cichocki, S.~Amari, R.~Zdunek, R.~Kompass, G.~Hori, and Z.~He, 
  ``Extended SMART algorithms for non-negative matrix factorization,'' 
  {\em Proc. International Conference on Artificial Intelligence and Soft Computing}, pp.~548--562, 2006.
  
  \bibitem{pssnmf}
  D.~Kitamura, N.~Ono, H.~Saruwatari, Y.~Takahashi, and K.~Kondo, 
  ``Effective basis learning for sound source separation by semi-supervised nonnegative matrix factorization,'' 
  {\em IEICE Technical Report}, EA2015--130, vol.~115, no.~521, pp.~355--360, 2016.

  \bibitem{isnmf}
  C.~Févotte, N.~Bertin, J.-L.~Durrieu, 
  ``Nonnegative matrix factorization with the Itakura-Saito divergence. With application to music analysis.'' 
  {\em Neural Computation}, vol.~21, 793--830, 2009
  
  \bibitem{MU}
  M.~Nakano, H.~Kameoka, J.~Le~Roux, Y.~Kitano, N.~Ono, S.~Sagayama, ``Convergence-guaranteed multiplicative algorithms for nonnegative matrix factorization with beta-divergence,''
  {\em Proc. IEEE International Workshop on Machine Learning for Signal Processing}, 2010, pp.~283--288.
  
  \bibitem{auxfunc}
  D.~R.~Hunter and K.~Lange, ``A tutorial on MM algorithms,'' 
  {\em The American Statistician,} vol.~58, no.~1, pp.~30-–37, 2004.
  
  \bibitem{paramilrma}
  Y.~Mitsui, D.~Kitamura, N.~Takamune, H.~Saruwatari, Y.~Saruwatari and K.~Kondo, ``Independent low-rank matrix analysis based on parametric majorization-equalization algorithm,'' 
  {\em Proc. IEEE International Workshop on Computational Adv  

  \bibitem{mm}
  C.~Févotte and J.~Idier, 
  ``Algorithms for nonnegative matrix factorization with the $\beta$-divergence,'' 
  {\em Neural Computation}, vol.~23, no.~9, pp.~2421--2456, 2011.

  \bibitem{intBSS}
  大島風雅, 中野将生, 北村大地,ances in Multi-Sensor Adaptive Processing}, pp. 98--102, 2017.
``ユーザーからの補助情報を用いる独立低ランク行列分析''
  {\em 日本音響学会2020年秋季研究発表会講演論文集}, pp.~269--272, 2020.    

  \bibitem{SDR}
  E.~Vincent, R.~Gribonval, and C.~F\'evotte, ``Performance measurement in blind audio source separation,'' 
  {\em IEEE Trans. Audio, Speech, and Language Processing}, vol.~14, no.~4, pp.~1462--1469, 2006.
  
  \bibitem{sisec}
  S.~Araki, A.~Ozerov, V.~Gowreesunker, H.~Sawada, F.~Theis, G.~Nolte, D.~Lutter, and N.~Q.~K.~Duong, 
  ``The 2010 signal separation evaluation campaign (SiSEC2010): Audio source separation,'' 
  {\em Proc. International Conference on Latent Variable Analysis and Signal Separation}, 2010, pp. 114–122.
  
  \bibitem{E2A}
  S.~Nakamura, K.~Hiyane, F.~Asano, T.~Nishiura, and T.~Yamada, 
  ``Acoustical sound database in real environments for sound scene understanding and hands-free speech recognition,'' 
  {\em Proc. International Conference on Language Resources and Evaluation}, 2000, pp. 965–968.

  \bibitem{tnmf}
  K.~Yoshii, K.~Itoyama, and M.~Goto, ``Student's $t$ nonnegative matrix factorization and positive semidefinite tensor    
  factorization for single-channel audio source separation,''
  {\em Proc.~ICASSP}, pp.51-55, 2016.

  \bibitem{cauchynmf}
  A.~Liutkus, D>~FitsGerald, and R>~Badeau, ``Cauchy nonnegative matrix factorization,''
  {\em Proc.~WASPAA}, 2015.

\end{thebibliography}
%%%%% 参考文献(BibTeXを使う場合) %%%%%
% \addcontentsline{toc}{chapter}{\bibname} % 参考文献を目次に表示
% \bibliography{ref_abb_full,references}

%%%%% 発表文献一覧 %%%%%
%%-%-%-%-%-%-%-%-%
\begin{figure}[tb]
\centering
\includegraphics[width=1.0\hsize]{./ch_conventional/fig/obsSpect.pdf}
\caption{Spectrogram obtained by observation signal..}
\label{fig:obsspect}
\end{figure}
%-%-%-%-%-%-%-%-%\input{publication/publication.tex}

%%%%%%%%%%%%%%%%%%%%%%%%%%% 付録 %%%%%%%%%%%%%%%%%%%%%%%%%%%
\appendix

%%%%% 付録A %%%%%
%\chapter{その他の観測信号に本研究で用いた手法を適用した実験結果}
\label{chap:ineq}

以下に,\ref{chap:bsshrexp}章及び\ref{chap:fbsshrexp}章で掲載していなかった,Back No.~4,Back No.~6,Bottom No.~1,Bottom No.~3,Bottom No.~5の実験結果を示す.但し,基底数可変型ILRMA及び$t$-ILRMAについてはBack No.~2についてのみ実験を行ったため掲載していない.また,以下に示すスペクトログラムのSTFTの窓長は64点(1.6~s)で計算を行い,心拍推定グラフは全ての分離信号の3番目に対して心拍推定アルゴリズムを適用したものである.

%-%-%-%-%-%-%-%-%
\begin{figure}[tb]
\centering
\includegraphics[width=1.0\hsize]{./appendix_aux/Back/no4siva.pdf}
\caption{Spectrograms of separated signals obtained by IVA (Back no.~4).}
\end{figure}
%-%-%-%-%-%-%-%-%

%-%-%-%-%-%-%-%-%
\begin{figure}[tb]
\centering
\includegraphics[width=1.0\hsize]{./appendix_aux/Back/no6siva.pdf}
\caption{Spectrograms of separated signals obtained by IVA (Back no.~6).}
\end{figure}
%-%-%-%-%-%-%-%-%

%-%-%-%-%-%-%-%-%
\begin{figure}[tb]
\centering
\includegraphics[width=1.0\hsize]{./appendix_aux/Bottom/no1siva.pdf}
\caption{Spectrograms of separated signals obtained by IVA (Bottom no.~1).}
\end{figure}
%-%-%-%-%-%-%-%-%

%-%-%-%-%-%-%-%-%
\begin{figure}[tb]
\centering
\includegraphics[width=1.0\hsize]{./appendix_aux/Bottom/no3siva.pdf}
\caption{Spectrograms of separated signals obtained by IVA (Bottom no.~3).}
\end{figure}
%-%-%-%-%-%-%-%-%

%-%-%-%-%-%-%-%-%
\begin{figure}[tb]
\centering
\includegraphics[width=1.0\hsize]{./appendix_aux/Bottom/no5siva.pdf}
\caption{Spectrograms of separated signals obtained by IVA (Back no.~5).}
\end{figure}
%-%-%-%-%-%-%-%-%

%-%-%-%-%-%-%-%-%
\begin{figure}[tb]
\centering
\includegraphics[width=1.0\hsize]{./appendix_aux/Back/no4sfiva.pdf}
\caption{Spectrograms of separated signals obtained by IVA (Back no.~4, high-pass-filtered).}
\end{figure}
%-%-%-%-%-%-%-%-%

%-%-%-%-%-%-%-%-%
\begin{figure}[tb]
\centering
\includegraphics[width=1.0\hsize]{./appendix_aux/Back/no6sfiva.pdf}
\caption{Spectrograms of separated signals obtained by IVA (Back no.~6, high-pass-filtered).}
\end{figure}
%-%-%-%-%-%-%-%-%

%-%-%-%-%-%-%-%-%
\begin{figure}[tb]
\centering
\includegraphics[width=1.0\hsize]{./appendix_aux/Bottom/no1sfiva.pdf}
\caption{Spectrograms of separated signals obtained by IVA (Bottom no.~1, high-pass-filtered).}
\end{figure}
%-%-%-%-%-%-%-%-%

%-%-%-%-%-%-%-%-%
\begin{figure}[tb]
\centering
\includegraphics[width=1.0\hsize]{./appendix_aux/Bottom/no3sfiva.pdf}
\caption{Spectrograms of separated signals obtained by IVA (Bottom no.~3, high-pass-filtered).}
\end{figure}
%-%-%-%-%-%-%-%-%

%-%-%-%-%-%-%-%-%
\begin{figure}[tb]
\centering
\includegraphics[width=1.0\hsize]{./appendix_aux/Bottom/no5sfiva.pdf}
\caption{Spectrograms of separated signals obtained by IVA (Bottom no.~5, high-pass-filtered).}
\end{figure}
%-%-%-%-%-%-%-%-%

%-%-%-%-%-%-%-%-%
%-%-%-%-%-%-%-%-%
\begin{figure}[tb]
\centering
\includegraphics[width=70mm]{./appendix_aux/hr/Back/no4iva.pdf}
  \caption{Third Estimated (red) and reference (blue) heart rates obtained by IVA (Back no.~4).}
\end{figure}
%-%-%-%-%-%-%-%-%

%-%-%-%-%-%-%-%-%
\begin{figure}[tb]
\centering
\includegraphics[width=70mm]{./appendix_aux/hr/Back/no6iva.pdf}
  \caption{Third Estimated (red) and reference (blue) heart rates obtained by IVA (Back no.~6).}
\end{figure}
%-%-%-%-%-%-%-%-%

%-%-%-%-%-%-%-%-%
\begin{figure}[tb]
\centering
\includegraphics[width=70mm]{./appendix_aux/hr/Bottom/no1iva.pdf}
  \caption{Third Estimated (red) and reference (blue) heart rates obtained by IVA (Bottom no.~1).}
\end{figure}
%-%-%-%-%-%-%-%-%

%-%-%-%-%-%-%-%-%
\begin{figure}[tb]
\centering
\includegraphics[width=70mm]{./appendix_aux/hr/Bottom/no3iva.pdf}
  \caption{Third Estimated (red) and reference (blue) heart rates obtained by IVA (Bottom no.~3).}
\end{figure}
%-%-%-%-%-%-%-%-%

%-%-%-%-%-%-%-%-%
\begin{figure}[tb]
\centering
\includegraphics[width=70mm]{./appendix_aux/hr/Bottom/no5iva.pdf}
  \caption{Third Estimated (red) and reference (blue) heart rates obtained by IVA (Bottom no.~5).}
\end{figure}
%-%-%-%-%-%-%-%-%

%-%-%-%-%-%-%-%-%
\begin{figure}[tb]
\centering
\includegraphics[width=70mm]{./appendix_aux/hr/Back/no4fiva.pdf}
  \caption{Third Estimated (red) and reference (blue) heart rates obtained by IVA (Back no.~4, high-pass-filtered).}
\end{figure}
%-%-%-%-%-%-%-%-%

%-%-%-%-%-%-%-%-%
\begin{figure}[tb]
\centering
\includegraphics[width=70mm]{./appendix_aux/hr/Back/no6fiva.pdf}
  \caption{Third Estimated (red) and reference (blue) heart rates obtained by IVA (Back no.~6, high-pass-filtered).}
\end{figure}
%-%-%-%-%-%-%-%-%

%-%-%-%-%-%-%-%-%
\begin{figure}[tb]
\centering
\includegraphics[width=70mm]{./appendix_aux/hr/Bottom/no1fiva.pdf}
  \caption{Third estimated (red) and reference (blue) heart rates of Bottom no.~1 obtained by IVA with filter.}
\end{figure}
%-%-%-%-%-%-%-%-%

%-%-%-%-%-%-%-%-%
\begin{figure}[tb]
\centering
\includegraphics[width=70mm]{./appendix_aux/hr/Bottom/no3fiva.pdf}
  \caption{Third Estimated (red) and reference (blue) heart rates obtained by IVA (Bottom no.~3, high-pass-filtered).}
\end{figure}
%-%-%-%-%-%-%-%-%

%-%-%-%-%-%-%-%-%
\begin{figure}[tb]
\centering
\includegraphics[width=70mm]{./appendix_aux/hr/Bottom/no5fiva.pdf}
  \caption{Third Estimated (red) and reference (blue) heart rates obtained by IVA (Bottom no.~5, high-pass-filtered).}
\end{figure}
%-%-%-%-%-%-%-%-%

%-%-%-%-%-%-%-%-%
\begin{figure}[tb]
\centering
\includegraphics[width=1.0\hsize]{./appendix_aux/Back/no4silrma1.pdf}
\caption{Spectrograms of separated signal obtained by ILRMA without partitioning function (Back no.~4, high-pass-filtered).}
\end{figure}
%-%-%-%-%-%-%-%-%

%-%-%-%-%-%-%-%-%
\begin{figure}[tb]
\centering
\includegraphics[width=1.0\hsize]{./appendix_aux/Back/no6silrma1.pdf}
\caption{Spectrograms of separated signal obtained by ILRMA without partitioning function (Back no.~6, high-pass-filtered).}
\end{figure}
%-%-%-%-%-%-%-%-%

%-%-%-%-%-%-%-%-%
\begin{figure}[tb]
\centering
\includegraphics[width=1.0\hsize]{./appendix_aux/Bottom/no1silrma1.pdf}
\caption{Spectrograms of separated signal obtained by ILRMA without partitioning function (Bottom no.~1, high-pass-filtered).}
\end{figure}
%-%-%-%-%-%-%-%-%

%-%-%-%-%-%-%-%-%
\begin{figure}[tb]
\centering
\includegraphics[width=1.0\hsize]{./appendix_aux/Bottom/no3silrma1.pdf}
\caption{Spectrograms of separated signal obtained by ILRMA without partitioning function (Bottom no.~3, high-pass-filtered).}
\end{figure}
%-%-%-%-%-%-%-%-%

%-%-%-%-%-%-%-%-%
\begin{figure}[tb]
\centering
\includegraphics[width=1.0\hsize]{./appendix_aux/Bottom/no5silrma1.pdf}
\caption{Spectrograms of separated signal obtained by ILRMA without partitioning function (Bottom no.~5, high-pass-filtered).}
\end{figure}
%-%-%-%-%-%-%-%-%

%-%-%-%-%-%-%-%-%
%-%-%-%-%-%-%-%-%
\begin{figure}[tb]
\centering
\includegraphics[width=70mm]{./appendix_aux/hr/Back/no4ilrma1.pdf}
\caption{Third Estimated (red) and reference (blue) heart rates obtained by ILRMA without partitioning function (Back no.~4, high-pass-filtered).}
\end{figure}
%-%-%-%-%-%-%-%-%

%-%-%-%-%-%-%-%-%
\begin{figure}[tb]
\centering
\includegraphics[width=70mm]{./appendix_aux/hr/Back/no6ilrma1.pdf}
\caption{Third Estimated (red) and reference (blue) heart rates obtained by ILRMA without partitioning function (Back no.~6, high-pass-filtered).}
\end{figure}
%-%-%-%-%-%-%-%-%

%-%-%-%-%-%-%-%-%
\begin{figure}[tb]
\centering
\includegraphics[width=70mm]{./appendix_aux/hr/Bottom/no1ilrma1.pdf}
\caption{Third Estimated (red) and reference (blue) heart rates obtained by ILRMA without partitioning function (Bottom no.~1, high-pass-filtered).}
\end{figure}
%-%-%-%-%-%-%-%-%

%-%-%-%-%-%-%-%-%
\begin{figure}[tb]
\centering
\includegraphics[width=70mm]{./appendix_aux/hr/Bottom/no3ilrma1.pdf}
\caption{Third Estimated (red) and reference (blue) heart rates obtained by ILRMA without partitioning function (Bottom no.~3, high-pass-filtered).}
\end{figure}
%-%-%-%-%-%-%-%-%

%-%-%-%-%-%-%-%-%
\begin{figure}[tb]
\centering
\includegraphics[width=70mm]{./appendix_aux/hr/Bottom/no5ilrma1.pdf}
\caption{Third Estimated (red) and reference (blue) heart rates obtained by ILRMA without partitioning function (Bottom no.~5, high-pass-filtered).}
\end{figure}
%-%-%-%-%-%-%-%-%


\end{document}
